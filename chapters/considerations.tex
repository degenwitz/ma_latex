\chapter{Final Considerations}
\label{chap:considerations}

%Regarding Final Considerations:
% I. Some teachers and/or methods use the term Conclusion for a text at the end of the paper that aims to expose the results achieved, this term is not incorrect, but many of the works are bibliographical reviews where in the end no conclusion is obtained and yes Several considerations that were found in the development of written work. Therefore, in each project should be considered/weighted if there will be a Conclusion or Final Considerations. Usually what else happens is that you have Final Considerations. The Final Considerations of a paper aims to show if the goal sought for the project was achieved, as well as give a view of the most important considerations and conclusions on the subject addressed, among other aspects. This should include:
%1. An explanation stating clearly whether or not it has achieved the stated objectives (a subdivision between general objectives and specific objectives can also be made here). In each case the reasons must be explained:
%The. If you have achieved the objectives: inform the main factors that contributed to the success, describing them briefly, but do not leave doubts;
%B. If you have not met the objectives: inform how much of the objective has been achieved and cite the factors that contributed to the failure, describing them briefly, but that leaves no doubt.
%2. Describe the main considerations and conclusions that were obtained as a result of the execution of the work. Here should not be repeated text already in the work, but write the impressions of these considerations and how they contributed to the implementation and achieved the goal;
%3. Name and describe the main difficulties encountered in the execution of the work and project. All the work developed means an evolution for the student, and to reach this evolution, it has had to overcome a series of obstacles. Reporting obstacles and overcoming (or not overcoming) helps to dignify and show the merit of the work itself to the reader/evaluator. It is also a contribution, in the sense that once problems and solutions are exposed, readers/evaluators learn/know ways of solving or approaching such problems;
%4. Discuss whether modifications occurred during the execution of the work within the scope defined in the Project phase and in what was developed. It should be explained what generated those modifications, substantiating and justifying such changes.
%5. The relationship between the proposed schedule and the work schedule can be described. This allows the reader/evaluator to learn from the indicated distortions/hits.
%6. Describe or cite future work that may be done based on this work. During the execution of work, it is sought to reach a defined objective in the project. However, several interesting subjects of research are revealed (being that the same ones are not treated/researched in the work because they do not match the objective and scope of the work). The description of such subjects/themes/research demonstrates the students' perception of development as well as their vision of objectivity in the execution of this work.

%II. Conclusion / Final Considerations aim to show the reader/evaluator the student's perception of the work done. In this way, it is not advisable to do citations and references because theoretically everything that was necessary to quote and refer should already be done within the content of the work. Only in some very specific cases/situations can you make referrals or citations in this part of the work (this should be discussed thoroughly with the supervisor/teacher).

%III. Looking at the described items that should compose the Conclusion / Final Considerations, it is difficult to imagine that this part of the work has less than one page;

\section{Summary}

% I did this in this way, that in that way... and so on

\section{Conclusions}

% Lessons learned

\section{Future Work}