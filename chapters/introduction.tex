\chapter{Introduction}
\label{introduction}


Artworks are sensitive to many external factors, such as humidity, temperature, and vibrations.
While these factors are generally well controlled in museums and storage, art pieces also need to be moved between these buildings.
Artworks include a variety of objects, ranging from photographs and paintings to the pieces needed for modern installations.
The transportation, therefore, needs to be secure and flexible.
Sensors play a critical role in the safe transportation of these objects.


\section{Motivation}

Certify is an international cooperative project between twelve partners in Switzerland and the EU %\citep{Brüschre von certify}.
One of those partners is the University of Zurich.
Its focus is on the development of Internet of Things (IoT) systems for security, monitoring, and detection %\citep{Brüschüre von certify}.
Next to certifications and the development of frameworks, Certify also concerns itself with the integration of IoT devices.


One of the multiple currently running pilots of Certify is the "Tracking and monitoring of artworks" %\citep{https://certify-project.eu/pilots/}.
The goal of this pilot is to enable the constant tracking and monitoring of artwork by attaching a device to it.
This device allows for unique identification using cryptographic methods.
It is also intended to act as a cluster of sensors that collect information about the artwork's surroundings that is relevant to the well-being of the artwork.
The goal is to have constant data on the artwork throughout its lifecycle.
This is intended to help secure the artwork and help with chain of custody monitoring.



\section{Thesis Goals}

This project aims to develop a system that implements a localized version of the artwork tracking envisioned by the Certify project, meant for transportation in a truck.
Additionally, the system will extend the Certify Projects goal by adding new detection methods and informing the truck driver about potential problems.

The goal is to develop a system that tracks the state of artwork in a truck using different detection methods.
The devices attached to the artworks, called tags, build a local decentralized network.
The driver's phone can query the network and display the collected metrics to them.
The system should alert the driver if a metric is outside the accepted norm.

This Thesis presents a proof of concept implementation.
The used metrics are not intended to be a complete representation of the sensors needed for security transport art.
Rather, they are intended to show different types of sensors that can be used.
This Thesis assumes that the data transfer to a server using 4G, as planned by the Certify project, will work and will not be implemented in this Thesis.



\section{Methodology}

This Thesis was made in four stages:

\subsubsection{Reserach}
In a first step, the basis of the Thesis had to be researched.
This involved familiarizing with existing research on artwork tracking, local IoT networks, and commonly used communication protocols.
Existing artwork tracking methods need to be analyzed and evaluated, considering their strength and shortcomings during transportation in a truck.
The types of sensors that could be relevant need to be chosen based on existing research, cost, and availability.
Options for the network architecture inside the truck needed to be researched and compared based on performance, stability, and security.
A communication protocol needed to be chosen based on the same criteria.

\subsubsection{Design}
Once the fundamental knowledge for the project had been acquired, the system had to be designed.
The design was chosen based on feasibility, security, and stability.

\subsubsection{Implementation}
The design was then implemented in a simplified manner based on the available material.
For this, four tags were built and equipped with sensors, communication capabilities, and a power supply.
Then, the required software was written, using existing implementations when possible and writing new code when required.
A simple example app was also developed, based on an existing communications app published by Nordic Semiconductors and installed on a phone.

\subsubsection{Evaluation}
The developed system of tags and phone was tested in a series of five experiments.
The first four experiments were intended to capture a specific part of the system, while the last was a general-purpose test.
The tests were performed in a manner that ensured minimal external influence.
The resulting data from the tests were analyzed using statistical methods.
The goal was to determine the system's reliability, find limitations, and look for improvements.

\section{Thesis Outline}

This Thesis is structured as follows:

Chapter Two presents the fundamental knowledge researched for this Thesis. 
Chapter Two also presents previous research on tracking artwork and sensor networks.
Chapter Three presents the design of the system and its inner workings and capabilities.
Chapter Four shows the implementation that was developed and used for this project.
Chapter Five describes the experiments performed. It presents the results of the experiments and discusses them.
Chapter Six summarizes the findings of this Thesis and discusses the most important aspects in a conclusion.


%\chapter{Introduction}
\label{introduction}


Artworks are sensitive to many externalfactors, such as humidity, temperature and vibrations.
While these factors are generally well controlled for in museums and storage, artpieces need also be moved between these buildings.
Artworks include a variety of objects, ranging from photographs and paintings the variety of pieces needed for modern installations.
The transportation therefore needs to be secure and flexible.
Sensors play a critical role in the safe transportation of these objects.


\section{Motivation}

Certify is an international cooperative project between twelve partners situated in Switzerland and the EU %\citep{Brüschre von certify}.
One of those partners is the university of zurich.
Its focus is on the development of Internet of Things (IoT) systems for security, monitoring and detection %\citep{Brüschüre von certify}.
Next to certifications and the development of frameworks, Certify also consirnes itself with the integration of IoT devices.


One of multiple currently running pilots of Certify is the "Tracking and monitoring of artworks" %\citep{https://certify-project.eu/pilots/}.
The goal of this pilot is to enable the constant tracking and monitoring of artworks by attaching a device to it.
This device allows for unique identification, by using cryptographic methods.
It is also intended to act as a cluster of sensors that collect information about the surounding of the artwork that are relevant to the wellbeeing of the artwork.
The goal is to have constant data on the artwork throughout its lifecycle.
This is intended to help with securing the artwork and helping with chain of custody monitoring.



\section{Thesis Goals}

The goal of this project is to develop a system that implements a localized version of the artwork tracking envisioned by the Certify project, ment for transportation in a truck.
Additionally the system will extend the Certify Projects goal by adding new detection methods and also informs the driver of the truck about potential problems.

The goal is to develop a system that tracks the state of artwork in a truck using different detection methods.
The devices attached to the artworks, called tags, build a local decentralized network.
The phone of the driver can query the network and displayed the collected metrics to them.
If a metric is outside of the accepted norm, the system should allert the driver.

This thesis presents a proof of concept implementation.
The used metrics are not intended to be a full representation of needed sensors to securley Transport art.
Rather they are intended to show different types of sensors that can be used.
This thesis assumes that the data transfer to a server using 4G, as planned by the Certify project, will work and will not implement it in this thesis.



\section{Methodology}

This Thesis was made in four stages:

\subsubsection{Reserach}
In a first step, the basis of the thesis had to be researched.
This involved familiarizing with existing research on the toping of artwork tracking, local IoT networks and commonly used communication protocols.
Existing artwork tracking methods need to be analyzed and evaluated, considering their strength and shortcommings during the transportation in a truck.
The types of sensors that could be relevant need to be chosen, based on existing research, cost and avaliablity.
Options for the network-architecture inside the truck needed to be researched and compared, based on performance, stability and security.
A communication protocol needed to be chosen, based on the same criterias.

\subsubsection{Design}
Once the fundamental knowledge for the project had been aquired, the system had to be designed.
The design was chosen based on feasability, security and stability.

\subsubsection{Implementation}
The desig then was implemented in a simplified manner based on the material that was avaliable.
For this four tags were build, equiped with sensors, communication-capabilities and power suply.
Then the required software was written, using existing implementations when possible and writing new code when required.
A simple example app was also developed, based on an existing communications app published by Nordic Semiconductors and installed on a phone.

\subsubsection{Evaluation}
The developed system of tags and phone was tested in a series of five experiments.
The first four experiments were intended to caputre a specific part of the system, while the last was a general purpose test.
The tests were performed in a manner that insured minimal external influence.
The resulting data from the testswere analyzed using statistical methods.
The goal was to determin the reliability of the system, find limitations and look for emprovements.

\section{Thesis Outline}

This thesis is structured as follows:

Chapter two presents fundamental knowledge researched for this thesis. 
Additionaly chapter two also present previous research done, related to the tracking of artwork and sensor networks.
Chapter three presents the design of the system and its inner workings and capabilities.
Chapter four shows the implementation that was developed and used for this project.
Chapter five decsibes the experiments performed. It presents the results of the experiments and discusses them.
chapter six summarizes the finding of this thesis and discusses the most important aspects in a conclusion.


%\chapter{Introduction}
\label{introduction}


Artworks are sensitive to many externalfactors, such as humidity, temperature and vibrations.
While these factors are generally well controlled for in museums and storage, artpieces need also be moved between these buildings.
Artworks include a variety of objects, ranging from photographs and paintings the variety of pieces needed for modern installations.
The transportation therefore needs to be secure and flexible.
Sensors play a critical role in the safe transportation of these objects.


\section{Motivation}

Certify is an international cooperative project between twelve partners situated in Switzerland and the EU %\citep{Brüschre von certify}.
One of those partners is the university of zurich.
Its focus is on the development of Internet of Things (IoT) systems for security, monitoring and detection %\citep{Brüschüre von certify}.
Next to certifications and the development of frameworks, Certify also consirnes itself with the integration of IoT devices.


One of multiple currently running pilots of Certify is the "Tracking and monitoring of artworks" %\citep{https://certify-project.eu/pilots/}.
The goal of this pilot is to enable the constant tracking and monitoring of artworks by attaching a device to it.
This device allows for unique identification, by using cryptographic methods.
It is also intended to act as a cluster of sensors that collect information about the surounding of the artwork that are relevant to the wellbeeing of the artwork.
The goal is to have constant data on the artwork throughout its lifecycle.
This is intended to help with securing the artwork and helping with chain of custody monitoring.



\section{Thesis Goals}

The goal of this project is to develop a system that implements a localized version of the artwork tracking envisioned by the Certify project, ment for transportation in a truck.
Additionally the system will extend the Certify Projects goal by adding new detection methods and also informs the driver of the truck about potential problems.

The goal is to develop a system that tracks the state of artwork in a truck using different detection methods.
The devices attached to the artworks, called tags, build a local decentralized network.
The phone of the driver can query the network and displayed the collected metrics to them.
If a metric is outside of the accepted norm, the system should allert the driver.

This thesis presents a proof of concept implementation.
The used metrics are not intended to be a full representation of needed sensors to securley Transport art.
Rather they are intended to show different types of sensors that can be used.
This thesis assumes that the data transfer to a server using 4G, as planned by the Certify project, will work and will not implement it in this thesis.



\section{Methodology}

This Thesis was made in four stages:

\subsubsection{Reserach}
In a first step, the basis of the thesis had to be researched.
This involved familiarizing with existing research on the toping of artwork tracking, local IoT networks and commonly used communication protocols.
Existing artwork tracking methods need to be analyzed and evaluated, considering their strength and shortcommings during the transportation in a truck.
The types of sensors that could be relevant need to be chosen, based on existing research, cost and avaliablity.
Options for the network-architecture inside the truck needed to be researched and compared, based on performance, stability and security.
A communication protocol needed to be chosen, based on the same criterias.

\subsubsection{Design}
Once the fundamental knowledge for the project had been aquired, the system had to be designed.
The design was chosen based on feasability, security and stability.

\subsubsection{Implementation}
The desig then was implemented in a simplified manner based on the material that was avaliable.
For this four tags were build, equiped with sensors, communication-capabilities and power suply.
Then the required software was written, using existing implementations when possible and writing new code when required.
A simple example app was also developed, based on an existing communications app published by Nordic Semiconductors and installed on a phone.

\subsubsection{Evaluation}
The developed system of tags and phone was tested in a series of five experiments.
The first four experiments were intended to caputre a specific part of the system, while the last was a general purpose test.
The tests were performed in a manner that insured minimal external influence.
The resulting data from the testswere analyzed using statistical methods.
The goal was to determin the reliability of the system, find limitations and look for emprovements.

\section{Thesis Outline}

This thesis is structured as follows:

Chapter two presents fundamental knowledge researched for this thesis. 
Additionaly chapter two also present previous research done, related to the tracking of artwork and sensor networks.
Chapter three presents the design of the system and its inner workings and capabilities.
Chapter four shows the implementation that was developed and used for this project.
Chapter five decsibes the experiments performed. It presents the results of the experiments and discusses them.
chapter six summarizes the finding of this thesis and discusses the most important aspects in a conclusion.


%\chapter{Introduction}
\label{introduction}


Artworks are sensitive to many externalfactors, such as humidity, temperature and vibrations.
While these factors are generally well controlled for in museums and storage, artpieces need also be moved between these buildings.
Artworks include a variety of objects, ranging from photographs and paintings the variety of pieces needed for modern installations.
The transportation therefore needs to be secure and flexible.
Sensors play a critical role in the safe transportation of these objects.


\section{Motivation}

Certify is an international cooperative project between twelve partners situated in Switzerland and the EU %\citep{Brüschre von certify}.
One of those partners is the university of zurich.
Its focus is on the development of Internet of Things (IoT) systems for security, monitoring and detection %\citep{Brüschüre von certify}.
Next to certifications and the development of frameworks, Certify also consirnes itself with the integration of IoT devices.


One of multiple currently running pilots of Certify is the "Tracking and monitoring of artworks" %\citep{https://certify-project.eu/pilots/}.
The goal of this pilot is to enable the constant tracking and monitoring of artworks by attaching a device to it.
This device allows for unique identification, by using cryptographic methods.
It is also intended to act as a cluster of sensors that collect information about the surounding of the artwork that are relevant to the wellbeeing of the artwork.
The goal is to have constant data on the artwork throughout its lifecycle.
This is intended to help with securing the artwork and helping with chain of custody monitoring.



\section{Thesis Goals}

The goal of this project is to develop a system that implements a localized version of the artwork tracking envisioned by the Certify project, ment for transportation in a truck.
Additionally the system will extend the Certify Projects goal by adding new detection methods and also informs the driver of the truck about potential problems.

The goal is to develop a system that tracks the state of artwork in a truck using different detection methods.
The devices attached to the artworks, called tags, build a local decentralized network.
The phone of the driver can query the network and displayed the collected metrics to them.
If a metric is outside of the accepted norm, the system should allert the driver.

This thesis presents a proof of concept implementation.
The used metrics are not intended to be a full representation of needed sensors to securley Transport art.
Rather they are intended to show different types of sensors that can be used.
This thesis assumes that the data transfer to a server using 4G, as planned by the Certify project, will work and will not implement it in this thesis.



\section{Methodology}

This Thesis was made in four stages:

\subsubsection{Reserach}
In a first step, the basis of the thesis had to be researched.
This involved familiarizing with existing research on the toping of artwork tracking, local IoT networks and commonly used communication protocols.
Existing artwork tracking methods need to be analyzed and evaluated, considering their strength and shortcommings during the transportation in a truck.
The types of sensors that could be relevant need to be chosen, based on existing research, cost and avaliablity.
Options for the network-architecture inside the truck needed to be researched and compared, based on performance, stability and security.
A communication protocol needed to be chosen, based on the same criterias.

\subsubsection{Design}
Once the fundamental knowledge for the project had been aquired, the system had to be designed.
The design was chosen based on feasability, security and stability.

\subsubsection{Implementation}
The desig then was implemented in a simplified manner based on the material that was avaliable.
For this four tags were build, equiped with sensors, communication-capabilities and power suply.
Then the required software was written, using existing implementations when possible and writing new code when required.
A simple example app was also developed, based on an existing communications app published by Nordic Semiconductors and installed on a phone.

\subsubsection{Evaluation}
The developed system of tags and phone was tested in a series of five experiments.
The first four experiments were intended to caputre a specific part of the system, while the last was a general purpose test.
The tests were performed in a manner that insured minimal external influence.
The resulting data from the testswere analyzed using statistical methods.
The goal was to determin the reliability of the system, find limitations and look for emprovements.

\section{Thesis Outline}

This thesis is structured as follows:

Chapter two presents fundamental knowledge researched for this thesis. 
Additionaly chapter two also present previous research done, related to the tracking of artwork and sensor networks.
Chapter three presents the design of the system and its inner workings and capabilities.
Chapter four shows the implementation that was developed and used for this project.
Chapter five decsibes the experiments performed. It presents the results of the experiments and discusses them.
chapter six summarizes the finding of this thesis and discusses the most important aspects in a conclusion.


%\input{chapters/introduction}
%\input{chapters/fundamentals}
%\input{chapters/design}
%\input{chapters/implementation}
%\input{chapters/evaluation}
%\input{chapters/considerations}

%\chapter{Fundamentals}

In this chapter, the fundamental knowledge that was researched during this thesis is presented.
Section \ref{s:background} introduces the background knowledge required for this thesis and Section.
Section \ref{s:related_works} presents the current state of research on topics surrounding this thesis.

\section{Background}
\label{s:background}

This Section describes the theoretical background used in this thesis.
It covers key aspects, such as communication protocols, two-way ranging, sensor networks, and required graph theory.


\subsection{Wirerless Sensor Networks (WSN)}
\label{S:WSN}
Kevin Ashton coined the term Internet of Things (IoT) in 1999, although the idea predates this term and was before known as embedded internet or pervasive computing \cite{alaba2024evolution}.
It describes the ubiquity of digital devices and their seamless integration into the physical world and everyday life.



At the end of the 90s and during the 2000s, the production of embedded systems and sensors, in particular, rose.
This led to falling prices and sensors becoming widespread.
With the new availability of sensors, Wireless Sensor Networks (WSN) became more widespread.
While not originating in during this time, the term itself was coined in 1980, and the research community became more focused on the topic \cite{jindal2018history}.

\cite{jindal2018history} determined four main challenges in the development of WSNs:
\begin{itemize}
\item Self-organization: A large number of nodes should not require manual installation and maintenance
\item Cooperative Processing: Sensor nodes have limited memory. Evaluating, compressing, and transporting the data becomes a major challenge.
\item Energy efficiency: WSNs often operate where no power supply is available. The sensors, therefore, must run on limited, battery-based energy.
\item Modularity: WSNs should work for various applications and sensor node types.
\end{itemize}

The Ad hoc On-Demand Distance Vector Routing (AODV) Protocol was published in 1999 by E. Perkins and E. M. Royer \cite{royer1999multicast}.
It presents a routing algorithm for wireless ad hoc networks, where routing is only established when needed, and devices can be added to or leave the network at will.
Doing this takes the problem of self-organization and modularity into account.
A modified version of AODV is used in Zigbee to this day \cite{mu2017improved}. 

In 2000, Heinzelman et al. published the low Energy Adaptive Clustering Hierarchy (LEACH) algorithm \cite{heinzelman2000energy}.
Leach divides the sensors into clusters based on location.
The clusters communicate with a network head using cluster heads that collect and transmit the cluster's data.
New cluster heads are elected periodically to spread the increased energy drain from the data transfer to the network head.
LEACH is used \cite{sefati2022cluster, ghazy2023low} and improved \cite{bagherzadeh2022survey} to this day.

\subsection{Ultra Wideband}

\subsubsection{IEEE}

The Ultra Wideband (UWB) communication protocol was introduced in 2003 by the Institute of Electrical and Electronics Engineers (IEEE) as part of the IEEE 802.15.4 standard.
In 2020, updates were made to the protocol when the IEEE 802.15.4z-2020 standard made improvements to the PHY layers of UWB connections. 
It achieved this by introducing a more robust timestamping system on the PHY layer.
This is supplemented by changes to the MAC layer that allow for the exchange of ranging information.
The result is short frames that are transmitted fast between devices, leading to short bursts of communications that are fast, secure, and ideal for ranging.


UWB works by using short radio frequency pulses, resulting in a large bandwidth.
UWB is a lower-power communication form.
This prevents it from interfering with other communication forms with which it shares its wavelength, such as WLAN or Bluetooth. 
Since UWB uses very short, distinct pulses over a short range, it has been found to be useful in ranging systems \cite{hsu_2021}. 
UWB is split into high-rate pulse (HRP) UWB and low-rate pulse (LRP) UWB.
Since ranging is part of this work and LRP is generally not used for ranging \cite{hsu_2021}, I will not discuss it further in this Thesis.
Since UWB devices tend to be small and have low energy consumption, in combination with the capability of ranging and data transfer, they have become popular as Internet of Things (IoT) devices. \\ 

The standard defines the PHY and MAC layer as well as frequency bands for communication.
The 4z expansion tries to integrate UWB into the WPAN standard. In Section \ref{sec:UWB MAC} and \ref{ss:UWB_PHY} I will discuss the PHY and MAC layer.\\

The sending devices emits pulses in a pre-set band of frequencies, using short bursts to transmit the bits.
The signal forms a concave curve in this band, where the two points 10 db below the maximum power spectral density are called the lower- and upper-frequency points, see Figure \ref{f:UWB_spectrum}.
These two points must be at least 500 Hz apart.
The maximum power spectral density must be below the noise level.
This process prevents conflicts with other communications, that use a single frequency with a high power spectral density and modulate signal transmission, such as WI-FI or Bluetooth.
The UWB protocol has the added benefit of being useful for high-accuracy localization.


\begin{figure}[ht!]
\centering
\includegraphics[width=\linewidth]{graphics/UWB_spectrum.jpg}
\caption{Power Spectral Density: Bandwidth B, lower-frequency f\textsubscript{L}, upper-frequency f\textsubscript{H}, \cite{hsu_2021}}
\label{f:UWB_spectrum}
\end{figure}


\subsubsection{UWB supported Nodes}

The IEEE 802.15.4 standard distinguishes between two types of devices.
Full-function devices (FFD) are capable of connecting to multiple other devices, receiving, transmitting, and coordinating. Reduced-function device (RFD), on the other hand, can only connect to one other device and act as a worker. 
In Topological terms, RFDs can only operate as leaves, while FFDs can be any node in a network, including leaves.
RFDs, therefore, are strictly worse but make up for it by requiring fewer resources, such as memory and power.
When FFDs work as PAN coordinators, they can use short addresses to address any node.
The PAN also has a PAN identifier to help communication across multiple networks while still using the short address.
Each device also has an extended address that is not assigned by any coordinator and serves as a universal unique identifier (UUID).


\subsection{UWB MAC}
\label{sec:UWB MAC}

The Mac Layer is part of the Data link layer.
The Mac Frame is the payload of the PHY frame. It carries information about the frame type, frame format, security mechanism, addressing, and frame validation.
The Mac Layer additionally provides rules for beacon management, and channel access.

\begin{figure}[ht!]
	\centering
	\includegraphics[width=\linewidth]{graphics/general_MAC_Frame_Format.png}
	\caption{General MAC Frame Format \cite{IEEE4-2020-7}}
	\label{f:MAC Frame Format}
\end{figure}


\subsubsection{MAC Frame Format}

Figure \ref{f:MAC Frame Format} shows the composition of a UWB-MAC frame.

In the MAC header (MHR), the Frame Control Field includes information about:

\begin{itemize}
  \item the frame-type
  \item if the Auxiliary Security Header Field is used and in what capacity
  \item if additional frames will follow
  \item if an acknowledgment message is expected
  \item if the message is between different PAN-Networks.
  \item of what type the receiver is (PAN coordinator, device, PAN-Network)
  \item the used frame-format standard
  \item where to find the source address
\end{itemize}

The Sequence Number counts up, helping to keep track of the order in which frames have arrived.
The Addressing Fields carry the IDs of the sender and recipient for the frame.
The Auxiliary Security Header Field only exists if specified in the control Field.
It contains additional information needed for the chosen security method.

There are two parts to the information element (IE).
The header IE specifies additional information about the frame, for example, data formatting information or channel time allocation.
The payload IE specifies the length and data type of the payload field.
The payload contains the data that is sent.
It and the IE are of variable length, depending on the frame type and data length.

The MAC footer (MFR) marcs the end of the frame.
It only contains the frame checking sequence (FCS) that can be used to detect corrupted frames using cyclic redundancy checks.

\subsection{UWB PHY}
\label{ss:UWB_PHY}

\subsubsection{PHY Chanel}

The IEEE 802.15.4z amendment defines 16 channels for communication for HRP UWB. 
A channel is defined by its center frequency.
UWB devices can transmit on three different bands: high band, low band, and sub-gigahertz.
For each band, there is one channel that is mandatory to support if a device supports the band.
The other channels are optional, but if two devices want to communicate with each other, they need to use the same band.
The bands, 16 channels, and their ranges, and which channels are mandatory can be found in the table  (see table \ref{Table:: UWB frequency and channel assignments}).


\begin{table}[ht!]
\centering
\begin{tabular}{|l|l|c|c|}
\hline
\textbf{Channel number} & \textbf{Center frequency (MHz)} & \textbf{HRP UWB band}       & \textbf{Mandatory}  \\ 
\hline
0                       & 499.2                           & sub-gigahertz               &  \checkmark     \\ 
\hline
1                       & 3494.4                          & \multirow{4}{*}{Low band}   &                   \\ 
\cline{1-2}\cline{4-4}
2                       & 3993.6                          &                             &                   \\ 
\cline{1-2}\cline{4-4}
3                       & 4492.8                          &                             & \checkmark      \\ 
\cline{1-2}\cline{4-4}
4                       & 3993.6                          &                             &                   \\ 
\hline
5                       & 6489.6                          & \multirow{11}{*}{High band} &                 \\ 
\cline{1-2}\cline{4-4}
6                       & 6988.8                          &                             &                   \\ 
\cline{1-2}\cline{4-4}
7                       & 6489.6                          &                             &                   \\ 
\cline{1-2}\cline{4-4}
8                       & 7488                            &                             &                   \\ 
\cline{1-2}\cline{4-4}
9                       & 7987.2                          &                             & \checkmark       \\ 
\cline{1-2}\cline{4-4}
10                      & 8486.4                          &                             &                   \\ 
\cline{1-2}\cline{4-4}
11                      & 7987.2                          &                             &                   \\ 
\cline{1-2}\cline{4-4}
12                      & 8985.6                          &                             &                   \\ 
\cline{1-2}\cline{4-4}
13                      & 9484.8                          &                             &                   \\ 
\cline{1-2}\cline{4-4}
14                      & 9984                            &                             &                   \\ 
\cline{1-2}\cline{4-4}
15                      & 9484.8                          &                             &                 \\
\hline
\end{tabular}
\caption{ HRP UWB Frequency and Channel Assignments  \cite{IEEE4-2020-7, IEEE4z}}
\label{Table:: UWB frequency and channel assignments}
\end{table}

\subsubsection{Scrambled timestamp sequence}

The 4z amendment added the option to include a scrambled timestamp sequence (STS) into the frame.
The STS is a cyphered sequence that includes the timestamp and is used for ranging.
It is meant to increase the accuracy and integrity of the raging results.
Before transmitting, the receiver and sender exchange a randomly generated key.
The key is then used to encrypt the timestamp using the advanced encryption standard (AES) with 128 bits.
This ensured that the signal had not been intercepted and changed to manipulate the ranging result.
Devices that support STS are called HRP-enhanced ranging capable
devices (HRP-ERDEV).

\subsubsection{Pulse Repetition Frequency}
\label{sec:pule repetition frequency}
The pulse repetition frequency (PRF) is the frequency at wich bursts are sent by the transmitter.
The mean PRF is the average PRF while sending the payload (power switching service data unit PSDU) \cite{hsu_2021}.
The higher the mean PRF, the shorter the airtime of each frame, and it allows for faster communication.
HRP-ERDEV uses a different mean PRF than general devices.
They can work in Base pulse repetition frequency (BPRF) operating at mean PRF 64 MHz or in higher pulse repetition frequency (HPRF) mode operating above BPRF (Table \ref{f:mean PRF}).

\begin{table}[ht!]
\centering
\begin{tabular}{|l|l|l|} 
\hline
\textbf{Standard}          & \textbf{HRP UWB mode} & \textbf{mean PRF}            \\ 
\hline
802.15.4                   & Non HRP ERDEV         & 3.9 MHz, 15.6 MHz, 62.4 MHz  \\ 
\hline
\multirow{2}{*}{802.15.4z} & HRP-ERDEV BPRF        & 62.4 MHz                     \\ 
\cline{2-3}
                           & HRP-ERDEV HPRF        & 124.8 MHz, 249.6 MHz         \\
\hline
\end{tabular}
\caption{HRP UWB Mean PRF (Based on IEEE 802.15.4 and IEEE 802.15.4z, \cite{IEEE4-2020-7, IEEE4z})}
\label{f:mean PRF}
\end{table} %cite euse eigeni bricht

\subsubsection{Symbol Encoding}
UWB sends symbols by transmitting a burst of pulses that encode the symbol.
Since the pulses have clean edges, the arrival time can be measured precisely.
This leads to the burst having two ways to carry information( \cite{QorvoGettingBacktoBasics}):
\begin{itemize}
  \item Binary phase-shift keying (BPSK: Encoding zeros and ones shifting the phases of the pulses so the burst beak for one has an opposite amplitude to the other. 
Figure \ref{f:UWB_signal_description} shows the single 101 binary phase-shift keyed. 
Each bit is set twice, to detect problems with transmission.
  \item Burst position modulation (BPM): Changing the timing of the burst so it falls into a different time slot inside of the possible burst position.	
Figure \ref{f:symbol structure} shows how the burst can be placed in a BPM interval. 
The burst cannot be placed in the guard interval. 
The guard exists to minimize inter-symbol interference from the
signals taking multiple paths.
\end{itemize}

\begin{figure}[ht!]
	\centering
	\includegraphics[width=\linewidth]{graphics/HRP_UWB_PHY_symbol_structure.jpg}
	\caption{HRP UWB PHY Symbol Structure \cite{hsu_2021}}
	\label{f:symbol structure}
\end{figure}

One or both of these encoding strategies can be used in a uwb transmission.
The position of the pulses inside of the burst (see figure \ref{f:UWB_signal_description})  relative to each other can be used to detect the presence of multipath effects and adjust for them. 
Using this, precise arrival times for the whole signal can be calculated.

\begin{figure}[ht!]
	\centering
	\includegraphics[width=\linewidth]{graphics/uwb_signal_tramsmission.png}
	\caption{UWB signal transmission byte encoding, \cite{QorvoGettingBacktoBasics}}
	\label{f:UWB_signal_description}
\end{figure}

Non-HRP ERDEV uses BPM and BPSK.
Some HRP-ERDEV can use only BPSK, which uses a higher PRF and, therefore, reduces airtime.








\subsubsection{PHY Frame}
Figure \ref{f:PPDU general} shows a schematic view of a PHY frame as defined by the IEEE 802.15.4 standard.
The Synchronization header (SHR) contains the information needed to detect the signal and adjust to its parameters.
The PHY header contains meta-information about the payload and its encoding.
The PHY payload contains the data to be sent, namly the MAC frame.

\begin{figure}[!ht]
\centering
\includegraphics[width=\linewidth]{graphics/Schematic_view_PPDU.png}
\caption{Schematic view of a PHY frame defined by IEEE 802.15.4 \cite{IEEE4-2020-7}}
\label{f:PPDU general}
\end{figure}


Figure \ref{f:SHR field} shows the synchronization header, consisting of two parts.
The SYNC section is detectable by the receiver and informs it that a transmission has started.
Depending on the predefined mode, pulses of different lengths are used.
The sequence of pulses specifies a set of channels that can be used for communication.
The preamble can also be used to identify a PAN coordinator.

The SHR ends with the Start of Frame Delimiter (SFD).
It indicates that the synchronization has ended, and the coming signals will be data, starting with the PHY header. 
It also contains a timestamp, which can be used for ranging using the time difference of arrival (ToA), see Section \ref{ss:two_way_ranging}


\begin{figure}[!ht]
\centering
\includegraphics[width=\linewidth]{graphics/SHR_field_structure.png}
\caption{SHR Field Structure \cite{IEEE4-2020-7}}
\label{f:SHR field}
\end{figure}	

The PHY header contains all the information needed to read the PHY payload (see Figure \ref{f:PHR general}).
The first bit defines the data rate used during the payload transfer (see section \ref{sec: pulse repetition frequency}).
The next seven bits define the length of the frame, with a frame length of a maximum of 128 bytes.
The 10th bit shows if ranging will be used with this frame.
The next bit is reserved.
Bits 11 and 12 define the preamble duration. It specifies how many repetitions are used, which can range from 16 to 4096.
The last 6 bits are single error correct, double error detect (SECDED) bits that form a Hammock block and can be used to correct single-bit errors and detect, but not fix double-bit errors.

The last part of the PHY frame is the PHY payload (PSDU).
This contains the MAC frame, as defined in section \ref{sec:UWB MAC}.

\begin{figure}[ht!]
\centering
\includegraphics[width=\linewidth]{graphics/PHR_field_format_4.png}
\caption{General PHR Field Format \cite{IEEE4-2020-7}}
\label{f:PHR general}
\end{figure}

The 802.15.4z amendment contains optional changes to the PHY frame format if the participating devices are HRP-ERDEV devices.
Figure \ref{f:HRP-erdev frame} shows the newly allowed structures for a UWB frame.
Configuration 1 is equivalent to the already existing PHY frame.
The others additionally contain a scrambled time stamp.
This can be placed in different places after the SHR.
Since UWB can also be used only for ranging without transmitting a message, configuration three only contains the SHR and STS, without a payload.

\begin{figure}[ht!]
\centering
\includegraphics[width=\linewidth]{graphics/HRP_ERDEV_frame_structures.jpg}
\caption{HRP-ERDEV Frame Structures \cite{hsu_2021}}
\label{f:HRP-erdev frame}
\end{figure}

Additionally, the PHR can be formatted differently (see figure \ref{f:PHR 4z}. 
The reserved field and preamble duration are removed to make more space for the frame length. 
This allows more data to be sent in one frame, increasing the throughput of the UWB communication.

\begin{figure}[ht!]
\centering
\includegraphics[width=\linewidth]{graphics/HRP_ERDEV_HPRF_mode_PHR.png}
\caption{PHR Field Format for HRP-ERDEV in HPRF Mode \cite{IEEE4z}}
\label{f:PHR 4z}
\end{figure}


\subsection{Two-way ranging}
\label{ss:two_way_ranging}
The IEEE 802.15.4z UWB standard describes two ranging methods: single-sided two-way ranging (SS-TWR) and Double-sided two-way ranging (DS-TWR).
In both instances, the distance measurement is done by calculating the time of flight (ToF) of a signal sent between two devices using timestamps and multiplying it by the speed of light. 
In this Section, both SS-TWR and DS-TWR will be discussed.
Two-way ranging(TWR) refers to DS-TWR in all other parts of the Thesis.

\textbf{Single-sided two-way ranging (SS-TWR)}:
During  SS-TWR, one device sends a message to a second and measures the round-trip time (see figure \ref{f:ss_twr}).
Device A sends a message to B and records a timestamp when the message was sent, $T_{A0}$.
When device B receives the response, it also records a timestamp, $T_{B0}$.
After some delay, device B will send a response to A that contains $T_{B0}$ and the current timestamp $T_{B1}$.
On receiving the response, Device A records its timestamp, $T_{A1}$.
The round trip time $T_{round}$ can now be calculated using the timestamps from A:
\begin{equation}
	\mbox{$T_{round}$} =
	\mbox{$T_{A1} - T_{A0}$}
\end{equation}
The reply delay $T_{reply}$ is calculated using the timestamps from B:
\begin{equation}
	\mbox{$T_{reply}$} =
	\mbox{$T_{B1} - T_{B0}$}
\end{equation}

The ToF can be calculated by subtracting these values.
Since the messages were sent twice the same distance, the ToF must be halved before multiplying it with the speed of light to get the distance.
\begin{equation}
	\mbox{$distance$} =
	\mbox{$(\frac{1}{2}\cdot T_{round}-T_{reply}) \cdot c_{air}$}
\end{equation}

\begin{figure}[ht!]
\centering
\includegraphics[width=\linewidth]{graphics/schematics/SingleSidedTwoWayRanging.png}
\caption{timeline of single-sided two-way ranging (SS-TWR)\cite{IEEE4z}}
\label{f:ss_twr}
\end{figure}

\textbf{Double-sided two-way ranging (DS-TWR)}:
DS-TWR involves both devices A and B performing an SS-TWR and calculating the average between the results.
Figure \ref{f:ds_twr} shows the two separate ranging sessions.
Their result can then be combined to the average ToF for a single message:
\begin{equation}
	\mbox{$T_{prop}$} =
	\mbox{$\frac {T_{Round1}\cdot T_{Round2}-T_{Reply1}\cdot T_{Reply2}}{T_{Round1}+T_{Round2}+T_{Reply1}+T_{Reply2}}$}
\end{equation}
\begin{equation}
	\mbox{$distance$} =
	\mbox{$T_{prop} \cdot c_{air}$}
\end{equation}

The two ranging sessions can have one message overlapping.
Figure \ref{f:ds_twr_3} shows the timeline of an overlapping DS-TWR that only requires three messages.


\begin{figure}[ht!]
\centering
\includegraphics[width=\linewidth]{graphics/schematics/TwoSidedTwoWayRanging.PNG}
\caption{timeline of double-sided two-way ranging (DS-TWR)\cite{IEEE4z}}
\label{f:ds_twr}
\end{figure}

\begin{figure}[ht!]
\centering
\includegraphics[width=\linewidth]{graphics/schematics/TwoSidedRangingThreeMessages.PNG}
\caption{timeline of double-sided two-way ranging (SS-TWR) with three messages\cite{IEEE4z}}
\label{f:ds_twr_3}
\end{figure}


\subsection{K-Connected Graphs}
\label{ss:k_connected_explained}
In order to build a working positional model based on distance measurement, some background in graph theory is required.
The k-connected subgraph of a graph is a subgraph where it takes at least k removals of vertices to create two isolated subgraphs.
A graph $(V,E)$ can have multiple k-connected subgraphs.
They build the set $S_{(V,E)}$.\\
A minimally weighted k-connected subgraph of a weighted graph $S_{(V,E,w)}$, is a k-connected subgraph $(V',E',w') \in S_{(V,E)}$ that has the smallest sum of weights of all k-connected subgraphs.\\
A metric graph is a weighted graph that satisfies the triangle condition.
Meaning for any three edges $e_{AB},e_{BC},e_{CA} \in E$  that connect three verteces $A, B, C \in V$, it holds that $e_{AB} + e_{BC} \geq e_{CA}$.


Finding minimally weighted k-connected subgraphs is an NP-hard problem.
Kahn et al. \cite{khan2007simple} developed distributed approximation algorithm that finds weighted k-connected subgraphs in metric graphs.
It gives an approximation ratio of $\mathcal{O}(k\cdot log(n))$.
This means that the aproximated solution $w^{apr}$ and the optimal sollution $w^{opt}$ fullfill $w^{apr} < k\cdot log(n)\cdot w^{opt})$.
The algorithm puts all vertices in an order, which is determined randomly, and assigns each vertex a rank based on the order.
Each vertex then removes all edges, except for the k lowest weighted ones that connect it to a vertex with a higher rank.\\
There are more precise approximation algorithms to find minimally weighted k-connected subgraphs \cite{kortsarz2003approximating, kortsarz2004approximation}. However, they are centralized, meaning the graph has to be known in its entirety by one actor.


\section{Related Work} % application of concepts
\label{s:related_works}

This Section presents on overview of the literature relevant to this thesis.
This includes IoT systems used in the art world, sensor networks, and wireless ranging, showing the current practices and current state of research.


\subsection{Artwork Tracking}


Since art preservation is an old field and temperature, humidity, light, and vibrations have been known to be detrimental to most artworks, especially paintings, most research in this direction is older than 20 years \cite{mecklenburg1991mechanical, michalski1991paintings, saunders2004effect}.
Still, the invention of new technologies, such as pattern recognition using artificial intelligence, an improvement on existing tools like infrared imaging, and an active need for solutions have kept the research into artwork preservation an active field \cite{borg2020application, schito2017integrated}.
One such new technologies are sensor networks, which have become widespread in the field of art preservation \cite{shah2016customized}.


Artwork tracking during transportation has not been a significant focus in academia.
The most relevant related research was done by Fort et al. \cite{landi2022iot}.
They developed a low-cost, low-powered sensor node to track the temperature, humidity, pressure, and vibrations of artwork and wooden structures.
The sensor node would then report its findings to a remote server.
They confirmed the validity of their sensor in a series of experiments performed in a static building.
They also presented a theoretical framework for their sensor to be used in a transportation scenario, but they do not report implementing or testing this system.
Their sensor used an accelerometer to detect vibrations, and the Bosch BME280 sensor to detect pressure, temperature, and humidity.
Their sensors did not build a network and were not queried but reported their findings directly to either a Wlan router or a BLE-capable smart device.
Fort et al.'s research showes the value of low-cost sensors in detecting threats to artwork.

\subsection{Sensor Networks}

Wireless Sensor Networks (WSN) have become a central aspect if IoT.
Researchers have tried to focus on the most prevalent problems arising from the development of NSWs, mainly power management, security and privacy, data integrity, and availability \cite{gulati2022review}.


\cite{garg2021healthcare} researched WSNs outside of the controlled environment of a house. They propose a WSN that can track the vitals of mountaineers and call for help when measurements have dangerous values. 
They used an Arduino Mega board equipped with a radio transceiver, using LoRa with a star topology.


\cite{jones2021wireless} created a WSN of NRF24101 board intended to monitor linear infrastructures like deepsea wires, using radio and Wi-Fi for communication. Using deep sleep, they were able to optimize energy usage, so the sensor is predicted to last five years on battery.


\cite{spandonidis2022evaluation} used an accelerometer to detect pipeline vibrations to discover leaks. 
They used a narrowband connection for communication and GPS for localization.
Their sensors could query each other for data, to provide a more complete image of the situation.


\subsection{Wireless ranging}

\cite{li2024indoor} made an overview of publications involving positioning systems for industrial settings. 
They looked at the positioning systems in papers using RFID, BLE, UWB, Wi-Fi, and ZigBee. They found that UWB consistently reported the highest accuracy of these methods.
UWB was the least affected by multipath effects, although it was still the most common issue with this technology.


Early research on ranging using UWB was done by Gezici et al. \cite{gezici2008survey, gezici2005localization}.
These papers gave an overview of the different positioning systems for UWB, angle of arrival, received signal strength, time of arrival, and time difference of arrival.
Time of arrival and time difference of arrival were studied further in these publications, presenting error sources and mitigation tools.


Early research focused on the augmentation of UWB-ranging methods.
\cite{venkatesh2007nlos} proposed using integer programs for mitigating the error for ranging without line-of-sight.
\cite{guvencc2007nlos} tried to solve the same issue by using methods based on the statistics of multipath effects.
BiasSub and BiasRed were proposed to reduce the bias in time difference of arrival by applying of a well-known algebraic explicit solution for source localization  \cite{ho2012bias}.
\cite{fan2017performance} improved UWB ranging by eliminating random error. They did this by pre-filtering, using an anti-magnetic ring to eliminate outliers, and using the double-state adaptive Kalman filter to improve position accuracy.
Newer research has also begun incorporating neural networks into UWB positioning systems \cite{stahlke2020nlos, ridolfi2021uwb, che2020machine}.

UWB localization has been used in many applied contexts.
It has been proposed for pedestrian tracking \cite{otim2019effects}, drone flying \cite{macoir2019uwb}, robot navigation \cite{zhu2020adapted}, navigation system for visualy impared people \cite{rosiak2024effectiveness} and tracking people in buildings \cite{elbaum2024investigating}.
UWB positioning systems are particularly interesting for industrial IoT settings.
\cite{barbieri2021uwb} measured the performance of three different UWB antennas: Qorvo, Sewio, and Ubisense. They encountered many multipath-effects in such a complex environment. They mitigated this by employing a Bayesian filtering method.
\cite{belli2024cloud} used UWB positioning, in combination with Real-time kinematic positioning, to track workers while monitoring the factory. The goal was to trigger an alarm if a dangerous situation occurred.



%\chapter{Design}
\label{c:design}

This section presents the principle design of the monitoring system.
The used components are presented in the section \ref{ss:hardware}.
Section \ref{Dataflow} describes the functionalities and responsibilities of the system components.
In Section \ref{ss:network} the network topology and data-flow is discussed

\section{Hardware}
\label{ss:hardware}

This section describes the hardware used in the project. The setup consists of two distinct components: the artwork-tags, of which there are four, and one Phone that provides the interface to the user. The tag itself consists of 4 components:
\begin{enumerate}
	\item nRF52840 microcontroller
	\item DWM3000 UWB Shield
	\item DHT22 temperature and humidity sensor
	\item MPU6050 accelerometer and gyroscope
\end{enumerate}

\subsection{Microcontroler}
The artwork-tag's fundament is built by the nRF52840 DK microcontroller developed by Nordic Semiconductors. 
It is part of the nRF52 series of microcontrollers intended for development.
The nRF52840 DF is specialized for BLE communication, for which it already includes the necessary components.
It is compatible with the nRF52 Software Development Kit (SDK) developed by Nordic Semiconductors.
The SDK makes it possible to use the ble functionalities and to control the pins. 
It also includes implementations for a plethora of pin-based protocols.
It contains 58 pins, 48 of which are data pins, and 10 manage the power supply for additional modules, which include 3.5 and 5 Volt supply pins.
Thirty-two of the pins are installed the same way as the pins on the Arduino uno, making it compatible with many peripherals designed with this common board in mind, such as the dwm3000.
The remaining ten pins are enough to attach the sensors.
The nRF52840 DK includes a USB-B port that is used for powersuply. Additionally, the USB-B port is connected to two pins and is used for UART communication and debugging.
The nRF52 was chosen since it was available, and previous projects have been done with it in combination with the DWM3000 shield.
As a result, a lot of initial setup was already available.

\subsection{UWB shield}
For communication between the tags and distance measurement, the DWM3000 UWB shield developed by Qorvo was chosen.
The DWM3000 is a commonly used device for research involving UWB \cite{coppens2022overview, leu2022ghost, stocker2022performance}.
It allows low-level access but includes an SDK written in C that makes many processes transparent to the user if they wish.
The SDK uses the Serial Peripheral Interface (SPI) for communication between the shield and the microcontroler.



\subsection{Humidity and temperature sensor}

For humidity and temperature sensors I decided to use the DHT22(AM2302) produced by Guangzhou Aosong Electronic Co. \cite{AM2302}. 
It is a commonly used sensor in IoT monitoring systems \cite{ahmad2021evaluation}. 
The vendor claims a temperature range from -40\degree to 80\degree Celsius with a precision of 0.5\degree. 
\cite{ahmad2021evaluation} could experimentally confirm that errors did not exceed 0.1\degree Celsius. 
They also concluded that the sensor is slow in detecting temperature changes. 
This is also confirmed by the user manual \cite{AM2302}, which states that a read-interval of less than 2s is impossible. 

The humidity sensor can detect the full range from 0\% to 99.9\% humidity, with an advertised maximum error of 2 \% pts \cite{AM2302}.
No research confirming or denieing these claims could be found.

The DHT22 sensor uses three pins from the microcontroller: two pins for power supply and ground and one for single-bus communication.
Since no SDK for this type of communication has been built for the nRF52 board series, it had to be implemented manually by reading the high and low voltage on the communication pin, detecting headers and footers, and parsing the binary messages. 

\subsection{Accelerometer and Gyroscope}
The MPU6050 sensor produced by InvenSense Incorporated provides accelerometer and gyroscope data.
The accelerometer reports the acceleration in the three cardinal directions in meters per second.
The gyroscope reports the rotation around the three Euclidean axes in degrees per second.
In this project, the accelerometer data was not used, only the gyroscope.

The MPU6050 uses four pins: two for power supply and ground and two for communication.
The sensor communicates using the I2C protocol, a serial synchronous communication system.
The microcontroller acts as the master and would, in theory, support multiple workers on the same bus. 
Here, only the MPU6050 uses I2C and is the only worker.
While the nRF52 SDK does not supply an I2C API, it offers a Two Wire Interface (TWI) implementation compatible with the I2C protocol.
It used to offer MPU6050-specific support in older versions of the SDK. 

\subsection{Tag technical plan}
The microcontroller builds the base of the tag. 
The other devices are attached to it over the available pins.
In the nRF52 SDK, each data pin is assigned an integer value. 
These often correspond with the pin's name according to the nRF52830 DK manual, but not always.
This Thesis will use the names in the manual to describe the pins.
Pins are the method by which a microcontroller controls its peripherals.


\begin{table}[ht!]
	\centering
	\begin{subtable}[b]{.4\linewidth}
		\centering
		\begin{tabular}{l|l|l|l|l|}
			& Pin & \rotatebox{90}{DWM3000\phantom{.}} & \rotatebox{90}{DHT22}  & \rotatebox{90}{MPU6050\phantom{.}}   \\
			\hline \multicolumn{1}{|l|}{\multirow{10}{*}{\rotatebox{90}{P4}}}
			& P1.10  & \checkmark    &             &             \\
			\multicolumn{1}{|l|}{} & P1.11  & \checkmark    &             &             \\
			\multicolumn{1}{|l|}{} & P1.12  & \checkmark    &             &             \\
			\multicolumn{1}{|l|}{} & P1.13  & \checkmark    &             &             \\
			\multicolumn{1}{|l|}{} & P1.14  & \checkmark    &             &             \\
			\multicolumn{1}{|l|}{} & P1.15  & \checkmark    &             &             \\
			\multicolumn{1}{|l|}{} & GND    & \checkmark    &             &             \\
			\multicolumn{1}{|l|}{} & P0.02  &               &             &             \\
			\multicolumn{1}{|l|}{} & P0.26  & \checkmark    &             &             \\
			\multicolumn{1}{|l|}{} & P0.27  &               &             &             \\
			\hline \multicolumn{1}{|l|}{\multirow{8}{*}{\rotatebox{90}{P3}}}
			& P1.01  & \checkmark    &             &             \\
			\multicolumn{1}{|l|}{} & P1.02  & \checkmark    &             &             \\
			\multicolumn{1}{|l|}{} & P1.03  & \checkmark    &             &             \\
			\multicolumn{1}{|l|}{} & P1.04  & \checkmark    &             &             \\
			\multicolumn{1}{|l|}{} & P1.05  & \checkmark    &             &             \\
			\multicolumn{1}{|l|}{} & P1.06  & \checkmark    &             &             \\
			\multicolumn{1}{|l|}{} & P1.07  & \checkmark    &             &             \\
			\multicolumn{1}{|l|}{} & P1.08  & \checkmark    &             &             \\
			\multicolumn{1}{|l|}{} & P1.10  & \checkmark    &             &             \\
			\hline \multicolumn{1}{|l|}{\multirow{8}{*}{\rotatebox{90}{P1}}}
			& VDD    &               &             &             \\
			\multicolumn{1}{|l|}{} & VDD    &               &             &             \\
			\multicolumn{1}{|l|}{} & RESET  &               &             &             \\
			\multicolumn{1}{|l|}{} & VDD    & \checkmark    & \checkmark  &             \\
			\multicolumn{1}{|l|}{} & 5V     & \checkmark    &             & \checkmark  \\
			\multicolumn{1}{|l|}{} & GND    & \checkmark    & \checkmark  & \checkmark  \\
			\multicolumn{1}{|l|}{} & GND    & \checkmark    &             &             \\
			\multicolumn{1}{|l|}{} & N.C.   &               &             &             \\
			\hline \multicolumn{1}{|l|}{\multirow{6}{*}{\rotatebox{90}{P2}}}
			& P0.03  & \checkmark    &             &             \\
			\multicolumn{1}{|l|}{} & P0.04  & \checkmark    &             &             \\
			\multicolumn{1}{|l|}{} & P0.28  &               &             &             \\
			\multicolumn{1}{|l|}{} & P0.29  &               &             &             \\
			\multicolumn{1}{|l|}{} & P0.30  &               &             &             \\
			\multicolumn{1}{|l|}{} & P0.31  &               &             &             \\
			\hline 
		\end{tabular}
		\caption{Adruino compatible pin assignment}
		\label{table:ArdruinoPins}
	\end{subtable}
	\hspace{.1\linewidth}
	\begin{subtable}[b]{.4\linewidth}
		\centering
		\begin{tabular}{l|l|l|l|l|}
			& Pin & \rotatebox{90}{DWM3000\phantom{.}} & \rotatebox{90}{DHT22}  & \rotatebox{90}{MPU6050\phantom{.}}   \\
			\hline \multicolumn{1}{|l|}{\multirow{8}{*}{\rotatebox{90}{P6}}} 
			& P0.00  &               &             &             \\
			\multicolumn{1}{|l|}{} & P0.01  &               &             &             \\
			\multicolumn{1}{|l|}{} & P0.05  &               &             &             \\
			\multicolumn{1}{|l|}{} & P0.06  &               &             &             \\
			\multicolumn{1}{|l|}{} & P0.07  &               &             &             \\
			\multicolumn{1}{|l|}{} & P0.08  &               &             &             \\
			\multicolumn{1}{|l|}{} & P0.09  &               &             &             \\
			\multicolumn{1}{|l|}{} & P0.10  &               &             &             \\
			\hline \multicolumn{1}{|l|}{\multirow{18}{*}{\rotatebox{90}{P24}}} 
			& P0.11  &               &             & \checkmark  \\
			\multicolumn{1}{|l|}{} & P0.12  &               &             & \checkmark  \\
			\multicolumn{1}{|l|}{} & P0.13  &               & \checkmark  &             \\
			\multicolumn{1}{|l|}{} & P0.14  &               &             &             \\
			\multicolumn{1}{|l|}{} & P0.15  &               &             &             \\
			\multicolumn{1}{|l|}{} & P0.16  &               &             &             \\
			\multicolumn{1}{|l|}{} & P0.17  &               &             &             \\
			\multicolumn{1}{|l|}{} & P0.18  &               &             &             \\
			\multicolumn{1}{|l|}{} & P0.19  &               &             &             \\
			\multicolumn{1}{|l|}{} & P0.20  &               &             &             \\
			\multicolumn{1}{|l|}{} & P0.21  &               &             &             \\
			\multicolumn{1}{|l|}{} & P0.22  &               &             &             \\
			\multicolumn{1}{|l|}{} & P0.23  &               &             &             \\
			\multicolumn{1}{|l|}{} & P0.24  &               &             &             \\
			\multicolumn{1}{|l|}{} & P0.25  &               &             &             \\
			\multicolumn{1}{|l|}{} & P1.00  &               &             &             \\
			\multicolumn{1}{|l|}{} & P1.09  &               &             &             \\
			\multicolumn{1}{|l|}{} & GND    &               &             &             \\
			\hline
		\end{tabular}
		\caption{Non-Adruino compatible pin assignment}
		\label{table:otherPins}
	\end{subtable}
	\caption{Pin assignments and compatibilities}
\end{table}

Some pins are intended for power supply.
On the NRF52840, these pins are located in section P1; see table \ref{table:ArdruinoPins}.
The three VDD pins supply electricity with a Voltage of 3.5 Volts.
A secondary power supply that uses 5 Volts is also available.
What voltage is needed depends on the peripheral.
In this case, the DHT22 runs on 3.5 Volts, while the MPU6050 is made for 5 Volts.
The P1 section also contains two ground pins that need to be connected to the peripherals and a reset pin to restart the microcontroller.
The last pin is not connected (N.C.).
There are additional ground pins in sections P4 and P24 of the board.

The other pins are called data pins.
These pins can transfer data and be used for communication by using voltage modulation.
The nRF52840 has an I/O voltage of 3.3 Volt.
This means that a voltage of 3.3 Volt corresponds to a \textit{Logic high} and 0 Volts represents a \textit{Logic low}.
This allows the data pin to transfer communication in a binary encoding.
How a signal is interpreted is defined by the used communication protocol.
The MPU6050, for example, uses the I2C protocol and uses two datapins.
The protocol states that one pin is used for a serial clock, and the other pin transmits data.
For the data transmission, the protocol defines what a package looks like.
This includes the start condition, the voltage characteristics that signal the beginning of a package, addressing, data encoding, acknowledgments, and stop condition.

The DHT22 sensor does not use a given communication protocol.
It uses one data point to report its sensor data.
How that data is encoded to high and low voltage is specified in the user manual \cite{AM2302} and has to be implemented manually.

The DWM3000 shield is mounted on the 32 pins for Arduino one connections. 
All pins are forwarded and can be used by other devices in a common Arduino-stackable style.
If they are data pins, they will share the data.
Table \ref{table:ArdruinoPins} shows which devices use which pins.
The only pins shared by multiple devices are power and ground pins.
The microcontroller supplies enough power to support this.

The sensors are attached to the same power source and ground as the shield but use different data pins.
The DWM3000 leaves enough pins unused that both sensors could be attached to them.
Since it is not visible which pins the shield leaves free, it was decided to use data pins that are not attached to the DWM3000 for the DHT22 and MPU6050.
Table \ref{table:otherPins} shows how the sensors are connected to the remaining open pins.

\section{Architecture}
\label{ss:dataflow}

In order to discuss how the dataflow works, first, Section \ref{ss:responsibility} will establish what services are implemented in each part of the system.
Section \ref{ss:dataflow} will explain what triggers events and how they are handled inside the system.

\subsection{Responsibilities}
\label{ss:responsibility}
The system consists of the tags, the sensor network, and the phone.
These parts all have their own responsibilities.

\textbf{Tag:} 
The tag is responsible for managing its sensors. 
It has to do the correct setup and convert its output into an understandable form.
The tag can perform ranging with all its neighbors.
Additionally, the tags are responsible for searching for networks to join and reacting appropriately to network requests, be those queries for sensor data, ranging requests, or network management jobs. 
The tags provide a unique, secure universal identifier to be used by queries or the network.
How this is done is part of the certified project and will not be discussed in this thesis.
The tag is also responsible for its own power management.
This is not the focus of this thesis and will only be mentioned when relevant.
A guideline on power management will not be provided here.

\textbf{Network:} The network is responsible for keeping track of all tags taking part in the network. 
It offers a joining protocol for new devices and remains stable when devices leave or become unavaliable. 
It offers the possibility for phones to connect to the network. 
It ensures quieries from phones get transported to the correct tag and the answers to the correct phone.
It ensures a network topology that corresponds to a graph that is at least 3-connected.
On request, it returns a list of devices connected to the phone.

\textbf{Phone:} The phone connects to the network via the provided method.
It offers a graphical user interface (GUI) for the driver to use.
The GUI offers a method for the driver to set the acceptable ranges for all sensor data.
Additionally, it offers a method to set query interval-length.
The phone is responsible for querying sensor data for each tag and measuring once at each interval.
The phone has to evaluate the answer.
The phone has to report the results to the driver using the GUI.
If a parameter falls outside of the acceptable range for its type, the phone is responsible for alerting the driver to this fact.

The Certify project also plans to collect the sensor data on remote servers using a 4G connection.
The plan is to equip each tag with antennas to allow it to send the data directly to the server itself.
Since this is not a part of this thesis, the responsibility for the tag to do this was not added to a list.
A known problem with this plan is, that a 4G connection is not always possible.
Since small tags have very limited memory, the plan to store the sensor data on the tag is not feasible.
If the setup presented in this thesis is used, it would allow for the storage of the data on the phone, which has a much larger memory.
This, again, was not added since it is not part of this thesis.


\subsection{Dataflow}
\label{ss:dataflow}

\begin{figure}[ht!]
	\includegraphics[width=\linewidth]{graphics/schematics/obervation_loop.png}
	\caption{Sequence diagram of setup and observation loop. Setup is performed once, and the observation loop repeats until it is stopped.}
	\label{f:observation_loop}
\end{figure}

Section \ref{s:network} describes how a tag connects to the network.
Figure \ref{f:observation_loop} shows a sequence diagram of the setup and main observation loop of the system.
At the top, the communicating parts are listed.
\begin{itemize}
	\item Human is the driver of the truck
	\item Phone is the phone used by the Human
	\item Network consists of all the used tags and the network they build.
	\item Connected Tag is part of Network but is listed separately. It represents the tag that is communicating to the phone
	\item Tag-j is also part of Network. It represents the tag that is queried during the observation loop
\end{itemize}
Phone and Human communicate by using a GUI. 
Phone and Connected Tag communicate using BLE. 
Every communication inside Network happens using UWB. This includes the communication between the Connected Tag, Network, and Tag-j. \\
When Phone wants to connect to Network, it looks for advertised BLE devices.
It then displays the devices to Human and lets them pick one.
The phone then pairs with the chosen tag, making it the Connected Tag and Phone's connection to Network.
Once connected to Network, Phone will prompt Human to enter the parameters.
These consist of:
\begin{itemize}
	\item Upper and lower limit for sensor data, like temperature and humidity
	\item Maximal displacement value for distance and gyro. These values represent the maximal difference in registered values that is allowed for positional measurements.
	\item Time between measurements. This gives the time period that will pass between measurements for each device and measurement type.
\end{itemize}
Once the parameters are chosen, Human can start the observation.\\
Each iteration of the observation loop begins with a call to Network for a list of all tags currently in Network.
Since the tag network is a dynamic sensor network, the tags in Network can theoretically change. 
In practice, this should only happen when artwork is loaded/unloaded or a tag becomes faulty.
The request for the list is transmitted to Connected Tag over BLE, which then queries Network for all connected devices.
The response is returned to Phone.
Phone then starts a nested loop, iterating over the list of tags and metrics captured by the system.
For each measurement and tag combination (i,j) Pphone contacts Connected Tag for the value, which in turn queries Network.
Once the message arrives at Tag-j, Tag-j gets measurement i. 
In the case of sensors, this entails contacting the sensor and requesting a value.
If metric i is a distance measurement, tag j will commence a two-way ranging operation over UWB with all its registered neighbors and will report the list of distances, together with the tag addresses they correspond to.
Metric i is then transported over Network back to Connected Tag and finally to Phone.
Phone must then evaluate the retrieved data. \\
During the evaluation process, Phone creates an evaluated measurement and marks it as problematic or unproblematic.
What the evaluation looks like depends on the metric.
\begin{itemize}
	\item For most metrics, like humidity and temperature, the evaluated measurement is equivalent to the received measurement. It is then checked if the measurement falls into the acceptable measurement parameters set by Human.
	If it does not, the evaluated value is marked as problematic.
	\item Some metrics require comparison to the previous data. The gyroscope reports the current orientation of the tag.
	This is then compared to all previous measurements, and the maximal angular difference forms the evaluated measurement.
	If the evaluated metric is bigger than allowed by the set parameters, the measurement is marked as problematic
	After evaluation, the original measurement is added to the list of previous measurements.
	\item The distance measurement has a unique evaluation process, which is described in Section \ref{ss:distance_eval}.
\end{itemize}
Once the data evaluation is done, the evaluated measurement is presented to the user over the GUI, together with the address of the tag it belongs to.
If the evaluated measurement is problematic, the driver will be alarmed.


\subsubsection{Distance evaluation}
\label{ss:distance_eval}
The goal of the distance evaluation is to build a working model of where every tag is.
To achieve this, a quadratic program is solved to get the coordinates of all tags.
The steps to do this are as follows:
\begin{enumerate}
	\item Get a list of all current tags, $T:=\{ t_1, t_2, ... \}$.
	\item For each tag, get the last known distance measurements and put it into a set $S_D:=\{ (t_i, t_j, d_{ij}) \} $, where $t_i$ is the tag which measured, $t_2$ the tag that was measured to and $d_{ij}$ the distance measured.
	\item If a tag has no distance measurements, remove it from the list.
	\item Assign each tag $t_i$ a position in a 3D coordinate system, $(x_i,y_i,z_i)$
	\item Pick one random tag $t_{o}$.
	\item Set the values $x_{o},y_{o},z_{o}$ all to 0.
	\item Create the objective function: $L(X,Y,Z) = \sum\limits_{t_i, t_j, d_{ij} \in S_D}|(x_i-x_j)^2+(y_i-y_j)^2+(z_i-z_j)^2-d_{ij}^2|$. For x,y, and z, use variables for all but the six values set in step (6).
	\item Solve the quadratic program consisting of the Objective function L, and no constraints.
\end{enumerate}
Quadratic Programs, in general, are NP-Hard, but Quadratic Programs with a convex function can be solved efficiently.
$(a-b)^2$ and $c$ are convex functions.
The sum of a convex function is always a convex function.
The objective function in (7) only sums up convex functions and is, therefore, convex.
The quadratic program can, therefore, be solved efficiently.\\
By setting the values of the tag $t_{o}$ to zero, the results of the quadratic function become grounded.
It is not strictly necessary, but without it, the returned solution could have values anywhere in the Euclidean space.
The solution will place the other tags near that region by setting one tag to the coordinates at the origin.
There are still an infinite amount of solutions to this quadratic function since all solutions can be rotated around any axis and still return the same objective function.

\begin{figure}[ht!]
	\begin{subfigure}{.4\linewidth}
		\centering
		\includegraphics[height=150px]{graphics/schematics/connected_dots.png}
		\caption{3-edge connected graph}
	\end{subfigure}
\hspace{1cm}
	\begin{subfigure}{.4\linewidth}
		\centering
		\includegraphics[height=150px]{graphics/schematics/connected_dots_k_connected.png}
		\caption{2 connected graph}
	\end{subfigure}
	\caption{ Left: Five dots, all having at least two connections, still blue can move independently. Right: minimal 2-connected graph, no movement possible. }
	\label{f:connected_dots}
\end{figure}

For a point to be clearly placed in Euclidean space, three distances to other points must be known.
This alone is not sufficient to ensure unique results.
The left of Figure \ref{f:connected_dots} illustrates this point in two-dimensional space.
Every circle is connected to two others, but the blue circles can still move without the whole figure moving.
What is needed to keep every point static is for known distances and tags to build a four-connected graph (three in two dimensions).
The left of Figure \ref{f:connected_dots} shows a solution to the problem on the right by creating a three-connected subgraph.

Once the coordinates for all tags are found, they are compared to previous results.
For each tag, the phone calculates how much it has moved.
The evaluated measurement is the distance of the tag that has moved the most.
If the evaluated measurement is larger then the maximal allowed displacement, the measurement is problematic.



\section{Network}
\label{s:network}

For the presented network to work, tags need to be ranked.
This means that for each tag pair $i,j$, one can either say that $rank(i)<rank(j)$ or $rank(j)<rank(i)$.
To achieve this, the universal unique identifiers are used.
No matter what form the UUID has, it can be converted to an integer, by simply interpreting its binary code as one.
Since the UUIDs are unique, no to tags will have the same resulting integer.
When referring to the rank of a tag in this section, the integer representing the UUID is intended.

While not connected to a phone, the tags inside the truck form a decentralized mesh using UWB for communication.
Each tag keeps two lists: a list of known devices and a list of neighbors.
When a new tag joins the network, it sends a joining request over UWB, containing its UUID, using a weak signal.
All tags in the network that receive this request add the new device to their list if known devices. 
If the new device also has a higher rank, they additionally add it to their list of neighbors.
They then answer by sending their own UUID and adress back to the new tag.
By waiting an amount of time that correlates with their UUID, the tags in the network can ensure that their answers don't overlap.
The new tag adds the received addresses and UUIDs to its known device list. If the added tag's rank is also higher than the new tag, it will add it to the list of neighbors.
If the new tag now has four neighbors, it stops. Otherwise, it will repeat the process with an increasingly stronger signal until it has either found four tags with a higher rank or reached maximum signal strength.
Afterward, it starts advertising its BLE connection.
This concludes the network joining process.


A user with a phone can connect to any of the advertised BLE connections.
Once that happens, the tags in the tracks will switch from their decentralized mesh to a star topology, with the connected device serving as the coordinator.
The coordinator will inform all tags about their new status by sending a message using a strong UWB signal.
The tags will then acknowledge this message in order of rank.
The tags in the network will still keep their stored neighbors and known devices.
The coordinator records a list of all acknowledgemtns, thus creating a list of all devices in the network.


The phone can request the list of all tags from the coordinator.
The phone can now also query the tags in the truck by sending the query to the coordinator over BLE, which then will pass it directly to the appropriate tag using BLE.
For all sensor data, this is a simple call-and-response request. \\
If a distance measurement is queried, a tag take the following steps:
\begin{itemize}
	\item It conducts a UWB two-way ranging session with each tag in the neighbor list.
	\item It reports those results to the coordinator tag.
	\item It orders all received distances.
	\item It keeps the tags with the four lowest distances and deletes the rest from the neighbor list.
\end{itemize}
The first time a distance is requested, the tag will perform more ranging sessions than necessary to build a 4-connected graph.
Afterward, it only performs ranging with four other tags unless a new device is added.
Suppose a ranging session does not report a result because a tag left or became unavailable. In that case, the tag adds ads the tag with the shortest previously measured distance and higher rank from the list of known devices back intro the list of neighbours.


This design mirrors the algorithm proposed by \cite[khan2007simple] and presented in Section \ref{ss:k_connected_explained}.
It creates an approximation of a minimally weighted k-connected subgraph based on the measured distances.
This is allowed since the distances are in Euclydean space, which, when mapped to a graph, forms a metric graph.
As discussed in section \label{ss:distance_eval}, a four-connected graph is needed to uniquly identify the position of each tag.
The graph should be minimally weighted so that measurements are between tags that are as close as possible to each other.
This reduces the multipath effect and theirfore increases precision.


If the tags are not connected to a phone and report their data to a remote server, they can still use the same distance measurement to approximate the k-connected subgraph.
The quadratic program can then be calculated on the server.
%\chapter{Implementation}
\label{c:implementation}

In this section, the implementation that was used for the experiment is discussed.
In section \ref{s:tag} the implementation of the tags is presented.
Section \ref{s:app} is about the implementation of the App.

\section{Tag}
\label{s:tag}
The software of the tags consits of n modules:
\begin{enumerate}
	\item Temperature and humidity sensor
	\item Gyroscope
	\item UWB network
	\item Two way ranging
	\item BLE communication
	\item Job handler
\end{enumerate}
The following subsections will discuss the first five modules, followed by how they interact using the job handler module.
The section \ref{ss:combination} discusses chalanges from combining these modules and how they were solved.

\subsection{Temperature and humidity module}
\label{ss:temp_hum_module}
This module is responsible for managing the DHT22 humidity and temperature sensor.
It is responsible to setup the sensor during initial startup and to provide the sensors measurements when queried.
The DHT22 sensor communicates using only one data pin, pin 13, which will be referd to as the data pin in this section.
Dmitry Sysoletin created an implementation \ref{sysoletin2021nrf52_dht11} for the DHT11 sensor together with the nRF52840 board that build the basis for this implementation, by addapting it to the DHT22 and adding functionalities needed by the job handler module.


Since the DHT22 is a very simple sensor, using single bus comunication, not much setup is needed.
The evaluation of the sensor data requires that the voltage of the pin is read out in pre-defined intervals, when reading the sensor data.
To do this, a clock is required.
This resource has to be reserved an initiated at startup.
This is the only setup that is required for the DHT22 sensor.


To initiate a sensor-read the voltage of the data pin is set to 0.
When the sensor is in standby mode, the data pin is on \textit{logic high}, and when set to \textit{logic low}, the sensor will respond with a read of its current value.
A schematc view of a sensor read of the DHT22 can be seen in Figure \ref{f:dht22_signal}.
The temperature and humidity module will then check the Pin State in intervals of 5ms, until a \textit{logic low} is registered, signalling that the sensor has registered the request. 
The module will now monitor the pin state, waiting for \textit{logic low} followed by a \textit{logic high}, this beeing the start condition of the data transfer. \\
The data is transfered in five chunks of eight bits.
Each bit is preceeded by a prolonged \textit{logic low} state, that is detected by the module
The module then proceeds to write the state of the data pin into a 8-bit buffer, \textit{logic high} corresponding to a 1 and \textit{logic low} to 0.\\
Once all five chunks are read, the communication has ended and the module can verify the data.
The first two bytes correspond are combined to form the temperature information in celcius, the second and third form the humidity.
Both values are multiplied by 100 and stored in a 16-bit integer. This doesn't loose data, since the sensor only measures up to a precision of 1 after the decimal point.
The data beeing stored in an integer help with data transfer.
It will be converted back on the phone.
The fith chunk contains the parity and is used to accept or reject the humidity and temperature values.
If the process fails at any state, -100\degree C is returned for the temperature and -100% for humidity.
These form both impossible values, since humidity can't be negative and the DHT22 sensor can only detect temperatures as low as -20\degree C.


\begin{figure}[ht!]
\centering
\includegraphics[width=\linewidth]{graphics/DHT22_signal.png}
\caption{Signal of a DHT22 sensor-read as presented in the manual \cite{AM2302}.}
\label{f:dht22_signal}
\end{figure}

\subsection{Gyroscope}
\label{ss:gyro_module}
This module manages the MPU6050 gyroscope and accelerometer.
It is responsible to setup the sensor and report its result.
An implementation for the MPU6050 was present in the nRF52 15.3.0 SDK, but is no longer avaliable for the nRF52 17.1.0 SDK, used in this project.
The old implementation was ported to this project.
This consisted of replacing deprecated parts of the SDK with updated ones and adding newly required flags to the build.


MPU6050 sensors use the I2C communication protocol.
The nRF52 SDK does not include an implementation for this protocol, but has a Two Wire Interface (TWI) implementation that is compatible with the I2C protocol.
During startup the TWI module has to be initialized.
This is handled by the SDK, but requires some parameters to be passed.
\begin{itemize}
	\item The Serial Clock Line (SCL) defines what pin will be used for the clock shared in the TWI. This implementation uses pin 11.
	\item The Serial Data Line (SDA) defines which pin is used for the data communication. Pin 12 was used.
	\item The frequency which the TWI uses. It is defined in MPU6050 data sheet, and is 100 kHz \cite{MPU6050}.
	\item The Interrupt priority is a rank that determins, how easyely this process can cause an interrupt. It is set to high.
\end{itemize}
After the TWI service is initiated with these parameters, it is enabled, ensuring that its resources are locked and can not be used by other services.


Afterwards the results from the sensor can be read using the TWI service again.
The TWI-TX requires the adress of the read device and a registry where to write the MPU6050 datasheet \cite{MPU6050}.
The adress of the sensor is the same for all MPU6050 sensors and can be found in the manua
It sets a flag to true once the sensor has writen the data, which then can be read using the TWO-RX function.
The result consist of three 16-bit integers, representing the angular velocity arround the X,Y and Z axis, shown in figure \ref{f:MPU6050_orientation}.\\
Returning this data when queried has only limited use.
It represents a measurement of the current situation.
The caller is more interested what happened since the last query.
Two different implementations for the read of the gyroscope were used during the experimental phase of this thesis.
One would try to return the current orientation of the tag. This read will be called the \textit{orientational read}.
The other would return the maximal registered angular velocity since the last read. This will be called the \textit{angular velocity read}.


\begin{figure}[ht!]
\centering
\includegraphics[width=200px]{graphics/MPU6050_orientation.png}
\caption{Schematic view of the MPU6050, showing the direction of the three axis X,Y,Z.}
\label{f:MPU6050_orientation}
\end{figure}



To achieve the orientational read, three orientational variables $x_{angle}$, $y_{angle}$ and $z_{angle}$ keep track of the current rotation around their corresponding axes.
During setup, all three angles are set to zero.
The MPU6050 is read out periodically in between calls.
The elapsed time since the last read is multiplied with the angular velocity at this moment arround the axis and is added to the orientational variables.
When the gyroscope module is queried for its measurement, $x_{angle}$, $y_{angle}$ and $z_{angle}$ are returned.


The angualr velocity read is achieved in a similar manner.
Three angular velocity variables $x_{max}$, $y_{max}$ and $z_{max}$ are created and set to zero during initiation.
The MPU6050 is read out periodically and its values are compared to the angular velocity variables.
If any of the angular velocity values is smaller in absolute magnitude than the corresponding read value, it is replaced by that read value.
When the the gyroscope module is queried, the values of $x_{max}$, $y_{max}$ and $z_{max}$ are returned.
The angular velocity variables $x_{max}$, $y_{max}$ and $z_{max}$ are then set to zero again.

\subsection{Network}
\label{s:network}
The network module is responsible for the managment of the network.
This consists of: sending requests to join a network, managing requests to join a network, keeping track of its neighbours, transmitting messages and sending messages.
Since only four devices were used in this implementation, the processes for the network are much more simplified, then presented in design chapter.
A 4-connected minial graph of 4 verteces must nececarily include that all the nodes are fully connected.
This leads to a simplified network architecture.
Since this implementation was build to run experiments and not to be used in real-world applications, a lot of security measures were canceled.
Messages are not encrypted and devices are not authenticated.
All messages are assumed to reach their destination and no devices is expected to become unavaliable. \\
The Network-module is based on the implementation of \cite{degkwitz2023ultrawideband}.
It is based on published examples from Qorvo, the producer of the DWM3000 shield.
It uses the DWS3000 SDK to communicate with the DWM3000.


The DWM3000 uses the Serial Peripheral Interface (SPI) protocol.
This requires some resources that have to be reserved and some configurations that need to be set.
This is the first thing that happens during the setup of the UWB network.
Next the interrupt-priorities and the communication speed of the SPI connection are configured.
Then the DWM3000 is reset, to insure no cross-effects from previous sessions are possible.
Then the board is told to initialize.
After that the used configurations are send to the shield.
This includes information like chanel number, preamble codes, data rates and header modes.
The SDK contains many pre-defined configurations.
All configurations that allow for RX and TX and that use scrambled timestampt (STS) work for this usecase.
It is crucal that all tags use the same configurations.
For this implementation the same configurations were used, as in \cite{degkwitz2023ultrawideband}.
The configurations can be seen in table \ref{table:DWM_settings}.
The setup finishes with initiating the LEDs, that serve no critical service, but are usefull for debugging.



\begin{figure}[ht]
\caption{Configurations of the DWM3000 for UWB communication}
\begin{longtable}{|l|c|}
\hline
\textbf{Description} & \textbf{Value} \\
\hline
\endfirsthead
\hline
\endfoot
Channel number & 5 \\
TX preamble length & 128 symbols \\
RX preamble acquisition chunk size & 8 chunks \\
TX preamble code & 9 \\
RX preamble code & 9 \\
SFD type selection & 4z 8 symbol\\
Data rate & 6.8 Mbits/s \\
PHY header mode & standard PHR mode \\
PHY header rate & standard PHR rate \\
SFD timeout & 129 \\
STS mode & enabled \\
STS length & 128 bits \\
PDOA mode & off \\
\hline
\end{longtable}
\label{table:DWM_settings}
\end{figure}


The Certify project uses unique, falsifiable identifiers for its tags.
Since this is not avaliable for the tags used here, the device-ID was used instead.
It serves as a 8-bit long adress for the purposes of this implementation.
Each tag also keeps a list of all known adresses, called neighbours.


The adress \textit{0x3F} was used, when a tag wants to join a network.
This was chosen since none of the used devixes had this device-ID, and it corresponds to a question mark when using ASCI encoding.
When a tag wants to joind a message, it sends the adress \textit{0x3F}, followed by the message 'findnet', and its own adress.
It then start listening for answers.
If the listening timed out without any answers, it sends the message again.


For the network to function, the receiving and sending of messages is critical.
The UWB listener function from project \cite{degkwitz2023ultrawideband} was modified.
It waits for a listened message from the shield.
If it receives a message, it coppies it to a buffer.
It then checks the first bit of the message for the receiver adress.
If the receiver adress is equivalent to the tags own adress, it passes the message on to the job-handler module for further evaluation.
Otherwise the message is discarded.
An exception is made, if the receiver-adress is "\textit{0x3F}", indicating that a tag is looking for a network.
In that case, the network module adds the tag to the list of neighbours.
It then waits for a time proportional to its own adress, before continuing.
Since adresses are unique, this ensures that no two tags responde to the new tag at the same time.
Afterwards it sends a new message, beginning with the adress of the new tag, followed by the string 'NEW' and its own adress.
This way it can be added to the neighbours of the new tag as well.\\
For sending messages, the implementation of \cite{degkwitz2023ultrawideband} was modified.
It sets the DWM3000 to TX, passes a int-buffer and lets it transmit, before returning to RX mode.
Do to limitations discussed in section \ref{ss:combination}, the message length could not exceed 10 bytes. 


\subsection{Two-way ranging}
\label{ss:two_way_ranging}
The two-way ranging module is responsible for measuring its distances to the tags in the neighbourhood.
Since it also uses the DWM3000 shield, it requires no additional setup.


When the two way ranging module gets a distance request, it loops over the list of neighbours, performing two-way ranging with each of them.
First it sends a prepare-rainging request to the neighbour it wants to performe ranging with, before performing the ranging.
It then sends the result back over the network to the requesting tag with the following format: 
\begin{equation}
	\mbox{$a_r$DST$a_ta_ncd_{tn}$}
\end{equation}
with
\begin{itemize}
	\item $a_r$: The adress of the requesting tag.
	\item DST: The string "DST", indicating the purpose of the message.
	\item $a_t$: The own adress of the tag performing the measurement.
	\item $a_n$: The adress of the neighbour that the distance was measured to.
	\item $c$: A boolean. If false, this is the last neighbout measured for this query.
	\item $d_{tn}$: The distance to measured. 
\end{itemize}
The reason for each measurement triggering its own response is the message-lenght limitation mentioned in section \ref{s:network}.

When a tag receives a prepare-rainging request intended for another device, it enters a short sleep.
This is because rainging envolves multiple messages beeing send netween both participants.
This would unnecesarily drain energy from the tags that are not envolved.
Because of that they sleep for the expected duration. \\
If the tag is the entended receiver for the prepare-rainging message, it will enter the preparation part of the two-way ranging module.
If will function as deivce A in respect to figure \ref{f:ds_twr_3}.
In a first step, it will clear all RX and TX buffers.
It then sets the expected RT and TX antenna delays, $d_{rx}$ and $d_{tx}$.
They represent the expected time loss during receiving or transmitting messages and are device specific.
These delays will automatically be taken into account, when calculating the timestamps.
It then sends the first polling message and imideatly starts waiting for a response.
The polling message is a constant string with no changing data.
The DWM3000 will automatically store the transmission and reception timestamps, their is no need to retreive it right away.
When the response is received, it checks if it is the expected response.
If it is, the two timestamps $T_{TX_1}^A$ and $T_{RX}^A$ are retireved.
The final transmission time $T_{TX_2}^A$ is calculated by adding a constant $c_A$ to $T_{RX}^A$:\begin{equation}
	\mbox{$T_{TX_2}^A$=}
	\mbox{$T_{RX}^A+c_A$}
\end{equation}
The final message is then prepared, containing all three timestamps $T_{TX_1}^A$, $T_{RX}^A$ and $T_{TX_2}^A$.
The message is loaded into the message buffer of the DWM3000 and a delayed transmission is started.
The delayed tranmission takes the timestamp $T_{TX_2}^A$ and will start the transmission once that timestamp is reached.
Afterwards all caches are cleaned and the tag returns to its previous state, listening for requests.

The tag that performs the ranging roccesponds to device B in figure \ref{f:ds_twr_3}.
Once it has sent the the prepare-rainging message to its neighbour, it will enter the revceiving part of the two-way ranging module.
As device A, device B will also start by settings its antenna delays $d_{rx}$ and $d_{tx}$ and clear all its RX and TX buffers.
It will then start polling for a message.
Once a message from device A is received and validated, it will retreive the timestamp when the message was received, $T_{RX_1}^B$.
Device B will add a constant $c_B$ to this timestamp to get $T_{TX}^B$:
\begin{equation}
	\mbox{$T_{TX}^B$=}
	\mbox{$T_{RX_1}^B+c_B$}
\end{equation}
It will then start a delayed transmission for the response message at $T_{TX}^B$.
The response is a constant string without any data.
Once the response is sent, device B starts to listen for messages again.
When the final message is received from device A and validated, $T_{TX_1}^A$, $T_{RX}^A$ and $T_{TX_2}^A$ are extracted from the message.
Device B also retreives its final timestamp, $T_{RX_2}^B$.
Once this is done, the time of flight for a single message can be calculated, and from that the distance:
\begin{align}
    T_{round1} &= (T_{RX}^A - T_{TX_1}^A) \\
    T_{round2} &= (T_{RX_2}^B - T_{TX}^B) \\
    T_{reply1} &= (T_{TX}^B - T_{RX_1}^B) \\
    T_{reply2} &= (T_{TX_2}^A - T_{RX}^A) \\
    ToF^{AB} &= \frac{(T_{round1}\cdot T_{round2}) - (T_{reply1}\cdot T_{reply2})}{T_{round1} + T_{round2}) + (T_{reply1} + T_{reply2}} \\
    distance &= ToF^{AB} \cdot c_{air}
\end{align}
The distance is then returned, all caches cleared and the module continues with the next distance measurent, if any are remaining.


The TX and RX antenna delay $d_{rx}$, $d_{tx}$ are different for each device.
Qorvo supplies a default value, but it is the same on all devices.
Since te antenna delays are multiplied with the speed of light, even small mistakes in calibration can lead to big errors.
According to qorvo, without the calibration of antenna delays, a measurement can be off by up to 40 cm \cite{DWM3000Calib}.
This will be a constant bias and not change over measurements.\\
Qorov has published a manual on how to calibrate their devices \cite{DWM3000Calib}.
They have not published a codebasis that implements this process.
The calibration process published by Qorvo required things that were not part of this project:
\begin{itemize}
	\item A synchronized clock, shared over all devices, without significant clockdrift
	\item A UART connection to a computer
	\item A pipeline performing statistical analysis and coordinating the devices.
\end{itemize} 
Since implementing this calibration process whould have been out of scope for this thesis, a simpler version was designed.
The tags were set up in a theathedron, so each tag was 30 cm apart from each other.
Then one tag would perform two way ranging with another tag, chosen at random.
The result would be shared between both tags.
If the result was larger than 30 cm, $d_{rx}$ or $d_{tx}$ would be chosen at random and increased.
If it was lower, $d_{rx}$ or $d_{tx}$ would be increased.
Then the second tag would start a new ranging session with a random tag.
This system was left running for over one hour, until all distances measured were in the range of [27 cm, 33 cm].


\subsection{BLE}
\label{ss:ble_module}
The BLE module is responsible for the communication between the UWB network and the phone.
It advertises the tag to the phone and receives messages from the phone and sends messages to the phone using BLE.
The nRF52840 microcontoler is equiped with a antenna with BLE capabilities.
The nRF52 SDK includes libraries for the managment of this antenna.
It also includes the \textit{ble{\_}app{\_}uart} example project.
This project offers advertises a ble connection, handles the paring process.
Once connected, it forwards all incomming comunication to a USB-UART mdoule connected to a computer.
Input from the computer ver USB-UART is sent as a message to the paired device.
The  \textit{ble{\_}app{\_}uart} example project was took as a basis to build the BLE-module.


The nRF52 SDK for BLE requires the use of the S140 SoftDevice.
The S140 SoftDevice is a BLE protocol stack that can be used for the 811, 820, 833 and 840 series of nRF52 boards.
In order for the SoftDevice to be avaliable, a memory 156 kilobyte segment of memory has to be reserved for it, starting at  memory segment 0x0.
The SoftDevice then has to be flashed to the board.

During startup, the BLE module has to initialize a few services and reserve some resources.
Firstly a nRF clock has to be reserved for the BLE module.
Then the powermanagment for the SoftDevice has to be initiated, before the BLE stack inside the SoftDevice can be initialized.
Next the Generic Access Profile (GAP) and the Generic Attribute Profile (GATT) have to be prepared.
The information what functions to call when the SoftDevice receives data has to be set, as well as the advertized name, the UUID, timeout durations and what to do on faults.
The advertized name was left unchanged from the \textit{ble{\_}app{\_}uart} example, "Nordic{\_}UART".\\
Once the SoftDevice is initialized and the tag has connecteced to the UWB network, the BLE connection can be advertized.
The avertisement function of the nRF52 SDK was used for this.


The BLE module listens for queries sent from the Phone to the tag using BLE.
To achiev this, a query-handler function was passed to the SoftDevice during initiation.
All incomming messages will be passed to this function by the SoftDevice.
When a query is received, the BLE module interprets the message.
It checks what is beeing queried and transformes it into a job, readable by the Job Handler module.
The BLE module also offers a service to send messages to the phone.
This service uses the nRF52 SDK to load the message into a the SoftDevice and send it to the phone.


\subsection{Job Handler}
\label{ss:job_handler_module}

The job-handler module connects all other module.
It takes job structs (see figure \ref{code:job_struct}, interprets which module is responsible for handeling them and calles the job together with the relevant data.
The job struct consits of a field for the job-type, that tells the job-handler what type of job this is. It also includes fields to store data, that is needed for the job.

\begin{figure}[h]
    \centering
    \begin{lstlisting}[language=c]
    struct job {
  		enum job_types type;
  		uint8_t* data;
  		int length;
};
    \end{lstlisting}
    \caption{Job struct}
	\label{code:job_struct}
\end{figure}

There are 14 total job types.The following list while decribe the meaning of them, as well as how they are handled by the job-handler:
\begin{itemize}
  \item \textbf{search for network}: This job is triggered after setup. The tag is not connected to the network. It will be passed to the Network module without any aditinal data.
  \item \textbf{join network request}: This job commes from the Network module, when it receives a request from another tag to join the network. It will be passed back to the Network module, with the data of the new devices id.
  \item \textbf{set network and address}: This job commes from the Network module. It informs the network connection has been established. The job is handed back to the Network module, with the received message, to be added to the list of neighbours.
  \item \textbf{ble temp hum request}: This job commes from the BLE module, where a query for temperature and humidity has been registered. The requested tag is extracted from the job. If the request is for this tag, the job is handed to the Temperature and Humidity module. Otherwise it is passed to the network module, to be transmitted to the requested tag.
  \item \textbf{temp hum request }: This job commes from the Network module and informs that a request for a temperature and humidity read has been made. It is passed to the Temperature and Humidity module, together with the requesting tags adress.
  \item \textbf{temp hum answer}: This job commes from the Network module and carries the respons to a temperature and humidity request. It is passed to the BLE module, togehter with the measurement, which will be passed to the phone.
  \item \textbf{ble gyro request}: This follows the same logic as "ble temp hum request", but with the gyroscope module.
  \item \textbf{gyro request}:This follows the same logic as "temp hum request", but with the gyroscope module.
  \item \textbf{gyro answer}:This follows the same logic as "temp hum answer"
  \item \textbf{ble distance request}: This job comes from the BLE module. The phone has queried for a distance. If the queried tag is not this tag, the message is passed to the Network module. Otherwise, it is passed to the Two-Way Ranging module.
  \item \textbf{distances reques}: This job comes from the Network Module. It requests a distance measurement. The job is passed to the Two-Way Ranging module, together with the requeting tags adress.
  \item \textbf{distances prepare}: This job comes from the Network Module. It informs, that another tag is requesting a ranging session. If the ranging session is with this tag, the job is passed to the Two-Way Ranging module. Otherwise the tag goes to sleep for a short time.
  \item \textbf{distances answer}: This job comes from the Network module. It reports that a distance measurement as been returned. The job is handed ober to the BLE module, together with the content of the message.
  \item \textbf{ble get known devices}: This job comes from the BLE module. It requests a list of all neighbours. The job is transfered to the Network-Module.
\end{itemize}


\subsection{Combining modules}
\label{ss:combination}
Each module except for the job-handler module was developed in seperate projects, to ensure operability.
Afterwards the modules were merged into one project.
The Network module was chosen as the base project, that the other projects were merged into.
This was chosen since the Network module was based on \cite{degkwitz2023ultrawideband}, which intern was based on a example published by Qorvo.
The Qorvo example uses a lot of shorthand, magic numbers and development shortcut, that are not easely readable to developers outside the firm.
The Network module whas therefore chosen as a basis, since merging it into another project would likely be cumbersome, since parts would easely be forgotten or interact poorly, without the knowledge or udnerstanding of the developer.
Combining the modules came with several chalanges, that described in this section.


The Qorvo example that builds the basis of the Network module uses the pin-mapping PCA10056.
This is the pin mapping for boards that include the NRF52840 board, but not the NRF52840 development board, that this example was made for and is used in this thesis.
The NRF52840 board does not contain the nesecary pins to attack a DWM3000 board to it.
This wrong pin-mapping leads to mistakes that the Qorvo example has to work around.\\
When switching to the correct pin-mapping, PCA10040, the Network mdoule would no longer work, since those work-arounds now introduced mistakes now.
Sice fixing the Qorvo example code would have been cumbersome, it was decided to instead change the other modules that used pins, the Gyroscope module and the Themperature and Humidity module.
The pins for those modules, pin 11, 13, and 13. where hard coded into the modules, instead of using the pin-mapping.


The nRF52 SDK offers a rich selection of tools, such as SPI and TWI communication, clocks, ble capabilities, SoftDevice, UUIDs.
These tools are all enabled or disbaled in the sdk{\_}config file.
Merging in general requires only to enable the tools needed by the merged module.\\
Three mdoules requie a nRF clock,, Two-Way Ranging, Temperature and BLE.
The nRF SDK offers  exactly three clcoks slots, so all of them have to be enabled with the apropriate clock type.
Each module has to be adapted, so it uses its assigned clock-slot. \\
The nRF52 SDK can suport up to three SPI or TWI conenctions simultaniosly, nemaed SPI0, SPI1, SPI2, TWI1,TWI2 and TWI3.
SPI and TWI share their memory, so SPI0 can not be used while TWI0 is used and vise-versa.
Since the DWM3000 uses two SPI connections and the MPU6050 uses one TWI connection, exactly enough resources remain, for both devices to run simultanisously.
SPI0 and SPI1 were used for the DWM3000 and TWI3 for the MPU6050.\\
All other SDK resources were non-conflicting.
They were ported from the original module implementation to the merged one without change.


As most embeded systems do, the nRF52840 requires static memory allocation during flashing.
The avaliable memory is seperated into flash-memory and random-access-memory (RAM).
Some memory segments are required by every runable system:
\begin{itemize}
	\item FLASH, \textbf{vectors}: The interrupt vector table defines the interrput handlers for the system, like resets, faults.
	\item FLASH, \textbf{init}: The initialization routine sets up clocks, pins and other peripherals.
	\item FLASH, \textbf{text}: This section contains the executable code in mashine language.
	\item FLASH, \textbf{data}: This section contains the initial values for all global values..
	\item \textbf{rodata}: This section contains the constant variables, that will not change at runtime.
	\item RAM, \textbf{data}: During startup, the initial values for changable global variables are coppied to this section. They can change at runtime.
	\item RAM, \textbf{bss}: This section contains the global variables that do not have initial values.
	\item RAM, \textbf{stack} and \textbf{heap}: The stack and heap that build the runntime environment.
\end{itemize}
Neither the MPU6050 nor the DHT22 require any additional memory segments.
The DWM3000 and the BLE module both require additional memory segments. \\
The BLE module reuqires the SoftDevice to be added to memory. The Softdevice requires 156 KB of Flash and 10.7 KB of RAM.Those reserved memory segments need to be the first one in both Flash and RAM. This additionaly requires SoftDevice oberservers for System on Chip (SoC), BLE, state and stack. Additionaly a segment to house the nRF52 SDK memory allocator is required, nrf{\_}balloc. These segments are rather small, never exceeding 32 bytes.\\
The DWM3000 shield requires two additional memory segments, fConfig in Flash and nrf{\_}balloc in RAM. 
Qorvo does not publish what the fConfig module is for, but it is required for the shield to work. \\
Since the base project was made for the DWM3000 shield, it had to be addapted to addinionaly fit the segments needed for the BLE module. This mainly consisted of moving all segments to later adress-spaces to add room for the SoftDevice reserved memory. All other memory segments had to be added as well. To make room for this, the Flash memory had to be expanded. \\
The Qorvo example implementation for the DWM3000 shield uses some work-arrounds. An example of this is the "NRFX{\_}SPIM3{\_}NRF52840{\_}ANOMALY{\_}198{\_}WORKAROUND{\_}ENABLED" present in the SDF configuration. These workarounds let the SPI communication with the shield perform certen memory manipulation. If these workarounds are necessary is doubtfull, but fixing them would have been out of scope for this thesis.
The workarounds do generaly have no effect on the implementation, with one excpetion. When the DWM3000 receiver sends a message longer than 10 bytes to the microcontroler over SPI, it incroaches on the SoftDevice RAM. This behaviour was found experimentaly, the responsible code could not be located. Since the system can be implemented with the restriction of 10 byte messages, this was done.



\section{App}
\label{s:app}
Nordic Semi Conductors, the maker of the used microcontrolers, published the code to a simple app that allows for BLE communication with their devices.
It is called nRF Toolbox.
It is intended to pair with the \textit{ble{\_}app{\_}uart} example, published in the nRF52 SDK.
Since this example code was used as the basis for the ble communication used in this project, it was addapted to work with this project.


The App contains different modules, intended for different examples, among them the Universal Asynchronus Receiver/Transmitter (UART) module (see \ref{f:Toolbox_modules}).
It is intended to be used with the ble\_ app\_ uart example.
When opem it shows the ble services that are currently beeing addvertised and allows the user to connect to one of them \ref{f:Toolbox_connect}.
It then opens a window similar to phone messangers, were the keyboard can be used to tpye messages, that are sent to the connected devices.

\begin{figure}[ht!]
\centering
\includegraphics[width=200px]{graphics/nRF_toolbox_modules.jpg}
\caption{nRF Toolbox module menue, with the added Art Tracking Module}
\label{f:Toolbox_modules}
\end{figure}

\begin{figure}[ht!]
\centering
\includegraphics[width=200px]{graphics/nRF_toolbox_connect.jpg}
 \caption{nRF Toolbox shows avaliable devices to connect to}
\label{f:Toolbox_connect}
\end{figure}

\begin{figure}[ht!]
\centering
\includegraphics[width=200px]{graphics/nRF_toolbox_messanger.jpg}
\caption{nRF Toolbox UART module screen}
\label{f:Toolbox_output}
\end{figure}

Since the development of an application was not the primary focus of this thesis, it was decided to take the nRF Toolbox app and add a new module for art-traking to it.
The UART module searved as the basis for this new module, since it had a lot of usefull services already implemented.
As with the UART module the art-tracking module opens up the same connection page \ref{f:Toolbox_connect}, that allows the user to select the art-tracking and connect to it.

Once connected, the observation screen is shown (figure \ref{f:Toolbox_art_tracking_empty}).
At the bottom seven parameters can be set: \textit{time}, \textit{max Temp}, \textit{min Temp}, \textit{max Hum}, \textit{min Hum}, \textit{max Angle}, \textit{max Dist}.
The parameters \textit{max}/\textit{min} \textit{Temp}/\textit{Hum} represent the expected range of humidity and temperature.
Any measurement outside these parameter will be considered a dangerous value by the app.
The tollerated difference in angle compared to the previous measurement is set by \textit{max Angle}, larger differences are considered dangerous values.
Distance measurement work analogously with \textit{max Dist} in meters.
The \textit{time} set defines the time that passes enbetween measurements in seconds.
The default is set to 350 seconds.
This means that the time that passes between, for example, the temperature measurements on tag 2 are 350 seconds.


\begin{figure}[ht!]
\centering
\includegraphics[width=200px]{graphics/nRF_toolbox_art_tracking_empty.jpg}
\caption{Art Tracking module oberservation screen before measurements}
\label{f:Toolbox_art_tracking_empty}
\end{figure}


When the user presses the \textit{Start Service} button, a services starts that poeriodically queries the tags for the Measurements.
Figure \ref{code:App_main_loop} shows the measurement loop.
Each sensor is assigned a character.
\textit{T} for temperature and humidity, \textit{G} for gyro and \textit{D} for distance.
Each tag has a number, here from one to four since four tags were used in the experiments.
The loop concatenates these two characters and sends the resulting query to the connected tag.
Then the next tag-number is prepared for the next query.
Once all tags have been queried for a sensor, the tag-number starts with the first again and the next sensor is queried.
In between cals the app waits.
The call time for distance-measurement is fixed at 80 seconds.
Distance measurement takes longer than the other sensors, since for every devices three measurements need tobe conducted.
Additionaly the sensors that do not participate in a ranging session are sleeping for a quite generous amount of time, to ensure they don't distrub the ranging session.
80 seconds has been chosen, since it allows enough time for all the ranging to happen, plus two repeats per sensor in case the ranging session fails.
For the other sensors the waiting time in between queries is calculated from the remaining set time, after the ranging time is deducted.

\begin{figure}[h]
    \centering
    \begin{lstlisting}[language=Java]
    private val sensors = listOf("T", "G", "D")
    private val devices = listOf("1", "2", "3", "4")
    private var measurement_type = 0
    private var tag = 0
    private var timeBetweenCals: Long = 3750

    private val runnable = object : Runnable {
        override fun run() {
            if (tag >= list2.size) {
                tag = 0
                measurement_type += 1
            }
            if (measurement_type >= list1.size) {
                measurement_type = 0
            }
            val textToSend = "${list1[measurement_type]}${list2[tag]}"
            artRepository.sendText(textToSend, MacroEol.LF)
            tag += 1
            if(list1[measurement_type] == "D"){
                handler.postDelayed(this, 80000)
            } else {
                handler.postDelayed(this, timeBetweenCals)
            }
        }
    }
    \end{lstlisting}
    % Optionally, add a caption to the figure
    \caption{Section from the ArtMetricService.kt, main measurement loop}
	\label{code:App_main_loop}
\end{figure}

Once the process has started, the queries will appear in the chat window on the right side of the screen.
The responses are on the right side, see figure \ref{code:App_main_loop}.
If the response is inside the set parameters, the message bubble will appear blue.
If the measured value is considered a dangerous value, the text bubble will appear red (see figure \ref{f:Toolbox_art_filled}).
Since the message display is programmed in a asynchronus way, it can happen, that the answer to a query appears before the querry itself, if the querried tag is the same as the connected tag.
The service can be stopped by pressing the \textit{start service} button again or by exiting this screen in any way.


\begin{figure}[ht!]
\centering
\includegraphics[width=200px]{graphics/nRF_toolbox_Bad_Value_2.jpg}
\caption{Art Tracking module, queries and responses}
\label{f:Toolbox_art_filled}
\end{figure}


%TODO: add image that shows an example file
The query-answers are appended to a file that is safed in the app-storage.
The information appendded consits of: the queried tag, the returned values, a timestamp and if the value was unproblematic.
This functionality is intended for experimental evaluation. 
In a real word application, this data should be periodically backed up on a server in a compressed manner.
When pressing the share-button on the top right of the message-box \ref{f:Toolbox_art_filled}.
It will open the Android naitive share functionality, to share the file over mail, an installed messanger, save it to onedrive or send it over Bluetooth.
In this project all files were sent with email.
Pressing the trashcan next to it will delete the chat and empty the file.
This allows the user to distinguish between different testing session.
%\chapter{Evaluation}
\label{chap:evaluation}



\section{Experiments}
\label{s:Experiments}
Five experiments were performed to validate the functionality of the tags.
The first two are non specific and ment to test the setup in a stable environment.
Experiments three to five are intended to test the detection of unwanted circumstances.
For all experiments the query-frequency was set to 330s, so measurment was evaluated once every 330s.
The measurements queries are spread across this timeframe.
Each experiment lasted between 40 minutes and one hour.
The results were stored on the phone and then exported using email.
The analysis of the data and creation of graphs was then performed using a Jupiter Notebook, using Pandas and Pyplot for datamanagment and the creation of graphs.


\subsection{Experiment 1: Static}
\label{ss:exp_1}
The four tags where placed on the corners of a 80 cm by 50 cm rectangle on a wooden table.
Each tag was turned on sequentially and given enough time to establish the network.
The phone then was connected to one tag.
The parameters in the app were left unchanged.
The default parameters are large enough, that no measurement should be large enough to trigger a warning.
The setup was then left untouched for 35 min.
The goal of this experiment was, to gauge by how much the measurements can vary in a static environment.
 

\subsection{Experiment 2: Coordinated movement}
\label{ss:exp_2}

\subsection{Experiment 3: Temperature}
\label{ss:exp_3}
The four tags were placed in the same 80 cm by 50 cm rectangle as in experiment one.
One tag placed on a elevated surface, 4 cm above the table.
Under the tag seven candles were placed (see figure TODO, Bild einfügen).
Next to the tag two thermometers detectors were placed.
Each tag was turned on sequentially and given enough time to establish the network.
The phone then was connected to one tag.
The max Temperature parameter in the app was changed to 35°C.
After 20 minutes the candles were lit.
The experiment was then left alone for another 30 minutes.
The independent thermometers were filmed during the process, to allow for later review and comparesment.
The goal of experiment 3 was to test the temperature detection capabilities of the system.


\subsection{Experiment 4: Gyroscope}
\label{ss:exp_4_1}
Again all for tags were placed on a 80 cm by 50 cm rectangle.
Each tag was turned on sequentially and given enough time to establish the network.
The phone then was connected to one tag.
The maximal allowed angular difference was set to 30°.
After 20 minutes one tag was turned by 90° counterclockwise.
The experiment then ran for another 30 minutes.
The goal of experiment number 4 was to test the detection of unwanted rotations.
Experiment four was repeated with, with the gyro sending the angular velocity for all axes instead of the current position.


\subsection{Experiment 5: Distance}
\label{ss:exp_5}
The same 80 cm by 50 cm rectangle setup was used.
The tags were turned on sequentilay, giving them enough time to build the network.
The phone was connected to one tag.
The max distance parameter was set to 20 centimeter.
After 20 minutes, one tag was moved parallel to the shorter rectangle line about 20 cm towards the tag on the next corner.
The system was then left resting for another 30 minutes.
The goal of experiment number 5 was to test the detection of unwanted movement.

\section{Experiment Results}
\label{s:exp_res}

In this section the results of the experiments are presented.
All eperiments were performed two to three times.
In each section only the data-set from the first experiment run is presented fully.
The other experiments will be mentioned only, if they have differing data or to confrim an unexpected datapoint.

\subsection{Experiment 1: Static}
\label{ss:exp_1_result}
In eperiment one, all measurements are expected to be unchanging.
Table \ref{t:exp1_means} shows the mean values for temperature, humidity and angle during the experiment by tag.
Figures \ref{f:exp1_graphs_temp}, \ref{f:exp1_graphs_hum}, \ref{f:exp1_graphs_gyro}, \ref{f:exp1_graphs_dist} shows the change of these values over time.

\begin{table}[h!]
    \centering
    \begin{tabular}{|c|c|c|}
        \hline
        \textbf{Tag} & \textbf{Temp Mean} & \textbf{Hum Mean} \\
        \hline
        1 & 22.06 & 32.56 \\
        2 & 21.90 & 33.93 \\
        3 & 22.06 & 32.94 \\
        4 & 21.87 & 32.80 \\
        \hline
    \end{tabular}
    \caption{Mean and Variances for Temperature, Humidity, and Gyroscope Data by Tag during experiment 1}
    \label{t:exp1_means}
\end{table}

\begin{figure}[ht!]
\includegraphics[width=\linewidth]{graphics/exp/exp1_temp_plot_0.png}
 \caption{Experiment 1, temperature over time.}
\label{f:exp1_graphs_temp}
\end{figure}


All four tags have a similar mean temperature and are all less than 0.2 °C apart from each other.
The varaince are also small, tag two having the highest one with 0.05 °C variance. 
The graph shows that all tags have a rising temperature.
The increase is quite small with tag two having the biggest increase of 0.5 °C over 20 minutes.
When the experiment was repeated, the means stayed similar and the variance small, but the temperature changed course.
A downward trend was visible, instead of the upward trend seen during the first experiment.

\begin{figure}[ht!]
\includegraphics[width=\linewidth]{graphics/exp/exp1_hum_plot_0.png}
 \caption{Experiment 1, humidty over time.}
\label{f:exp1_graphs_hum}
\end{figure}

Humidity follows a similar trajectory as.
The means only vary by 1.5 \% pt.
The variance is small, with tag three having the biggest variance with 0.06\% pt.
During the first experiment, humidity increased by a small amount.
When the experiment was repeated, the humidity dropped during the experiment.


\begin{figure}[ht!]
\includegraphics[width=\linewidth]{graphics/exp/exp1_gyro_data_plot_0.png}
 \caption{Experiment 1, angles over time.}
\label{f:exp1_graphs_gyro}
\end{figure}

Since all tags were stationary during the experiment, the gyro sensor was expected to be unchanging.
This is not what happened.
looking at the graph \ref{f:exp1_graphs_gyro} it is clear, that the measurement shows a wide range of angles for each tag and axis.
The only exception is tag 2 around the x axis, which stays at 0 for the whole measurement duration.
Since angle measurements fall into modular arithmetic, it "wrappes around" at 360°, means can only meanigfully be taken if the angles are in a small range.
Since this is not the case for most tags, the only thing that can be said is, that tag 2 has a mean of 0 with variance 0 around axis x.


\begin{figure}[ht!]
\includegraphics[width=\linewidth]{graphics/exp/exp1_dist_data_plot_0.png}
 \caption{Experiment 1, distance over time.}
\label{f:exp1_graphs_dist}
\end{figure}

Table \ref{t:exp1_means}, shows the mean of the measured distances, the row entry beeing the queried tag that initiates the distance measurement, and the row corresponding to the responding tag.
By looking to the measurements diagonaly oposed to each other, one can see that the measured distaances is the the same, indipendent of who initiated the measurement, up to a range of two centimeters.
The varince on table \ref{t:exp1_dist_var} also show, that these measurements are stable over time and don't change by much.
Looking at the graphs on figure \ref{f:exp1_graphs_dist}, the pairs of measurements are visible.
One outlier happens when tag 3 measures the distance to tag 1 at very end of the measurements.
When repearing the measurements these outliers happened again, a bit less frequently then twice per hour.
The outliers always affected a measurement involving tag 1.
The distances measured do not correspond to the actual distances the tags had to each other, also seen in table \ref{t:exp1_dist_means}.
The measured distances can be as far of as 0.5 meters.
The two larger distances, 0.8 and 0.94 meters, correspond to the two larger measured values for each tag, while the smallest measured value always corresponds to the smallest distance, 0.5 meters.
The two larger values are not always ordered correctly, 0.94 meters sometimes beeing measured smaller then 0.8 meters.
In repeated experiments, all these facts stayed true.

\begin{table}[h!]
    \centering
    \begin{tabular}{|c|c c c c|}
        \hline
          & \textbf{1} & \textbf{2} & \textbf{3} & \textbf{4} \\
        \hline
		\textbf{1} & 0.0 & 1.094 & 1.084 & 0.657 \\
		\textbf{2} & 1.080 & 0.0 & 0.356 & 0.989 \\
		\textbf{3} & 1.007 & 0.367 & 0.0 & 1.279 \\
		\textbf{4} & 0.666 & 0.987 & 1.281 & 0.0 \\
        \hline
    \end{tabular}
\begin{tabular}{|c|c c c c|}
        \hline
          & \textbf{1} & \textbf{2} & \textbf{3} & \textbf{4} \\
        \hline
		\textbf{1} & 0.0 & 0.8 & 0.94 & 0.5 \\
		\textbf{2} & 0.8 & 0.0 & 0.5 & 0.94 \\
		\textbf{3} & 0.94 & 0.5 & 0.0 & 0.8 \\
		\textbf{4} & 0.5 & 0.94 & 0.8 & 0.0 \\
        \hline
\end{tabular}
    \caption{Left: Mean distances between tags in experiment 1. Right: Expected values.}
    \label{t:exp1_dist_means}
\end{table}

\begin{table}[h!]
    \centering
    \begin{tabular}{|c|c c c c|}
        \hline
          & \textbf{1} & \textbf{2} & \textbf{3} & \textbf{4} \\
        \hline
		\textbf{1} & 0.0 & 0.002 & 0.000 & 0.000 \\
		\textbf{2} & 0.001 & 0.0 & 0.000 & 0.001 \\
		\textbf{3} & 0.024 & 0.001 & 0.0 & 0.000 \\
		\textbf{4} & 0.000 & 0.000 & 0.000 & 0.0 \\
        \hline
    \end{tabular}
    \caption{Variance of distances between tags in experiment 1. Row corresponds to queried tag.}
    \label{t:exp1_dist_var}
\end{table}

\subsection{Experiment 3: Temperature}
\label{ss:exp_3_result}
Experiment 3 introduced heat-sources the system.
Since the main setup was the same as experiment 1 \ref{ss:exp_1_result}, many of the findings are the same.
In this section, only differences in results are discussed.
If a metric is not measioned, one can assume it behaved the same as for experiment 1  (see section \ref{ss:exp_1_result}).

\begin{figure}[ht!]
\includegraphics[width=\linewidth]{graphics/exp/exp3_temp_plot_1.png}
 \caption{Experiment 3, temperature over time, mith external measurement added.}
\label{f:exp3_graphs_temp}
\end{figure}

The progression of the external thermoeter and the internal temperature sensor can be seen in figure \ref{f:exp3_graphs}.
The candles, that functioned as the heat source, were lit at 15.10.
During the next measurement of tag 2, at 15.12, both the external thermoeter and the temperature sensor on tag 2 had not yet registered any change, remaing at 22.4 °C.
The extrenal thermometer started rising 1 minutes later, at 15.13.
During the next measurement at 15.18, the temperature-sensor registered a slightly increased temperature of 23.6 °C, while the external thermometer registered 24.7 °C.
During the next measurement at 17.24 the tag reported 26.3 °C while the thermometer showed 27.1 °C. 
The measured temperature of the external thermometer keeps klimbing faster than the internal temperature sensor, until the end of the experiment, as seen in Figure \ref{f:exp3_graphs_temp}.
Their distance nether exeeds 1 °C and gets smaller towards the end of the experiment.
The other tags do not report any significant change in temperature.

\begin{figure}[ht!]
\includegraphics[width=\linewidth]{graphics/exp/exp3_hum_plot_1.png}
 \caption{Experiment 3, humidity over time, with external measurement added.}
\label{f:exp3_graphs_hum}
\end{figure}


Experiment 3 was intended to test the temperature and not the humidity.
Luckily, the external thermoeter also included a humidity sensor, that could retroactivly be used for evaluation.
Figure \ref{f:exp3_graphs_hum} shows the humidity over time, with a added humidity sensor added to the graph.
Since the external humidity sensor was initialy not intended to be used, it is not perticulalry precise and does not display any digits after the decimal point.
The humidity sensor consitently shows a much higher humidity than the one on the tag.
Once the experiment starts at 15.10, the humidity behaves inversly to the temperature and starts falling.
This happens with the external sensor as well as the internal one in parallel.
The registered values plato at 34\% for the external and 26\% for the internal sensor.
The other tags do not report any significant change in humidity.

\subsection{Experiment 4: Gyroscope}
\label{ss:exp_4_result}

Experiment 4 was intended to check the functionality of the gyroscope.
Temperature and humidity behaviour was the same as in the static experiment \ref{ss:exp_1_result}.
As already seen in during the evaluation of experiment 1, the gyroscope does not work as planned.
Figure \ref{f:exp4_graphs_gyro} shows the values of the gyro over time.
Tag 1 was rotated by 90° at 22.25 around the Z axis.
Their is no disernable change in the output of the gyro during or after this process.

\begin{figure}[ht!]
\includegraphics[width=\linewidth]{graphics/exp/exp4_gyro_data_plot_1.png}
 \caption{Experiment 4, gyroscope over time.}
\label{f:exp4_graphs_gyro}
\end{figure}


Figure \ref{f:exp4_graphs_dist} shows the distances of the tags during the experiment.
Before the event, all tags are in a stable state.
As in experiment 1 \ref{ss:exp_1_result} the distances do not represent what is physically happening.
After the tag is turned at 22.26, all measurements involving tag 1 change, and becomming stable again afterwards.
This can be bit hard to see, since "2-1" and "3-1" have an outlier measurement right before and "1-2" right after.
Distance 1 to 2 and 1 to 3 changes between 0.2 and 0.3 meters and distance 1 to 4 changes by arround 0.4 dm.

\begin{figure}[ht!]
\includegraphics[width=\linewidth]{graphics/exp/exp4_dist_data_plot_1.png}
 \caption{Experiment 4, gyroscope over time.}
\label{f:exp4_graphs_dist}
\end{figure}


\subsection{Experiment 5: Distance}
\label{ss:exp_3_result}

Experiment 5 was intended to test the distance measurement capabilities of the setup.
Temperature and humidity and gyro behave as they do in experiment 1 \ref{ss:exp_1_result}.
They will not be discussed for this experiment.

\begin{figure}[ht!]
\includegraphics[width=\linewidth]{graphics/exp/exp5_dist_data_plot_2.png}
 \caption{Experiment 5, distance over time.}
\label{f:exp5_graphs_dist}
\end{figure}

Figure \ref{f:exp5_graphs_dist} shows the measured distances of the 4 tags over time.
As in the static experiment, the measured distances of two devices are similar and mostly stable, before any movement is introduced.
As in experiment 1 the values reported are not correct.
At 14.24 tag 1 is moved by 0.23 meters toward tag 2.
The measured distances to tag 3 increases while the distance to tags 2 and 4 dicreases.
This represent what is happening in reality, since tag 1 is now closer to tag 2 and 4 and further away from tag 3 as before.
The difference in distance is roughly 0.2 meters for tag 2.
This is correct, since tag 1 was moved about that distance towards tag 2.
The measurements show tag 4 now 0.15 meters closer to tag 1.
The effect on tag 4 should be notissable but not as large as it is.
Since the tag moves lateraly towards tag 4, the difference should only be 0.11 meters.
The same is true for tag 3.
The difference in measured distance between 1 and 3 is between 0.15 and 0.2 meters. 
This is too large for the difference a latteral move, it should only be a 0.02 meters difference.
Their is also a small increase in the distance between tags 2 and 3, but which starts before tag 1 was moved.




%\chapter{Final Considerations}
\label{chap:considerations}

%Regarding Final Considerations:
% I. Some teachers and/or methods use the term Conclusion for a text at the end of the paper that aims to expose the results achieved, this term is not incorrect, but many of the works are bibliographical reviews where in the end no conclusion is obtained and yes Several considerations that were found in the development of written work. Therefore, in each project should be considered/weighted if there will be a Conclusion or Final Considerations. Usually what else happens is that you have Final Considerations. The Final Considerations of a paper aims to show if the goal sought for the project was achieved, as well as give a view of the most important considerations and conclusions on the subject addressed, among other aspects. This should include:
%1. An explanation stating clearly whether or not it has achieved the stated objectives (a subdivision between general objectives and specific objectives can also be made here). In each case the reasons must be explained:
%The. If you have achieved the objectives: inform the main factors that contributed to the success, describing them briefly, but do not leave doubts;
%B. If you have not met the objectives: inform how much of the objective has been achieved and cite the factors that contributed to the failure, describing them briefly, but that leaves no doubt.
%2. Describe the main considerations and conclusions that were obtained as a result of the execution of the work. Here should not be repeated text already in the work, but write the impressions of these considerations and how they contributed to the implementation and achieved the goal;
%3. Name and describe the main difficulties encountered in the execution of the work and project. All the work developed means an evolution for the student, and to reach this evolution, it has had to overcome a series of obstacles. Reporting obstacles and overcoming (or not overcoming) helps to dignify and show the merit of the work itself to the reader/evaluator. It is also a contribution, in the sense that once problems and solutions are exposed, readers/evaluators learn/know ways of solving or approaching such problems;
%4. Discuss whether modifications occurred during the execution of the work within the scope defined in the Project phase and in what was developed. It should be explained what generated those modifications, substantiating and justifying such changes.
%5. The relationship between the proposed schedule and the work schedule can be described. This allows the reader/evaluator to learn from the indicated distortions/hits.
%6. Describe or cite future work that may be done based on this work. During the execution of work, it is sought to reach a defined objective in the project. However, several interesting subjects of research are revealed (being that the same ones are not treated/researched in the work because they do not match the objective and scope of the work). The description of such subjects/themes/research demonstrates the students' perception of development as well as their vision of objectivity in the execution of this work.

%II. Conclusion / Final Considerations aim to show the reader/evaluator the student's perception of the work done. In this way, it is not advisable to do citations and references because theoretically everything that was necessary to quote and refer should already be done within the content of the work. Only in some very specific cases/situations can you make referrals or citations in this part of the work (this should be discussed thoroughly with the supervisor/teacher).

%III. Looking at the described items that should compose the Conclusion / Final Considerations, it is difficult to imagine that this part of the work has less than one page;

\section{Summary}

% I did this in this way, that in that way... and so on

\section{Conclusions}

% Lessons learned

\section{Future Work}

%\chapter{Fundamentals}

In this chapter, the fundamental knowledge that was researched during this thesis is presented.
Section \ref{s:background} introduces the background knowledge required for this thesis and Section.
Section \ref{s:related_works} presents the current state of research on topics surrounding this thesis.

\section{Background}
\label{s:background}

This Section describes the theoretical background used in this thesis.
It covers key aspects, such as communication protocols, two-way ranging, sensor networks, and required graph theory.


\subsection{Wirerless Sensor Networks (WSN)}
\label{S:WSN}
Kevin Ashton coined the term Internet of Things (IoT) in 1999, although the idea predates this term and was before known as embedded internet or pervasive computing \cite{alaba2024evolution}.
It describes the ubiquity of digital devices and their seamless integration into the physical world and everyday life.



At the end of the 90s and during the 2000s, the production of embedded systems and sensors, in particular, rose.
This led to falling prices and sensors becoming widespread.
With the new availability of sensors, Wireless Sensor Networks (WSN) became more widespread.
While not originating in during this time, the term itself was coined in 1980, and the research community became more focused on the topic \cite{jindal2018history}.

\cite{jindal2018history} determined four main challenges in the development of WSNs:
\begin{itemize}
\item Self-organization: A large number of nodes should not require manual installation and maintenance
\item Cooperative Processing: Sensor nodes have limited memory. Evaluating, compressing, and transporting the data becomes a major challenge.
\item Energy efficiency: WSNs often operate where no power supply is available. The sensors, therefore, must run on limited, battery-based energy.
\item Modularity: WSNs should work for various applications and sensor node types.
\end{itemize}

The Ad hoc On-Demand Distance Vector Routing (AODV) Protocol was published in 1999 by E. Perkins and E. M. Royer \cite{royer1999multicast}.
It presents a routing algorithm for wireless ad hoc networks, where routing is only established when needed, and devices can be added to or leave the network at will.
Doing this takes the problem of self-organization and modularity into account.
A modified version of AODV is used in Zigbee to this day \cite{mu2017improved}. 

In 2000, Heinzelman et al. published the low Energy Adaptive Clustering Hierarchy (LEACH) algorithm \cite{heinzelman2000energy}.
Leach divides the sensors into clusters based on location.
The clusters communicate with a network head using cluster heads that collect and transmit the cluster's data.
New cluster heads are elected periodically to spread the increased energy drain from the data transfer to the network head.
LEACH is used \cite{sefati2022cluster, ghazy2023low} and improved \cite{bagherzadeh2022survey} to this day.

\subsection{Ultra Wideband}

\subsubsection{IEEE}

The Ultra Wideband (UWB) communication protocol was introduced in 2003 by the Institute of Electrical and Electronics Engineers (IEEE) as part of the IEEE 802.15.4 standard.
In 2020, updates were made to the protocol when the IEEE 802.15.4z-2020 standard made improvements to the PHY layers of UWB connections. 
It achieved this by introducing a more robust timestamping system on the PHY layer.
This is supplemented by changes to the MAC layer that allow for the exchange of ranging information.
The result is short frames that are transmitted fast between devices, leading to short bursts of communications that are fast, secure, and ideal for ranging.


UWB works by using short radio frequency pulses, resulting in a large bandwidth.
UWB is a lower-power communication form.
This prevents it from interfering with other communication forms with which it shares its wavelength, such as WLAN or Bluetooth. 
Since UWB uses very short, distinct pulses over a short range, it has been found to be useful in ranging systems \cite{hsu_2021}. 
UWB is split into high-rate pulse (HRP) UWB and low-rate pulse (LRP) UWB.
Since ranging is part of this work and LRP is generally not used for ranging \cite{hsu_2021}, I will not discuss it further in this Thesis.
Since UWB devices tend to be small and have low energy consumption, in combination with the capability of ranging and data transfer, they have become popular as Internet of Things (IoT) devices. \\ 

The standard defines the PHY and MAC layer as well as frequency bands for communication.
The 4z expansion tries to integrate UWB into the WPAN standard. In Section \ref{sec:UWB MAC} and \ref{ss:UWB_PHY} I will discuss the PHY and MAC layer.\\

The sending devices emits pulses in a pre-set band of frequencies, using short bursts to transmit the bits.
The signal forms a concave curve in this band, where the two points 10 db below the maximum power spectral density are called the lower- and upper-frequency points, see Figure \ref{f:UWB_spectrum}.
These two points must be at least 500 Hz apart.
The maximum power spectral density must be below the noise level.
This process prevents conflicts with other communications, that use a single frequency with a high power spectral density and modulate signal transmission, such as WI-FI or Bluetooth.
The UWB protocol has the added benefit of being useful for high-accuracy localization.


\begin{figure}[ht!]
\centering
\includegraphics[width=\linewidth]{graphics/UWB_spectrum.jpg}
\caption{Power Spectral Density: Bandwidth B, lower-frequency f\textsubscript{L}, upper-frequency f\textsubscript{H}, \cite{hsu_2021}}
\label{f:UWB_spectrum}
\end{figure}


\subsubsection{UWB supported Nodes}

The IEEE 802.15.4 standard distinguishes between two types of devices.
Full-function devices (FFD) are capable of connecting to multiple other devices, receiving, transmitting, and coordinating. Reduced-function device (RFD), on the other hand, can only connect to one other device and act as a worker. 
In Topological terms, RFDs can only operate as leaves, while FFDs can be any node in a network, including leaves.
RFDs, therefore, are strictly worse but make up for it by requiring fewer resources, such as memory and power.
When FFDs work as PAN coordinators, they can use short addresses to address any node.
The PAN also has a PAN identifier to help communication across multiple networks while still using the short address.
Each device also has an extended address that is not assigned by any coordinator and serves as a universal unique identifier (UUID).


\subsection{UWB MAC}
\label{sec:UWB MAC}

The Mac Layer is part of the Data link layer.
The Mac Frame is the payload of the PHY frame. It carries information about the frame type, frame format, security mechanism, addressing, and frame validation.
The Mac Layer additionally provides rules for beacon management, and channel access.

\begin{figure}[ht!]
	\centering
	\includegraphics[width=\linewidth]{graphics/general_MAC_Frame_Format.png}
	\caption{General MAC Frame Format \cite{IEEE4-2020-7}}
	\label{f:MAC Frame Format}
\end{figure}


\subsubsection{MAC Frame Format}

Figure \ref{f:MAC Frame Format} shows the composition of a UWB-MAC frame.

In the MAC header (MHR), the Frame Control Field includes information about:

\begin{itemize}
  \item the frame-type
  \item if the Auxiliary Security Header Field is used and in what capacity
  \item if additional frames will follow
  \item if an acknowledgment message is expected
  \item if the message is between different PAN-Networks.
  \item of what type the receiver is (PAN coordinator, device, PAN-Network)
  \item the used frame-format standard
  \item where to find the source address
\end{itemize}

The Sequence Number counts up, helping to keep track of the order in which frames have arrived.
The Addressing Fields carry the IDs of the sender and recipient for the frame.
The Auxiliary Security Header Field only exists if specified in the control Field.
It contains additional information needed for the chosen security method.

There are two parts to the information element (IE).
The header IE specifies additional information about the frame, for example, data formatting information or channel time allocation.
The payload IE specifies the length and data type of the payload field.
The payload contains the data that is sent.
It and the IE are of variable length, depending on the frame type and data length.

The MAC footer (MFR) marcs the end of the frame.
It only contains the frame checking sequence (FCS) that can be used to detect corrupted frames using cyclic redundancy checks.

\subsection{UWB PHY}
\label{ss:UWB_PHY}

\subsubsection{PHY Chanel}

The IEEE 802.15.4z amendment defines 16 channels for communication for HRP UWB. 
A channel is defined by its center frequency.
UWB devices can transmit on three different bands: high band, low band, and sub-gigahertz.
For each band, there is one channel that is mandatory to support if a device supports the band.
The other channels are optional, but if two devices want to communicate with each other, they need to use the same band.
The bands, 16 channels, and their ranges, and which channels are mandatory can be found in the table  (see table \ref{Table:: UWB frequency and channel assignments}).


\begin{table}[ht!]
\centering
\begin{tabular}{|l|l|c|c|}
\hline
\textbf{Channel number} & \textbf{Center frequency (MHz)} & \textbf{HRP UWB band}       & \textbf{Mandatory}  \\ 
\hline
0                       & 499.2                           & sub-gigahertz               &  \checkmark     \\ 
\hline
1                       & 3494.4                          & \multirow{4}{*}{Low band}   &                   \\ 
\cline{1-2}\cline{4-4}
2                       & 3993.6                          &                             &                   \\ 
\cline{1-2}\cline{4-4}
3                       & 4492.8                          &                             & \checkmark      \\ 
\cline{1-2}\cline{4-4}
4                       & 3993.6                          &                             &                   \\ 
\hline
5                       & 6489.6                          & \multirow{11}{*}{High band} &                 \\ 
\cline{1-2}\cline{4-4}
6                       & 6988.8                          &                             &                   \\ 
\cline{1-2}\cline{4-4}
7                       & 6489.6                          &                             &                   \\ 
\cline{1-2}\cline{4-4}
8                       & 7488                            &                             &                   \\ 
\cline{1-2}\cline{4-4}
9                       & 7987.2                          &                             & \checkmark       \\ 
\cline{1-2}\cline{4-4}
10                      & 8486.4                          &                             &                   \\ 
\cline{1-2}\cline{4-4}
11                      & 7987.2                          &                             &                   \\ 
\cline{1-2}\cline{4-4}
12                      & 8985.6                          &                             &                   \\ 
\cline{1-2}\cline{4-4}
13                      & 9484.8                          &                             &                   \\ 
\cline{1-2}\cline{4-4}
14                      & 9984                            &                             &                   \\ 
\cline{1-2}\cline{4-4}
15                      & 9484.8                          &                             &                 \\
\hline
\end{tabular}
\caption{ HRP UWB Frequency and Channel Assignments  \cite{IEEE4-2020-7, IEEE4z}}
\label{Table:: UWB frequency and channel assignments}
\end{table}

\subsubsection{Scrambled timestamp sequence}

The 4z amendment added the option to include a scrambled timestamp sequence (STS) into the frame.
The STS is a cyphered sequence that includes the timestamp and is used for ranging.
It is meant to increase the accuracy and integrity of the raging results.
Before transmitting, the receiver and sender exchange a randomly generated key.
The key is then used to encrypt the timestamp using the advanced encryption standard (AES) with 128 bits.
This ensured that the signal had not been intercepted and changed to manipulate the ranging result.
Devices that support STS are called HRP-enhanced ranging capable
devices (HRP-ERDEV).

\subsubsection{Pulse Repetition Frequency}
\label{sec:pule repetition frequency}
The pulse repetition frequency (PRF) is the frequency at wich bursts are sent by the transmitter.
The mean PRF is the average PRF while sending the payload (power switching service data unit PSDU) \cite{hsu_2021}.
The higher the mean PRF, the shorter the airtime of each frame, and it allows for faster communication.
HRP-ERDEV uses a different mean PRF than general devices.
They can work in Base pulse repetition frequency (BPRF) operating at mean PRF 64 MHz or in higher pulse repetition frequency (HPRF) mode operating above BPRF (Table \ref{f:mean PRF}).

\begin{table}[ht!]
\centering
\begin{tabular}{|l|l|l|} 
\hline
\textbf{Standard}          & \textbf{HRP UWB mode} & \textbf{mean PRF}            \\ 
\hline
802.15.4                   & Non HRP ERDEV         & 3.9 MHz, 15.6 MHz, 62.4 MHz  \\ 
\hline
\multirow{2}{*}{802.15.4z} & HRP-ERDEV BPRF        & 62.4 MHz                     \\ 
\cline{2-3}
                           & HRP-ERDEV HPRF        & 124.8 MHz, 249.6 MHz         \\
\hline
\end{tabular}
\caption{HRP UWB Mean PRF (Based on IEEE 802.15.4 and IEEE 802.15.4z, \cite{IEEE4-2020-7, IEEE4z})}
\label{f:mean PRF}
\end{table} %cite euse eigeni bricht

\subsubsection{Symbol Encoding}
UWB sends symbols by transmitting a burst of pulses that encode the symbol.
Since the pulses have clean edges, the arrival time can be measured precisely.
This leads to the burst having two ways to carry information( \cite{QorvoGettingBacktoBasics}):
\begin{itemize}
  \item Binary phase-shift keying (BPSK: Encoding zeros and ones shifting the phases of the pulses so the burst beak for one has an opposite amplitude to the other. 
Figure \ref{f:UWB_signal_description} shows the single 101 binary phase-shift keyed. 
Each bit is set twice, to detect problems with transmission.
  \item Burst position modulation (BPM): Changing the timing of the burst so it falls into a different time slot inside of the possible burst position.	
Figure \ref{f:symbol structure} shows how the burst can be placed in a BPM interval. 
The burst cannot be placed in the guard interval. 
The guard exists to minimize inter-symbol interference from the
signals taking multiple paths.
\end{itemize}

\begin{figure}[ht!]
	\centering
	\includegraphics[width=\linewidth]{graphics/HRP_UWB_PHY_symbol_structure.jpg}
	\caption{HRP UWB PHY Symbol Structure \cite{hsu_2021}}
	\label{f:symbol structure}
\end{figure}

One or both of these encoding strategies can be used in a uwb transmission.
The position of the pulses inside of the burst (see figure \ref{f:UWB_signal_description})  relative to each other can be used to detect the presence of multipath effects and adjust for them. 
Using this, precise arrival times for the whole signal can be calculated.

\begin{figure}[ht!]
	\centering
	\includegraphics[width=\linewidth]{graphics/uwb_signal_tramsmission.png}
	\caption{UWB signal transmission byte encoding, \cite{QorvoGettingBacktoBasics}}
	\label{f:UWB_signal_description}
\end{figure}

Non-HRP ERDEV uses BPM and BPSK.
Some HRP-ERDEV can use only BPSK, which uses a higher PRF and, therefore, reduces airtime.








\subsubsection{PHY Frame}
Figure \ref{f:PPDU general} shows a schematic view of a PHY frame as defined by the IEEE 802.15.4 standard.
The Synchronization header (SHR) contains the information needed to detect the signal and adjust to its parameters.
The PHY header contains meta-information about the payload and its encoding.
The PHY payload contains the data to be sent, namly the MAC frame.

\begin{figure}[!ht]
\centering
\includegraphics[width=\linewidth]{graphics/Schematic_view_PPDU.png}
\caption{Schematic view of a PHY frame defined by IEEE 802.15.4 \cite{IEEE4-2020-7}}
\label{f:PPDU general}
\end{figure}


Figure \ref{f:SHR field} shows the synchronization header, consisting of two parts.
The SYNC section is detectable by the receiver and informs it that a transmission has started.
Depending on the predefined mode, pulses of different lengths are used.
The sequence of pulses specifies a set of channels that can be used for communication.
The preamble can also be used to identify a PAN coordinator.

The SHR ends with the Start of Frame Delimiter (SFD).
It indicates that the synchronization has ended, and the coming signals will be data, starting with the PHY header. 
It also contains a timestamp, which can be used for ranging using the time difference of arrival (ToA), see Section \ref{ss:two_way_ranging}


\begin{figure}[!ht]
\centering
\includegraphics[width=\linewidth]{graphics/SHR_field_structure.png}
\caption{SHR Field Structure \cite{IEEE4-2020-7}}
\label{f:SHR field}
\end{figure}	

The PHY header contains all the information needed to read the PHY payload (see Figure \ref{f:PHR general}).
The first bit defines the data rate used during the payload transfer (see section \ref{sec: pulse repetition frequency}).
The next seven bits define the length of the frame, with a frame length of a maximum of 128 bytes.
The 10th bit shows if ranging will be used with this frame.
The next bit is reserved.
Bits 11 and 12 define the preamble duration. It specifies how many repetitions are used, which can range from 16 to 4096.
The last 6 bits are single error correct, double error detect (SECDED) bits that form a Hammock block and can be used to correct single-bit errors and detect, but not fix double-bit errors.

The last part of the PHY frame is the PHY payload (PSDU).
This contains the MAC frame, as defined in section \ref{sec:UWB MAC}.

\begin{figure}[ht!]
\centering
\includegraphics[width=\linewidth]{graphics/PHR_field_format_4.png}
\caption{General PHR Field Format \cite{IEEE4-2020-7}}
\label{f:PHR general}
\end{figure}

The 802.15.4z amendment contains optional changes to the PHY frame format if the participating devices are HRP-ERDEV devices.
Figure \ref{f:HRP-erdev frame} shows the newly allowed structures for a UWB frame.
Configuration 1 is equivalent to the already existing PHY frame.
The others additionally contain a scrambled time stamp.
This can be placed in different places after the SHR.
Since UWB can also be used only for ranging without transmitting a message, configuration three only contains the SHR and STS, without a payload.

\begin{figure}[ht!]
\centering
\includegraphics[width=\linewidth]{graphics/HRP_ERDEV_frame_structures.jpg}
\caption{HRP-ERDEV Frame Structures \cite{hsu_2021}}
\label{f:HRP-erdev frame}
\end{figure}

Additionally, the PHR can be formatted differently (see figure \ref{f:PHR 4z}. 
The reserved field and preamble duration are removed to make more space for the frame length. 
This allows more data to be sent in one frame, increasing the throughput of the UWB communication.

\begin{figure}[ht!]
\centering
\includegraphics[width=\linewidth]{graphics/HRP_ERDEV_HPRF_mode_PHR.png}
\caption{PHR Field Format for HRP-ERDEV in HPRF Mode \cite{IEEE4z}}
\label{f:PHR 4z}
\end{figure}


\subsection{Two-way ranging}
\label{ss:two_way_ranging}
The IEEE 802.15.4z UWB standard describes two ranging methods: single-sided two-way ranging (SS-TWR) and Double-sided two-way ranging (DS-TWR).
In both instances, the distance measurement is done by calculating the time of flight (ToF) of a signal sent between two devices using timestamps and multiplying it by the speed of light. 
In this Section, both SS-TWR and DS-TWR will be discussed.
Two-way ranging(TWR) refers to DS-TWR in all other parts of the Thesis.

\textbf{Single-sided two-way ranging (SS-TWR)}:
During  SS-TWR, one device sends a message to a second and measures the round-trip time (see figure \ref{f:ss_twr}).
Device A sends a message to B and records a timestamp when the message was sent, $T_{A0}$.
When device B receives the response, it also records a timestamp, $T_{B0}$.
After some delay, device B will send a response to A that contains $T_{B0}$ and the current timestamp $T_{B1}$.
On receiving the response, Device A records its timestamp, $T_{A1}$.
The round trip time $T_{round}$ can now be calculated using the timestamps from A:
\begin{equation}
	\mbox{$T_{round}$} =
	\mbox{$T_{A1} - T_{A0}$}
\end{equation}
The reply delay $T_{reply}$ is calculated using the timestamps from B:
\begin{equation}
	\mbox{$T_{reply}$} =
	\mbox{$T_{B1} - T_{B0}$}
\end{equation}

The ToF can be calculated by subtracting these values.
Since the messages were sent twice the same distance, the ToF must be halved before multiplying it with the speed of light to get the distance.
\begin{equation}
	\mbox{$distance$} =
	\mbox{$(\frac{1}{2}\cdot T_{round}-T_{reply}) \cdot c_{air}$}
\end{equation}

\begin{figure}[ht!]
\centering
\includegraphics[width=\linewidth]{graphics/schematics/SingleSidedTwoWayRanging.png}
\caption{timeline of single-sided two-way ranging (SS-TWR)\cite{IEEE4z}}
\label{f:ss_twr}
\end{figure}

\textbf{Double-sided two-way ranging (DS-TWR)}:
DS-TWR involves both devices A and B performing an SS-TWR and calculating the average between the results.
Figure \ref{f:ds_twr} shows the two separate ranging sessions.
Their result can then be combined to the average ToF for a single message:
\begin{equation}
	\mbox{$T_{prop}$} =
	\mbox{$\frac {T_{Round1}\cdot T_{Round2}-T_{Reply1}\cdot T_{Reply2}}{T_{Round1}+T_{Round2}+T_{Reply1}+T_{Reply2}}$}
\end{equation}
\begin{equation}
	\mbox{$distance$} =
	\mbox{$T_{prop} \cdot c_{air}$}
\end{equation}

The two ranging sessions can have one message overlapping.
Figure \ref{f:ds_twr_3} shows the timeline of an overlapping DS-TWR that only requires three messages.


\begin{figure}[ht!]
\centering
\includegraphics[width=\linewidth]{graphics/schematics/TwoSidedTwoWayRanging.PNG}
\caption{timeline of double-sided two-way ranging (DS-TWR)\cite{IEEE4z}}
\label{f:ds_twr}
\end{figure}

\begin{figure}[ht!]
\centering
\includegraphics[width=\linewidth]{graphics/schematics/TwoSidedRangingThreeMessages.PNG}
\caption{timeline of double-sided two-way ranging (SS-TWR) with three messages\cite{IEEE4z}}
\label{f:ds_twr_3}
\end{figure}


\subsection{K-Connected Graphs}
\label{ss:k_connected_explained}
In order to build a working positional model based on distance measurement, some background in graph theory is required.
The k-connected subgraph of a graph is a subgraph where it takes at least k removals of vertices to create two isolated subgraphs.
A graph $(V,E)$ can have multiple k-connected subgraphs.
They build the set $S_{(V,E)}$.\\
A minimally weighted k-connected subgraph of a weighted graph $S_{(V,E,w)}$, is a k-connected subgraph $(V',E',w') \in S_{(V,E)}$ that has the smallest sum of weights of all k-connected subgraphs.\\
A metric graph is a weighted graph that satisfies the triangle condition.
Meaning for any three edges $e_{AB},e_{BC},e_{CA} \in E$  that connect three verteces $A, B, C \in V$, it holds that $e_{AB} + e_{BC} \geq e_{CA}$.


Finding minimally weighted k-connected subgraphs is an NP-hard problem.
Kahn et al. \cite{khan2007simple} developed distributed approximation algorithm that finds weighted k-connected subgraphs in metric graphs.
It gives an approximation ratio of $\mathcal{O}(k\cdot log(n))$.
This means that the aproximated solution $w^{apr}$ and the optimal sollution $w^{opt}$ fullfill $w^{apr} < k\cdot log(n)\cdot w^{opt})$.
The algorithm puts all vertices in an order, which is determined randomly, and assigns each vertex a rank based on the order.
Each vertex then removes all edges, except for the k lowest weighted ones that connect it to a vertex with a higher rank.\\
There are more precise approximation algorithms to find minimally weighted k-connected subgraphs \cite{kortsarz2003approximating, kortsarz2004approximation}. However, they are centralized, meaning the graph has to be known in its entirety by one actor.


\section{Related Work} % application of concepts
\label{s:related_works}

This Section presents on overview of the literature relevant to this thesis.
This includes IoT systems used in the art world, sensor networks, and wireless ranging, showing the current practices and current state of research.


\subsection{Artwork Tracking}


Since art preservation is an old field and temperature, humidity, light, and vibrations have been known to be detrimental to most artworks, especially paintings, most research in this direction is older than 20 years \cite{mecklenburg1991mechanical, michalski1991paintings, saunders2004effect}.
Still, the invention of new technologies, such as pattern recognition using artificial intelligence, an improvement on existing tools like infrared imaging, and an active need for solutions have kept the research into artwork preservation an active field \cite{borg2020application, schito2017integrated}.
One such new technologies are sensor networks, which have become widespread in the field of art preservation \cite{shah2016customized}.


Artwork tracking during transportation has not been a significant focus in academia.
The most relevant related research was done by Fort et al. \cite{landi2022iot}.
They developed a low-cost, low-powered sensor node to track the temperature, humidity, pressure, and vibrations of artwork and wooden structures.
The sensor node would then report its findings to a remote server.
They confirmed the validity of their sensor in a series of experiments performed in a static building.
They also presented a theoretical framework for their sensor to be used in a transportation scenario, but they do not report implementing or testing this system.
Their sensor used an accelerometer to detect vibrations, and the Bosch BME280 sensor to detect pressure, temperature, and humidity.
Their sensors did not build a network and were not queried but reported their findings directly to either a Wlan router or a BLE-capable smart device.
Fort et al.'s research showes the value of low-cost sensors in detecting threats to artwork.

\subsection{Sensor Networks}

Wireless Sensor Networks (WSN) have become a central aspect if IoT.
Researchers have tried to focus on the most prevalent problems arising from the development of NSWs, mainly power management, security and privacy, data integrity, and availability \cite{gulati2022review}.


\cite{garg2021healthcare} researched WSNs outside of the controlled environment of a house. They propose a WSN that can track the vitals of mountaineers and call for help when measurements have dangerous values. 
They used an Arduino Mega board equipped with a radio transceiver, using LoRa with a star topology.


\cite{jones2021wireless} created a WSN of NRF24101 board intended to monitor linear infrastructures like deepsea wires, using radio and Wi-Fi for communication. Using deep sleep, they were able to optimize energy usage, so the sensor is predicted to last five years on battery.


\cite{spandonidis2022evaluation} used an accelerometer to detect pipeline vibrations to discover leaks. 
They used a narrowband connection for communication and GPS for localization.
Their sensors could query each other for data, to provide a more complete image of the situation.


\subsection{Wireless ranging}

\cite{li2024indoor} made an overview of publications involving positioning systems for industrial settings. 
They looked at the positioning systems in papers using RFID, BLE, UWB, Wi-Fi, and ZigBee. They found that UWB consistently reported the highest accuracy of these methods.
UWB was the least affected by multipath effects, although it was still the most common issue with this technology.


Early research on ranging using UWB was done by Gezici et al. \cite{gezici2008survey, gezici2005localization}.
These papers gave an overview of the different positioning systems for UWB, angle of arrival, received signal strength, time of arrival, and time difference of arrival.
Time of arrival and time difference of arrival were studied further in these publications, presenting error sources and mitigation tools.


Early research focused on the augmentation of UWB-ranging methods.
\cite{venkatesh2007nlos} proposed using integer programs for mitigating the error for ranging without line-of-sight.
\cite{guvencc2007nlos} tried to solve the same issue by using methods based on the statistics of multipath effects.
BiasSub and BiasRed were proposed to reduce the bias in time difference of arrival by applying of a well-known algebraic explicit solution for source localization  \cite{ho2012bias}.
\cite{fan2017performance} improved UWB ranging by eliminating random error. They did this by pre-filtering, using an anti-magnetic ring to eliminate outliers, and using the double-state adaptive Kalman filter to improve position accuracy.
Newer research has also begun incorporating neural networks into UWB positioning systems \cite{stahlke2020nlos, ridolfi2021uwb, che2020machine}.

UWB localization has been used in many applied contexts.
It has been proposed for pedestrian tracking \cite{otim2019effects}, drone flying \cite{macoir2019uwb}, robot navigation \cite{zhu2020adapted}, navigation system for visualy impared people \cite{rosiak2024effectiveness} and tracking people in buildings \cite{elbaum2024investigating}.
UWB positioning systems are particularly interesting for industrial IoT settings.
\cite{barbieri2021uwb} measured the performance of three different UWB antennas: Qorvo, Sewio, and Ubisense. They encountered many multipath-effects in such a complex environment. They mitigated this by employing a Bayesian filtering method.
\cite{belli2024cloud} used UWB positioning, in combination with Real-time kinematic positioning, to track workers while monitoring the factory. The goal was to trigger an alarm if a dangerous situation occurred.



%\chapter{Design}
\label{c:design}

This section presents the principle design of the monitoring system.
The used components are presented in the section \ref{ss:hardware}.
Section \ref{Dataflow} describes the functionalities and responsibilities of the system components.
In Section \ref{ss:network} the network topology and data-flow is discussed

\section{Hardware}
\label{ss:hardware}

This section describes the hardware used in the project. The setup consists of two distinct components: the artwork-tags, of which there are four, and one Phone that provides the interface to the user. The tag itself consists of 4 components:
\begin{enumerate}
	\item nRF52840 microcontroller
	\item DWM3000 UWB Shield
	\item DHT22 temperature and humidity sensor
	\item MPU6050 accelerometer and gyroscope
\end{enumerate}

\subsection{Microcontroler}
The artwork-tag's fundament is built by the nRF52840 DK microcontroller developed by Nordic Semiconductors. 
It is part of the nRF52 series of microcontrollers intended for development.
The nRF52840 DF is specialized for BLE communication, for which it already includes the necessary components.
It is compatible with the nRF52 Software Development Kit (SDK) developed by Nordic Semiconductors.
The SDK makes it possible to use the ble functionalities and to control the pins. 
It also includes implementations for a plethora of pin-based protocols.
It contains 58 pins, 48 of which are data pins, and 10 manage the power supply for additional modules, which include 3.5 and 5 Volt supply pins.
Thirty-two of the pins are installed the same way as the pins on the Arduino uno, making it compatible with many peripherals designed with this common board in mind, such as the dwm3000.
The remaining ten pins are enough to attach the sensors.
The nRF52840 DK includes a USB-B port that is used for powersuply. Additionally, the USB-B port is connected to two pins and is used for UART communication and debugging.
The nRF52 was chosen since it was available, and previous projects have been done with it in combination with the DWM3000 shield.
As a result, a lot of initial setup was already available.

\subsection{UWB shield}
For communication between the tags and distance measurement, the DWM3000 UWB shield developed by Qorvo was chosen.
The DWM3000 is a commonly used device for research involving UWB \cite{coppens2022overview, leu2022ghost, stocker2022performance}.
It allows low-level access but includes an SDK written in C that makes many processes transparent to the user if they wish.
The SDK uses the Serial Peripheral Interface (SPI) for communication between the shield and the microcontroler.



\subsection{Humidity and temperature sensor}

For humidity and temperature sensors I decided to use the DHT22(AM2302) produced by Guangzhou Aosong Electronic Co. \cite{AM2302}. 
It is a commonly used sensor in IoT monitoring systems \cite{ahmad2021evaluation}. 
The vendor claims a temperature range from -40\degree to 80\degree Celsius with a precision of 0.5\degree. 
\cite{ahmad2021evaluation} could experimentally confirm that errors did not exceed 0.1\degree Celsius. 
They also concluded that the sensor is slow in detecting temperature changes. 
This is also confirmed by the user manual \cite{AM2302}, which states that a read-interval of less than 2s is impossible. 

The humidity sensor can detect the full range from 0\% to 99.9\% humidity, with an advertised maximum error of 2 \% pts \cite{AM2302}.
No research confirming or denieing these claims could be found.

The DHT22 sensor uses three pins from the microcontroller: two pins for power supply and ground and one for single-bus communication.
Since no SDK for this type of communication has been built for the nRF52 board series, it had to be implemented manually by reading the high and low voltage on the communication pin, detecting headers and footers, and parsing the binary messages. 

\subsection{Accelerometer and Gyroscope}
The MPU6050 sensor produced by InvenSense Incorporated provides accelerometer and gyroscope data.
The accelerometer reports the acceleration in the three cardinal directions in meters per second.
The gyroscope reports the rotation around the three Euclidean axes in degrees per second.
In this project, the accelerometer data was not used, only the gyroscope.

The MPU6050 uses four pins: two for power supply and ground and two for communication.
The sensor communicates using the I2C protocol, a serial synchronous communication system.
The microcontroller acts as the master and would, in theory, support multiple workers on the same bus. 
Here, only the MPU6050 uses I2C and is the only worker.
While the nRF52 SDK does not supply an I2C API, it offers a Two Wire Interface (TWI) implementation compatible with the I2C protocol.
It used to offer MPU6050-specific support in older versions of the SDK. 

\subsection{Tag technical plan}
The microcontroller builds the base of the tag. 
The other devices are attached to it over the available pins.
In the nRF52 SDK, each data pin is assigned an integer value. 
These often correspond with the pin's name according to the nRF52830 DK manual, but not always.
This Thesis will use the names in the manual to describe the pins.
Pins are the method by which a microcontroller controls its peripherals.


\begin{table}[ht!]
	\centering
	\begin{subtable}[b]{.4\linewidth}
		\centering
		\begin{tabular}{l|l|l|l|l|}
			& Pin & \rotatebox{90}{DWM3000\phantom{.}} & \rotatebox{90}{DHT22}  & \rotatebox{90}{MPU6050\phantom{.}}   \\
			\hline \multicolumn{1}{|l|}{\multirow{10}{*}{\rotatebox{90}{P4}}}
			& P1.10  & \checkmark    &             &             \\
			\multicolumn{1}{|l|}{} & P1.11  & \checkmark    &             &             \\
			\multicolumn{1}{|l|}{} & P1.12  & \checkmark    &             &             \\
			\multicolumn{1}{|l|}{} & P1.13  & \checkmark    &             &             \\
			\multicolumn{1}{|l|}{} & P1.14  & \checkmark    &             &             \\
			\multicolumn{1}{|l|}{} & P1.15  & \checkmark    &             &             \\
			\multicolumn{1}{|l|}{} & GND    & \checkmark    &             &             \\
			\multicolumn{1}{|l|}{} & P0.02  &               &             &             \\
			\multicolumn{1}{|l|}{} & P0.26  & \checkmark    &             &             \\
			\multicolumn{1}{|l|}{} & P0.27  &               &             &             \\
			\hline \multicolumn{1}{|l|}{\multirow{8}{*}{\rotatebox{90}{P3}}}
			& P1.01  & \checkmark    &             &             \\
			\multicolumn{1}{|l|}{} & P1.02  & \checkmark    &             &             \\
			\multicolumn{1}{|l|}{} & P1.03  & \checkmark    &             &             \\
			\multicolumn{1}{|l|}{} & P1.04  & \checkmark    &             &             \\
			\multicolumn{1}{|l|}{} & P1.05  & \checkmark    &             &             \\
			\multicolumn{1}{|l|}{} & P1.06  & \checkmark    &             &             \\
			\multicolumn{1}{|l|}{} & P1.07  & \checkmark    &             &             \\
			\multicolumn{1}{|l|}{} & P1.08  & \checkmark    &             &             \\
			\multicolumn{1}{|l|}{} & P1.10  & \checkmark    &             &             \\
			\hline \multicolumn{1}{|l|}{\multirow{8}{*}{\rotatebox{90}{P1}}}
			& VDD    &               &             &             \\
			\multicolumn{1}{|l|}{} & VDD    &               &             &             \\
			\multicolumn{1}{|l|}{} & RESET  &               &             &             \\
			\multicolumn{1}{|l|}{} & VDD    & \checkmark    & \checkmark  &             \\
			\multicolumn{1}{|l|}{} & 5V     & \checkmark    &             & \checkmark  \\
			\multicolumn{1}{|l|}{} & GND    & \checkmark    & \checkmark  & \checkmark  \\
			\multicolumn{1}{|l|}{} & GND    & \checkmark    &             &             \\
			\multicolumn{1}{|l|}{} & N.C.   &               &             &             \\
			\hline \multicolumn{1}{|l|}{\multirow{6}{*}{\rotatebox{90}{P2}}}
			& P0.03  & \checkmark    &             &             \\
			\multicolumn{1}{|l|}{} & P0.04  & \checkmark    &             &             \\
			\multicolumn{1}{|l|}{} & P0.28  &               &             &             \\
			\multicolumn{1}{|l|}{} & P0.29  &               &             &             \\
			\multicolumn{1}{|l|}{} & P0.30  &               &             &             \\
			\multicolumn{1}{|l|}{} & P0.31  &               &             &             \\
			\hline 
		\end{tabular}
		\caption{Adruino compatible pin assignment}
		\label{table:ArdruinoPins}
	\end{subtable}
	\hspace{.1\linewidth}
	\begin{subtable}[b]{.4\linewidth}
		\centering
		\begin{tabular}{l|l|l|l|l|}
			& Pin & \rotatebox{90}{DWM3000\phantom{.}} & \rotatebox{90}{DHT22}  & \rotatebox{90}{MPU6050\phantom{.}}   \\
			\hline \multicolumn{1}{|l|}{\multirow{8}{*}{\rotatebox{90}{P6}}} 
			& P0.00  &               &             &             \\
			\multicolumn{1}{|l|}{} & P0.01  &               &             &             \\
			\multicolumn{1}{|l|}{} & P0.05  &               &             &             \\
			\multicolumn{1}{|l|}{} & P0.06  &               &             &             \\
			\multicolumn{1}{|l|}{} & P0.07  &               &             &             \\
			\multicolumn{1}{|l|}{} & P0.08  &               &             &             \\
			\multicolumn{1}{|l|}{} & P0.09  &               &             &             \\
			\multicolumn{1}{|l|}{} & P0.10  &               &             &             \\
			\hline \multicolumn{1}{|l|}{\multirow{18}{*}{\rotatebox{90}{P24}}} 
			& P0.11  &               &             & \checkmark  \\
			\multicolumn{1}{|l|}{} & P0.12  &               &             & \checkmark  \\
			\multicolumn{1}{|l|}{} & P0.13  &               & \checkmark  &             \\
			\multicolumn{1}{|l|}{} & P0.14  &               &             &             \\
			\multicolumn{1}{|l|}{} & P0.15  &               &             &             \\
			\multicolumn{1}{|l|}{} & P0.16  &               &             &             \\
			\multicolumn{1}{|l|}{} & P0.17  &               &             &             \\
			\multicolumn{1}{|l|}{} & P0.18  &               &             &             \\
			\multicolumn{1}{|l|}{} & P0.19  &               &             &             \\
			\multicolumn{1}{|l|}{} & P0.20  &               &             &             \\
			\multicolumn{1}{|l|}{} & P0.21  &               &             &             \\
			\multicolumn{1}{|l|}{} & P0.22  &               &             &             \\
			\multicolumn{1}{|l|}{} & P0.23  &               &             &             \\
			\multicolumn{1}{|l|}{} & P0.24  &               &             &             \\
			\multicolumn{1}{|l|}{} & P0.25  &               &             &             \\
			\multicolumn{1}{|l|}{} & P1.00  &               &             &             \\
			\multicolumn{1}{|l|}{} & P1.09  &               &             &             \\
			\multicolumn{1}{|l|}{} & GND    &               &             &             \\
			\hline
		\end{tabular}
		\caption{Non-Adruino compatible pin assignment}
		\label{table:otherPins}
	\end{subtable}
	\caption{Pin assignments and compatibilities}
\end{table}

Some pins are intended for power supply.
On the NRF52840, these pins are located in section P1; see table \ref{table:ArdruinoPins}.
The three VDD pins supply electricity with a Voltage of 3.5 Volts.
A secondary power supply that uses 5 Volts is also available.
What voltage is needed depends on the peripheral.
In this case, the DHT22 runs on 3.5 Volts, while the MPU6050 is made for 5 Volts.
The P1 section also contains two ground pins that need to be connected to the peripherals and a reset pin to restart the microcontroller.
The last pin is not connected (N.C.).
There are additional ground pins in sections P4 and P24 of the board.

The other pins are called data pins.
These pins can transfer data and be used for communication by using voltage modulation.
The nRF52840 has an I/O voltage of 3.3 Volt.
This means that a voltage of 3.3 Volt corresponds to a \textit{Logic high} and 0 Volts represents a \textit{Logic low}.
This allows the data pin to transfer communication in a binary encoding.
How a signal is interpreted is defined by the used communication protocol.
The MPU6050, for example, uses the I2C protocol and uses two datapins.
The protocol states that one pin is used for a serial clock, and the other pin transmits data.
For the data transmission, the protocol defines what a package looks like.
This includes the start condition, the voltage characteristics that signal the beginning of a package, addressing, data encoding, acknowledgments, and stop condition.

The DHT22 sensor does not use a given communication protocol.
It uses one data point to report its sensor data.
How that data is encoded to high and low voltage is specified in the user manual \cite{AM2302} and has to be implemented manually.

The DWM3000 shield is mounted on the 32 pins for Arduino one connections. 
All pins are forwarded and can be used by other devices in a common Arduino-stackable style.
If they are data pins, they will share the data.
Table \ref{table:ArdruinoPins} shows which devices use which pins.
The only pins shared by multiple devices are power and ground pins.
The microcontroller supplies enough power to support this.

The sensors are attached to the same power source and ground as the shield but use different data pins.
The DWM3000 leaves enough pins unused that both sensors could be attached to them.
Since it is not visible which pins the shield leaves free, it was decided to use data pins that are not attached to the DWM3000 for the DHT22 and MPU6050.
Table \ref{table:otherPins} shows how the sensors are connected to the remaining open pins.

\section{Architecture}
\label{ss:dataflow}

In order to discuss how the dataflow works, first, Section \ref{ss:responsibility} will establish what services are implemented in each part of the system.
Section \ref{ss:dataflow} will explain what triggers events and how they are handled inside the system.

\subsection{Responsibilities}
\label{ss:responsibility}
The system consists of the tags, the sensor network, and the phone.
These parts all have their own responsibilities.

\textbf{Tag:} 
The tag is responsible for managing its sensors. 
It has to do the correct setup and convert its output into an understandable form.
The tag can perform ranging with all its neighbors.
Additionally, the tags are responsible for searching for networks to join and reacting appropriately to network requests, be those queries for sensor data, ranging requests, or network management jobs. 
The tags provide a unique, secure universal identifier to be used by queries or the network.
How this is done is part of the certified project and will not be discussed in this thesis.
The tag is also responsible for its own power management.
This is not the focus of this thesis and will only be mentioned when relevant.
A guideline on power management will not be provided here.

\textbf{Network:} The network is responsible for keeping track of all tags taking part in the network. 
It offers a joining protocol for new devices and remains stable when devices leave or become unavaliable. 
It offers the possibility for phones to connect to the network. 
It ensures quieries from phones get transported to the correct tag and the answers to the correct phone.
It ensures a network topology that corresponds to a graph that is at least 3-connected.
On request, it returns a list of devices connected to the phone.

\textbf{Phone:} The phone connects to the network via the provided method.
It offers a graphical user interface (GUI) for the driver to use.
The GUI offers a method for the driver to set the acceptable ranges for all sensor data.
Additionally, it offers a method to set query interval-length.
The phone is responsible for querying sensor data for each tag and measuring once at each interval.
The phone has to evaluate the answer.
The phone has to report the results to the driver using the GUI.
If a parameter falls outside of the acceptable range for its type, the phone is responsible for alerting the driver to this fact.

The Certify project also plans to collect the sensor data on remote servers using a 4G connection.
The plan is to equip each tag with antennas to allow it to send the data directly to the server itself.
Since this is not a part of this thesis, the responsibility for the tag to do this was not added to a list.
A known problem with this plan is, that a 4G connection is not always possible.
Since small tags have very limited memory, the plan to store the sensor data on the tag is not feasible.
If the setup presented in this thesis is used, it would allow for the storage of the data on the phone, which has a much larger memory.
This, again, was not added since it is not part of this thesis.


\subsection{Dataflow}
\label{ss:dataflow}

\begin{figure}[ht!]
	\includegraphics[width=\linewidth]{graphics/schematics/obervation_loop.png}
	\caption{Sequence diagram of setup and observation loop. Setup is performed once, and the observation loop repeats until it is stopped.}
	\label{f:observation_loop}
\end{figure}

Section \ref{s:network} describes how a tag connects to the network.
Figure \ref{f:observation_loop} shows a sequence diagram of the setup and main observation loop of the system.
At the top, the communicating parts are listed.
\begin{itemize}
	\item Human is the driver of the truck
	\item Phone is the phone used by the Human
	\item Network consists of all the used tags and the network they build.
	\item Connected Tag is part of Network but is listed separately. It represents the tag that is communicating to the phone
	\item Tag-j is also part of Network. It represents the tag that is queried during the observation loop
\end{itemize}
Phone and Human communicate by using a GUI. 
Phone and Connected Tag communicate using BLE. 
Every communication inside Network happens using UWB. This includes the communication between the Connected Tag, Network, and Tag-j. \\
When Phone wants to connect to Network, it looks for advertised BLE devices.
It then displays the devices to Human and lets them pick one.
The phone then pairs with the chosen tag, making it the Connected Tag and Phone's connection to Network.
Once connected to Network, Phone will prompt Human to enter the parameters.
These consist of:
\begin{itemize}
	\item Upper and lower limit for sensor data, like temperature and humidity
	\item Maximal displacement value for distance and gyro. These values represent the maximal difference in registered values that is allowed for positional measurements.
	\item Time between measurements. This gives the time period that will pass between measurements for each device and measurement type.
\end{itemize}
Once the parameters are chosen, Human can start the observation.\\
Each iteration of the observation loop begins with a call to Network for a list of all tags currently in Network.
Since the tag network is a dynamic sensor network, the tags in Network can theoretically change. 
In practice, this should only happen when artwork is loaded/unloaded or a tag becomes faulty.
The request for the list is transmitted to Connected Tag over BLE, which then queries Network for all connected devices.
The response is returned to Phone.
Phone then starts a nested loop, iterating over the list of tags and metrics captured by the system.
For each measurement and tag combination (i,j) Pphone contacts Connected Tag for the value, which in turn queries Network.
Once the message arrives at Tag-j, Tag-j gets measurement i. 
In the case of sensors, this entails contacting the sensor and requesting a value.
If metric i is a distance measurement, tag j will commence a two-way ranging operation over UWB with all its registered neighbors and will report the list of distances, together with the tag addresses they correspond to.
Metric i is then transported over Network back to Connected Tag and finally to Phone.
Phone must then evaluate the retrieved data. \\
During the evaluation process, Phone creates an evaluated measurement and marks it as problematic or unproblematic.
What the evaluation looks like depends on the metric.
\begin{itemize}
	\item For most metrics, like humidity and temperature, the evaluated measurement is equivalent to the received measurement. It is then checked if the measurement falls into the acceptable measurement parameters set by Human.
	If it does not, the evaluated value is marked as problematic.
	\item Some metrics require comparison to the previous data. The gyroscope reports the current orientation of the tag.
	This is then compared to all previous measurements, and the maximal angular difference forms the evaluated measurement.
	If the evaluated metric is bigger than allowed by the set parameters, the measurement is marked as problematic
	After evaluation, the original measurement is added to the list of previous measurements.
	\item The distance measurement has a unique evaluation process, which is described in Section \ref{ss:distance_eval}.
\end{itemize}
Once the data evaluation is done, the evaluated measurement is presented to the user over the GUI, together with the address of the tag it belongs to.
If the evaluated measurement is problematic, the driver will be alarmed.


\subsubsection{Distance evaluation}
\label{ss:distance_eval}
The goal of the distance evaluation is to build a working model of where every tag is.
To achieve this, a quadratic program is solved to get the coordinates of all tags.
The steps to do this are as follows:
\begin{enumerate}
	\item Get a list of all current tags, $T:=\{ t_1, t_2, ... \}$.
	\item For each tag, get the last known distance measurements and put it into a set $S_D:=\{ (t_i, t_j, d_{ij}) \} $, where $t_i$ is the tag which measured, $t_2$ the tag that was measured to and $d_{ij}$ the distance measured.
	\item If a tag has no distance measurements, remove it from the list.
	\item Assign each tag $t_i$ a position in a 3D coordinate system, $(x_i,y_i,z_i)$
	\item Pick one random tag $t_{o}$.
	\item Set the values $x_{o},y_{o},z_{o}$ all to 0.
	\item Create the objective function: $L(X,Y,Z) = \sum\limits_{t_i, t_j, d_{ij} \in S_D}|(x_i-x_j)^2+(y_i-y_j)^2+(z_i-z_j)^2-d_{ij}^2|$. For x,y, and z, use variables for all but the six values set in step (6).
	\item Solve the quadratic program consisting of the Objective function L, and no constraints.
\end{enumerate}
Quadratic Programs, in general, are NP-Hard, but Quadratic Programs with a convex function can be solved efficiently.
$(a-b)^2$ and $c$ are convex functions.
The sum of a convex function is always a convex function.
The objective function in (7) only sums up convex functions and is, therefore, convex.
The quadratic program can, therefore, be solved efficiently.\\
By setting the values of the tag $t_{o}$ to zero, the results of the quadratic function become grounded.
It is not strictly necessary, but without it, the returned solution could have values anywhere in the Euclidean space.
The solution will place the other tags near that region by setting one tag to the coordinates at the origin.
There are still an infinite amount of solutions to this quadratic function since all solutions can be rotated around any axis and still return the same objective function.

\begin{figure}[ht!]
	\begin{subfigure}{.4\linewidth}
		\centering
		\includegraphics[height=150px]{graphics/schematics/connected_dots.png}
		\caption{3-edge connected graph}
	\end{subfigure}
\hspace{1cm}
	\begin{subfigure}{.4\linewidth}
		\centering
		\includegraphics[height=150px]{graphics/schematics/connected_dots_k_connected.png}
		\caption{2 connected graph}
	\end{subfigure}
	\caption{ Left: Five dots, all having at least two connections, still blue can move independently. Right: minimal 2-connected graph, no movement possible. }
	\label{f:connected_dots}
\end{figure}

For a point to be clearly placed in Euclidean space, three distances to other points must be known.
This alone is not sufficient to ensure unique results.
The left of Figure \ref{f:connected_dots} illustrates this point in two-dimensional space.
Every circle is connected to two others, but the blue circles can still move without the whole figure moving.
What is needed to keep every point static is for known distances and tags to build a four-connected graph (three in two dimensions).
The left of Figure \ref{f:connected_dots} shows a solution to the problem on the right by creating a three-connected subgraph.

Once the coordinates for all tags are found, they are compared to previous results.
For each tag, the phone calculates how much it has moved.
The evaluated measurement is the distance of the tag that has moved the most.
If the evaluated measurement is larger then the maximal allowed displacement, the measurement is problematic.



\section{Network}
\label{s:network}

For the presented network to work, tags need to be ranked.
This means that for each tag pair $i,j$, one can either say that $rank(i)<rank(j)$ or $rank(j)<rank(i)$.
To achieve this, the universal unique identifiers are used.
No matter what form the UUID has, it can be converted to an integer, by simply interpreting its binary code as one.
Since the UUIDs are unique, no to tags will have the same resulting integer.
When referring to the rank of a tag in this section, the integer representing the UUID is intended.

While not connected to a phone, the tags inside the truck form a decentralized mesh using UWB for communication.
Each tag keeps two lists: a list of known devices and a list of neighbors.
When a new tag joins the network, it sends a joining request over UWB, containing its UUID, using a weak signal.
All tags in the network that receive this request add the new device to their list if known devices. 
If the new device also has a higher rank, they additionally add it to their list of neighbors.
They then answer by sending their own UUID and adress back to the new tag.
By waiting an amount of time that correlates with their UUID, the tags in the network can ensure that their answers don't overlap.
The new tag adds the received addresses and UUIDs to its known device list. If the added tag's rank is also higher than the new tag, it will add it to the list of neighbors.
If the new tag now has four neighbors, it stops. Otherwise, it will repeat the process with an increasingly stronger signal until it has either found four tags with a higher rank or reached maximum signal strength.
Afterward, it starts advertising its BLE connection.
This concludes the network joining process.


A user with a phone can connect to any of the advertised BLE connections.
Once that happens, the tags in the tracks will switch from their decentralized mesh to a star topology, with the connected device serving as the coordinator.
The coordinator will inform all tags about their new status by sending a message using a strong UWB signal.
The tags will then acknowledge this message in order of rank.
The tags in the network will still keep their stored neighbors and known devices.
The coordinator records a list of all acknowledgemtns, thus creating a list of all devices in the network.


The phone can request the list of all tags from the coordinator.
The phone can now also query the tags in the truck by sending the query to the coordinator over BLE, which then will pass it directly to the appropriate tag using BLE.
For all sensor data, this is a simple call-and-response request. \\
If a distance measurement is queried, a tag take the following steps:
\begin{itemize}
	\item It conducts a UWB two-way ranging session with each tag in the neighbor list.
	\item It reports those results to the coordinator tag.
	\item It orders all received distances.
	\item It keeps the tags with the four lowest distances and deletes the rest from the neighbor list.
\end{itemize}
The first time a distance is requested, the tag will perform more ranging sessions than necessary to build a 4-connected graph.
Afterward, it only performs ranging with four other tags unless a new device is added.
Suppose a ranging session does not report a result because a tag left or became unavailable. In that case, the tag adds ads the tag with the shortest previously measured distance and higher rank from the list of known devices back intro the list of neighbours.


This design mirrors the algorithm proposed by \cite[khan2007simple] and presented in Section \ref{ss:k_connected_explained}.
It creates an approximation of a minimally weighted k-connected subgraph based on the measured distances.
This is allowed since the distances are in Euclydean space, which, when mapped to a graph, forms a metric graph.
As discussed in section \label{ss:distance_eval}, a four-connected graph is needed to uniquly identify the position of each tag.
The graph should be minimally weighted so that measurements are between tags that are as close as possible to each other.
This reduces the multipath effect and theirfore increases precision.


If the tags are not connected to a phone and report their data to a remote server, they can still use the same distance measurement to approximate the k-connected subgraph.
The quadratic program can then be calculated on the server.
%\chapter{Implementation}
\label{c:implementation}

In this section, the implementation that was used for the experiment is discussed.
In section \ref{s:tag} the implementation of the tags is presented.
Section \ref{s:app} is about the implementation of the App.

\section{Tag}
\label{s:tag}
The software of the tags consits of n modules:
\begin{enumerate}
	\item Temperature and humidity sensor
	\item Gyroscope
	\item UWB network
	\item Two way ranging
	\item BLE communication
	\item Job handler
\end{enumerate}
The following subsections will discuss the first five modules, followed by how they interact using the job handler module.
The section \ref{ss:combination} discusses chalanges from combining these modules and how they were solved.

\subsection{Temperature and humidity module}
\label{ss:temp_hum_module}
This module is responsible for managing the DHT22 humidity and temperature sensor.
It is responsible to setup the sensor during initial startup and to provide the sensors measurements when queried.
The DHT22 sensor communicates using only one data pin, pin 13, which will be referd to as the data pin in this section.
Dmitry Sysoletin created an implementation \ref{sysoletin2021nrf52_dht11} for the DHT11 sensor together with the nRF52840 board that build the basis for this implementation, by addapting it to the DHT22 and adding functionalities needed by the job handler module.


Since the DHT22 is a very simple sensor, using single bus comunication, not much setup is needed.
The evaluation of the sensor data requires that the voltage of the pin is read out in pre-defined intervals, when reading the sensor data.
To do this, a clock is required.
This resource has to be reserved an initiated at startup.
This is the only setup that is required for the DHT22 sensor.


To initiate a sensor-read the voltage of the data pin is set to 0.
When the sensor is in standby mode, the data pin is on \textit{logic high}, and when set to \textit{logic low}, the sensor will respond with a read of its current value.
A schematc view of a sensor read of the DHT22 can be seen in Figure \ref{f:dht22_signal}.
The temperature and humidity module will then check the Pin State in intervals of 5ms, until a \textit{logic low} is registered, signalling that the sensor has registered the request. 
The module will now monitor the pin state, waiting for \textit{logic low} followed by a \textit{logic high}, this beeing the start condition of the data transfer. \\
The data is transfered in five chunks of eight bits.
Each bit is preceeded by a prolonged \textit{logic low} state, that is detected by the module
The module then proceeds to write the state of the data pin into a 8-bit buffer, \textit{logic high} corresponding to a 1 and \textit{logic low} to 0.\\
Once all five chunks are read, the communication has ended and the module can verify the data.
The first two bytes correspond are combined to form the temperature information in celcius, the second and third form the humidity.
Both values are multiplied by 100 and stored in a 16-bit integer. This doesn't loose data, since the sensor only measures up to a precision of 1 after the decimal point.
The data beeing stored in an integer help with data transfer.
It will be converted back on the phone.
The fith chunk contains the parity and is used to accept or reject the humidity and temperature values.
If the process fails at any state, -100\degree C is returned for the temperature and -100% for humidity.
These form both impossible values, since humidity can't be negative and the DHT22 sensor can only detect temperatures as low as -20\degree C.


\begin{figure}[ht!]
\centering
\includegraphics[width=\linewidth]{graphics/DHT22_signal.png}
\caption{Signal of a DHT22 sensor-read as presented in the manual \cite{AM2302}.}
\label{f:dht22_signal}
\end{figure}

\subsection{Gyroscope}
\label{ss:gyro_module}
This module manages the MPU6050 gyroscope and accelerometer.
It is responsible to setup the sensor and report its result.
An implementation for the MPU6050 was present in the nRF52 15.3.0 SDK, but is no longer avaliable for the nRF52 17.1.0 SDK, used in this project.
The old implementation was ported to this project.
This consisted of replacing deprecated parts of the SDK with updated ones and adding newly required flags to the build.


MPU6050 sensors use the I2C communication protocol.
The nRF52 SDK does not include an implementation for this protocol, but has a Two Wire Interface (TWI) implementation that is compatible with the I2C protocol.
During startup the TWI module has to be initialized.
This is handled by the SDK, but requires some parameters to be passed.
\begin{itemize}
	\item The Serial Clock Line (SCL) defines what pin will be used for the clock shared in the TWI. This implementation uses pin 11.
	\item The Serial Data Line (SDA) defines which pin is used for the data communication. Pin 12 was used.
	\item The frequency which the TWI uses. It is defined in MPU6050 data sheet, and is 100 kHz \cite{MPU6050}.
	\item The Interrupt priority is a rank that determins, how easyely this process can cause an interrupt. It is set to high.
\end{itemize}
After the TWI service is initiated with these parameters, it is enabled, ensuring that its resources are locked and can not be used by other services.


Afterwards the results from the sensor can be read using the TWI service again.
The TWI-TX requires the adress of the read device and a registry where to write the MPU6050 datasheet \cite{MPU6050}.
The adress of the sensor is the same for all MPU6050 sensors and can be found in the manua
It sets a flag to true once the sensor has writen the data, which then can be read using the TWO-RX function.
The result consist of three 16-bit integers, representing the angular velocity arround the X,Y and Z axis, shown in figure \ref{f:MPU6050_orientation}.\\
Returning this data when queried has only limited use.
It represents a measurement of the current situation.
The caller is more interested what happened since the last query.
Two different implementations for the read of the gyroscope were used during the experimental phase of this thesis.
One would try to return the current orientation of the tag. This read will be called the \textit{orientational read}.
The other would return the maximal registered angular velocity since the last read. This will be called the \textit{angular velocity read}.


\begin{figure}[ht!]
\centering
\includegraphics[width=200px]{graphics/MPU6050_orientation.png}
\caption{Schematic view of the MPU6050, showing the direction of the three axis X,Y,Z.}
\label{f:MPU6050_orientation}
\end{figure}



To achieve the orientational read, three orientational variables $x_{angle}$, $y_{angle}$ and $z_{angle}$ keep track of the current rotation around their corresponding axes.
During setup, all three angles are set to zero.
The MPU6050 is read out periodically in between calls.
The elapsed time since the last read is multiplied with the angular velocity at this moment arround the axis and is added to the orientational variables.
When the gyroscope module is queried for its measurement, $x_{angle}$, $y_{angle}$ and $z_{angle}$ are returned.


The angualr velocity read is achieved in a similar manner.
Three angular velocity variables $x_{max}$, $y_{max}$ and $z_{max}$ are created and set to zero during initiation.
The MPU6050 is read out periodically and its values are compared to the angular velocity variables.
If any of the angular velocity values is smaller in absolute magnitude than the corresponding read value, it is replaced by that read value.
When the the gyroscope module is queried, the values of $x_{max}$, $y_{max}$ and $z_{max}$ are returned.
The angular velocity variables $x_{max}$, $y_{max}$ and $z_{max}$ are then set to zero again.

\subsection{Network}
\label{s:network}
The network module is responsible for the managment of the network.
This consists of: sending requests to join a network, managing requests to join a network, keeping track of its neighbours, transmitting messages and sending messages.
Since only four devices were used in this implementation, the processes for the network are much more simplified, then presented in design chapter.
A 4-connected minial graph of 4 verteces must nececarily include that all the nodes are fully connected.
This leads to a simplified network architecture.
Since this implementation was build to run experiments and not to be used in real-world applications, a lot of security measures were canceled.
Messages are not encrypted and devices are not authenticated.
All messages are assumed to reach their destination and no devices is expected to become unavaliable. \\
The Network-module is based on the implementation of \cite{degkwitz2023ultrawideband}.
It is based on published examples from Qorvo, the producer of the DWM3000 shield.
It uses the DWS3000 SDK to communicate with the DWM3000.


The DWM3000 uses the Serial Peripheral Interface (SPI) protocol.
This requires some resources that have to be reserved and some configurations that need to be set.
This is the first thing that happens during the setup of the UWB network.
Next the interrupt-priorities and the communication speed of the SPI connection are configured.
Then the DWM3000 is reset, to insure no cross-effects from previous sessions are possible.
Then the board is told to initialize.
After that the used configurations are send to the shield.
This includes information like chanel number, preamble codes, data rates and header modes.
The SDK contains many pre-defined configurations.
All configurations that allow for RX and TX and that use scrambled timestampt (STS) work for this usecase.
It is crucal that all tags use the same configurations.
For this implementation the same configurations were used, as in \cite{degkwitz2023ultrawideband}.
The configurations can be seen in table \ref{table:DWM_settings}.
The setup finishes with initiating the LEDs, that serve no critical service, but are usefull for debugging.



\begin{figure}[ht]
\caption{Configurations of the DWM3000 for UWB communication}
\begin{longtable}{|l|c|}
\hline
\textbf{Description} & \textbf{Value} \\
\hline
\endfirsthead
\hline
\endfoot
Channel number & 5 \\
TX preamble length & 128 symbols \\
RX preamble acquisition chunk size & 8 chunks \\
TX preamble code & 9 \\
RX preamble code & 9 \\
SFD type selection & 4z 8 symbol\\
Data rate & 6.8 Mbits/s \\
PHY header mode & standard PHR mode \\
PHY header rate & standard PHR rate \\
SFD timeout & 129 \\
STS mode & enabled \\
STS length & 128 bits \\
PDOA mode & off \\
\hline
\end{longtable}
\label{table:DWM_settings}
\end{figure}


The Certify project uses unique, falsifiable identifiers for its tags.
Since this is not avaliable for the tags used here, the device-ID was used instead.
It serves as a 8-bit long adress for the purposes of this implementation.
Each tag also keeps a list of all known adresses, called neighbours.


The adress \textit{0x3F} was used, when a tag wants to join a network.
This was chosen since none of the used devixes had this device-ID, and it corresponds to a question mark when using ASCI encoding.
When a tag wants to joind a message, it sends the adress \textit{0x3F}, followed by the message 'findnet', and its own adress.
It then start listening for answers.
If the listening timed out without any answers, it sends the message again.


For the network to function, the receiving and sending of messages is critical.
The UWB listener function from project \cite{degkwitz2023ultrawideband} was modified.
It waits for a listened message from the shield.
If it receives a message, it coppies it to a buffer.
It then checks the first bit of the message for the receiver adress.
If the receiver adress is equivalent to the tags own adress, it passes the message on to the job-handler module for further evaluation.
Otherwise the message is discarded.
An exception is made, if the receiver-adress is "\textit{0x3F}", indicating that a tag is looking for a network.
In that case, the network module adds the tag to the list of neighbours.
It then waits for a time proportional to its own adress, before continuing.
Since adresses are unique, this ensures that no two tags responde to the new tag at the same time.
Afterwards it sends a new message, beginning with the adress of the new tag, followed by the string 'NEW' and its own adress.
This way it can be added to the neighbours of the new tag as well.\\
For sending messages, the implementation of \cite{degkwitz2023ultrawideband} was modified.
It sets the DWM3000 to TX, passes a int-buffer and lets it transmit, before returning to RX mode.
Do to limitations discussed in section \ref{ss:combination}, the message length could not exceed 10 bytes. 


\subsection{Two-way ranging}
\label{ss:two_way_ranging}
The two-way ranging module is responsible for measuring its distances to the tags in the neighbourhood.
Since it also uses the DWM3000 shield, it requires no additional setup.


When the two way ranging module gets a distance request, it loops over the list of neighbours, performing two-way ranging with each of them.
First it sends a prepare-rainging request to the neighbour it wants to performe ranging with, before performing the ranging.
It then sends the result back over the network to the requesting tag with the following format: 
\begin{equation}
	\mbox{$a_r$DST$a_ta_ncd_{tn}$}
\end{equation}
with
\begin{itemize}
	\item $a_r$: The adress of the requesting tag.
	\item DST: The string "DST", indicating the purpose of the message.
	\item $a_t$: The own adress of the tag performing the measurement.
	\item $a_n$: The adress of the neighbour that the distance was measured to.
	\item $c$: A boolean. If false, this is the last neighbout measured for this query.
	\item $d_{tn}$: The distance to measured. 
\end{itemize}
The reason for each measurement triggering its own response is the message-lenght limitation mentioned in section \ref{s:network}.

When a tag receives a prepare-rainging request intended for another device, it enters a short sleep.
This is because rainging envolves multiple messages beeing send netween both participants.
This would unnecesarily drain energy from the tags that are not envolved.
Because of that they sleep for the expected duration. \\
If the tag is the entended receiver for the prepare-rainging message, it will enter the preparation part of the two-way ranging module.
If will function as deivce A in respect to figure \ref{f:ds_twr_3}.
In a first step, it will clear all RX and TX buffers.
It then sets the expected RT and TX antenna delays, $d_{rx}$ and $d_{tx}$.
They represent the expected time loss during receiving or transmitting messages and are device specific.
These delays will automatically be taken into account, when calculating the timestamps.
It then sends the first polling message and imideatly starts waiting for a response.
The polling message is a constant string with no changing data.
The DWM3000 will automatically store the transmission and reception timestamps, their is no need to retreive it right away.
When the response is received, it checks if it is the expected response.
If it is, the two timestamps $T_{TX_1}^A$ and $T_{RX}^A$ are retireved.
The final transmission time $T_{TX_2}^A$ is calculated by adding a constant $c_A$ to $T_{RX}^A$:\begin{equation}
	\mbox{$T_{TX_2}^A$=}
	\mbox{$T_{RX}^A+c_A$}
\end{equation}
The final message is then prepared, containing all three timestamps $T_{TX_1}^A$, $T_{RX}^A$ and $T_{TX_2}^A$.
The message is loaded into the message buffer of the DWM3000 and a delayed transmission is started.
The delayed tranmission takes the timestamp $T_{TX_2}^A$ and will start the transmission once that timestamp is reached.
Afterwards all caches are cleaned and the tag returns to its previous state, listening for requests.

The tag that performs the ranging roccesponds to device B in figure \ref{f:ds_twr_3}.
Once it has sent the the prepare-rainging message to its neighbour, it will enter the revceiving part of the two-way ranging module.
As device A, device B will also start by settings its antenna delays $d_{rx}$ and $d_{tx}$ and clear all its RX and TX buffers.
It will then start polling for a message.
Once a message from device A is received and validated, it will retreive the timestamp when the message was received, $T_{RX_1}^B$.
Device B will add a constant $c_B$ to this timestamp to get $T_{TX}^B$:
\begin{equation}
	\mbox{$T_{TX}^B$=}
	\mbox{$T_{RX_1}^B+c_B$}
\end{equation}
It will then start a delayed transmission for the response message at $T_{TX}^B$.
The response is a constant string without any data.
Once the response is sent, device B starts to listen for messages again.
When the final message is received from device A and validated, $T_{TX_1}^A$, $T_{RX}^A$ and $T_{TX_2}^A$ are extracted from the message.
Device B also retreives its final timestamp, $T_{RX_2}^B$.
Once this is done, the time of flight for a single message can be calculated, and from that the distance:
\begin{align}
    T_{round1} &= (T_{RX}^A - T_{TX_1}^A) \\
    T_{round2} &= (T_{RX_2}^B - T_{TX}^B) \\
    T_{reply1} &= (T_{TX}^B - T_{RX_1}^B) \\
    T_{reply2} &= (T_{TX_2}^A - T_{RX}^A) \\
    ToF^{AB} &= \frac{(T_{round1}\cdot T_{round2}) - (T_{reply1}\cdot T_{reply2})}{T_{round1} + T_{round2}) + (T_{reply1} + T_{reply2}} \\
    distance &= ToF^{AB} \cdot c_{air}
\end{align}
The distance is then returned, all caches cleared and the module continues with the next distance measurent, if any are remaining.


The TX and RX antenna delay $d_{rx}$, $d_{tx}$ are different for each device.
Qorvo supplies a default value, but it is the same on all devices.
Since te antenna delays are multiplied with the speed of light, even small mistakes in calibration can lead to big errors.
According to qorvo, without the calibration of antenna delays, a measurement can be off by up to 40 cm \cite{DWM3000Calib}.
This will be a constant bias and not change over measurements.\\
Qorov has published a manual on how to calibrate their devices \cite{DWM3000Calib}.
They have not published a codebasis that implements this process.
The calibration process published by Qorvo required things that were not part of this project:
\begin{itemize}
	\item A synchronized clock, shared over all devices, without significant clockdrift
	\item A UART connection to a computer
	\item A pipeline performing statistical analysis and coordinating the devices.
\end{itemize} 
Since implementing this calibration process whould have been out of scope for this thesis, a simpler version was designed.
The tags were set up in a theathedron, so each tag was 30 cm apart from each other.
Then one tag would perform two way ranging with another tag, chosen at random.
The result would be shared between both tags.
If the result was larger than 30 cm, $d_{rx}$ or $d_{tx}$ would be chosen at random and increased.
If it was lower, $d_{rx}$ or $d_{tx}$ would be increased.
Then the second tag would start a new ranging session with a random tag.
This system was left running for over one hour, until all distances measured were in the range of [27 cm, 33 cm].


\subsection{BLE}
\label{ss:ble_module}
The BLE module is responsible for the communication between the UWB network and the phone.
It advertises the tag to the phone and receives messages from the phone and sends messages to the phone using BLE.
The nRF52840 microcontoler is equiped with a antenna with BLE capabilities.
The nRF52 SDK includes libraries for the managment of this antenna.
It also includes the \textit{ble{\_}app{\_}uart} example project.
This project offers advertises a ble connection, handles the paring process.
Once connected, it forwards all incomming comunication to a USB-UART mdoule connected to a computer.
Input from the computer ver USB-UART is sent as a message to the paired device.
The  \textit{ble{\_}app{\_}uart} example project was took as a basis to build the BLE-module.


The nRF52 SDK for BLE requires the use of the S140 SoftDevice.
The S140 SoftDevice is a BLE protocol stack that can be used for the 811, 820, 833 and 840 series of nRF52 boards.
In order for the SoftDevice to be avaliable, a memory 156 kilobyte segment of memory has to be reserved for it, starting at  memory segment 0x0.
The SoftDevice then has to be flashed to the board.

During startup, the BLE module has to initialize a few services and reserve some resources.
Firstly a nRF clock has to be reserved for the BLE module.
Then the powermanagment for the SoftDevice has to be initiated, before the BLE stack inside the SoftDevice can be initialized.
Next the Generic Access Profile (GAP) and the Generic Attribute Profile (GATT) have to be prepared.
The information what functions to call when the SoftDevice receives data has to be set, as well as the advertized name, the UUID, timeout durations and what to do on faults.
The advertized name was left unchanged from the \textit{ble{\_}app{\_}uart} example, "Nordic{\_}UART".\\
Once the SoftDevice is initialized and the tag has connecteced to the UWB network, the BLE connection can be advertized.
The avertisement function of the nRF52 SDK was used for this.


The BLE module listens for queries sent from the Phone to the tag using BLE.
To achiev this, a query-handler function was passed to the SoftDevice during initiation.
All incomming messages will be passed to this function by the SoftDevice.
When a query is received, the BLE module interprets the message.
It checks what is beeing queried and transformes it into a job, readable by the Job Handler module.
The BLE module also offers a service to send messages to the phone.
This service uses the nRF52 SDK to load the message into a the SoftDevice and send it to the phone.


\subsection{Job Handler}
\label{ss:job_handler_module}

The job-handler module connects all other module.
It takes job structs (see figure \ref{code:job_struct}, interprets which module is responsible for handeling them and calles the job together with the relevant data.
The job struct consits of a field for the job-type, that tells the job-handler what type of job this is. It also includes fields to store data, that is needed for the job.

\begin{figure}[h]
    \centering
    \begin{lstlisting}[language=c]
    struct job {
  		enum job_types type;
  		uint8_t* data;
  		int length;
};
    \end{lstlisting}
    \caption{Job struct}
	\label{code:job_struct}
\end{figure}

There are 14 total job types.The following list while decribe the meaning of them, as well as how they are handled by the job-handler:
\begin{itemize}
  \item \textbf{search for network}: This job is triggered after setup. The tag is not connected to the network. It will be passed to the Network module without any aditinal data.
  \item \textbf{join network request}: This job commes from the Network module, when it receives a request from another tag to join the network. It will be passed back to the Network module, with the data of the new devices id.
  \item \textbf{set network and address}: This job commes from the Network module. It informs the network connection has been established. The job is handed back to the Network module, with the received message, to be added to the list of neighbours.
  \item \textbf{ble temp hum request}: This job commes from the BLE module, where a query for temperature and humidity has been registered. The requested tag is extracted from the job. If the request is for this tag, the job is handed to the Temperature and Humidity module. Otherwise it is passed to the network module, to be transmitted to the requested tag.
  \item \textbf{temp hum request }: This job commes from the Network module and informs that a request for a temperature and humidity read has been made. It is passed to the Temperature and Humidity module, together with the requesting tags adress.
  \item \textbf{temp hum answer}: This job commes from the Network module and carries the respons to a temperature and humidity request. It is passed to the BLE module, togehter with the measurement, which will be passed to the phone.
  \item \textbf{ble gyro request}: This follows the same logic as "ble temp hum request", but with the gyroscope module.
  \item \textbf{gyro request}:This follows the same logic as "temp hum request", but with the gyroscope module.
  \item \textbf{gyro answer}:This follows the same logic as "temp hum answer"
  \item \textbf{ble distance request}: This job comes from the BLE module. The phone has queried for a distance. If the queried tag is not this tag, the message is passed to the Network module. Otherwise, it is passed to the Two-Way Ranging module.
  \item \textbf{distances reques}: This job comes from the Network Module. It requests a distance measurement. The job is passed to the Two-Way Ranging module, together with the requeting tags adress.
  \item \textbf{distances prepare}: This job comes from the Network Module. It informs, that another tag is requesting a ranging session. If the ranging session is with this tag, the job is passed to the Two-Way Ranging module. Otherwise the tag goes to sleep for a short time.
  \item \textbf{distances answer}: This job comes from the Network module. It reports that a distance measurement as been returned. The job is handed ober to the BLE module, together with the content of the message.
  \item \textbf{ble get known devices}: This job comes from the BLE module. It requests a list of all neighbours. The job is transfered to the Network-Module.
\end{itemize}


\subsection{Combining modules}
\label{ss:combination}
Each module except for the job-handler module was developed in seperate projects, to ensure operability.
Afterwards the modules were merged into one project.
The Network module was chosen as the base project, that the other projects were merged into.
This was chosen since the Network module was based on \cite{degkwitz2023ultrawideband}, which intern was based on a example published by Qorvo.
The Qorvo example uses a lot of shorthand, magic numbers and development shortcut, that are not easely readable to developers outside the firm.
The Network module whas therefore chosen as a basis, since merging it into another project would likely be cumbersome, since parts would easely be forgotten or interact poorly, without the knowledge or udnerstanding of the developer.
Combining the modules came with several chalanges, that described in this section.


The Qorvo example that builds the basis of the Network module uses the pin-mapping PCA10056.
This is the pin mapping for boards that include the NRF52840 board, but not the NRF52840 development board, that this example was made for and is used in this thesis.
The NRF52840 board does not contain the nesecary pins to attack a DWM3000 board to it.
This wrong pin-mapping leads to mistakes that the Qorvo example has to work around.\\
When switching to the correct pin-mapping, PCA10040, the Network mdoule would no longer work, since those work-arounds now introduced mistakes now.
Sice fixing the Qorvo example code would have been cumbersome, it was decided to instead change the other modules that used pins, the Gyroscope module and the Themperature and Humidity module.
The pins for those modules, pin 11, 13, and 13. where hard coded into the modules, instead of using the pin-mapping.


The nRF52 SDK offers a rich selection of tools, such as SPI and TWI communication, clocks, ble capabilities, SoftDevice, UUIDs.
These tools are all enabled or disbaled in the sdk{\_}config file.
Merging in general requires only to enable the tools needed by the merged module.\\
Three mdoules requie a nRF clock,, Two-Way Ranging, Temperature and BLE.
The nRF SDK offers  exactly three clcoks slots, so all of them have to be enabled with the apropriate clock type.
Each module has to be adapted, so it uses its assigned clock-slot. \\
The nRF52 SDK can suport up to three SPI or TWI conenctions simultaniosly, nemaed SPI0, SPI1, SPI2, TWI1,TWI2 and TWI3.
SPI and TWI share their memory, so SPI0 can not be used while TWI0 is used and vise-versa.
Since the DWM3000 uses two SPI connections and the MPU6050 uses one TWI connection, exactly enough resources remain, for both devices to run simultanisously.
SPI0 and SPI1 were used for the DWM3000 and TWI3 for the MPU6050.\\
All other SDK resources were non-conflicting.
They were ported from the original module implementation to the merged one without change.


As most embeded systems do, the nRF52840 requires static memory allocation during flashing.
The avaliable memory is seperated into flash-memory and random-access-memory (RAM).
Some memory segments are required by every runable system:
\begin{itemize}
	\item FLASH, \textbf{vectors}: The interrupt vector table defines the interrput handlers for the system, like resets, faults.
	\item FLASH, \textbf{init}: The initialization routine sets up clocks, pins and other peripherals.
	\item FLASH, \textbf{text}: This section contains the executable code in mashine language.
	\item FLASH, \textbf{data}: This section contains the initial values for all global values..
	\item \textbf{rodata}: This section contains the constant variables, that will not change at runtime.
	\item RAM, \textbf{data}: During startup, the initial values for changable global variables are coppied to this section. They can change at runtime.
	\item RAM, \textbf{bss}: This section contains the global variables that do not have initial values.
	\item RAM, \textbf{stack} and \textbf{heap}: The stack and heap that build the runntime environment.
\end{itemize}
Neither the MPU6050 nor the DHT22 require any additional memory segments.
The DWM3000 and the BLE module both require additional memory segments. \\
The BLE module reuqires the SoftDevice to be added to memory. The Softdevice requires 156 KB of Flash and 10.7 KB of RAM.Those reserved memory segments need to be the first one in both Flash and RAM. This additionaly requires SoftDevice oberservers for System on Chip (SoC), BLE, state and stack. Additionaly a segment to house the nRF52 SDK memory allocator is required, nrf{\_}balloc. These segments are rather small, never exceeding 32 bytes.\\
The DWM3000 shield requires two additional memory segments, fConfig in Flash and nrf{\_}balloc in RAM. 
Qorvo does not publish what the fConfig module is for, but it is required for the shield to work. \\
Since the base project was made for the DWM3000 shield, it had to be addapted to addinionaly fit the segments needed for the BLE module. This mainly consisted of moving all segments to later adress-spaces to add room for the SoftDevice reserved memory. All other memory segments had to be added as well. To make room for this, the Flash memory had to be expanded. \\
The Qorvo example implementation for the DWM3000 shield uses some work-arrounds. An example of this is the "NRFX{\_}SPIM3{\_}NRF52840{\_}ANOMALY{\_}198{\_}WORKAROUND{\_}ENABLED" present in the SDF configuration. These workarounds let the SPI communication with the shield perform certen memory manipulation. If these workarounds are necessary is doubtfull, but fixing them would have been out of scope for this thesis.
The workarounds do generaly have no effect on the implementation, with one excpetion. When the DWM3000 receiver sends a message longer than 10 bytes to the microcontroler over SPI, it incroaches on the SoftDevice RAM. This behaviour was found experimentaly, the responsible code could not be located. Since the system can be implemented with the restriction of 10 byte messages, this was done.



\section{App}
\label{s:app}
Nordic Semi Conductors, the maker of the used microcontrolers, published the code to a simple app that allows for BLE communication with their devices.
It is called nRF Toolbox.
It is intended to pair with the \textit{ble{\_}app{\_}uart} example, published in the nRF52 SDK.
Since this example code was used as the basis for the ble communication used in this project, it was addapted to work with this project.


The App contains different modules, intended for different examples, among them the Universal Asynchronus Receiver/Transmitter (UART) module (see \ref{f:Toolbox_modules}).
It is intended to be used with the ble\_ app\_ uart example.
When opem it shows the ble services that are currently beeing addvertised and allows the user to connect to one of them \ref{f:Toolbox_connect}.
It then opens a window similar to phone messangers, were the keyboard can be used to tpye messages, that are sent to the connected devices.

\begin{figure}[ht!]
\centering
\includegraphics[width=200px]{graphics/nRF_toolbox_modules.jpg}
\caption{nRF Toolbox module menue, with the added Art Tracking Module}
\label{f:Toolbox_modules}
\end{figure}

\begin{figure}[ht!]
\centering
\includegraphics[width=200px]{graphics/nRF_toolbox_connect.jpg}
 \caption{nRF Toolbox shows avaliable devices to connect to}
\label{f:Toolbox_connect}
\end{figure}

\begin{figure}[ht!]
\centering
\includegraphics[width=200px]{graphics/nRF_toolbox_messanger.jpg}
\caption{nRF Toolbox UART module screen}
\label{f:Toolbox_output}
\end{figure}

Since the development of an application was not the primary focus of this thesis, it was decided to take the nRF Toolbox app and add a new module for art-traking to it.
The UART module searved as the basis for this new module, since it had a lot of usefull services already implemented.
As with the UART module the art-tracking module opens up the same connection page \ref{f:Toolbox_connect}, that allows the user to select the art-tracking and connect to it.

Once connected, the observation screen is shown (figure \ref{f:Toolbox_art_tracking_empty}).
At the bottom seven parameters can be set: \textit{time}, \textit{max Temp}, \textit{min Temp}, \textit{max Hum}, \textit{min Hum}, \textit{max Angle}, \textit{max Dist}.
The parameters \textit{max}/\textit{min} \textit{Temp}/\textit{Hum} represent the expected range of humidity and temperature.
Any measurement outside these parameter will be considered a dangerous value by the app.
The tollerated difference in angle compared to the previous measurement is set by \textit{max Angle}, larger differences are considered dangerous values.
Distance measurement work analogously with \textit{max Dist} in meters.
The \textit{time} set defines the time that passes enbetween measurements in seconds.
The default is set to 350 seconds.
This means that the time that passes between, for example, the temperature measurements on tag 2 are 350 seconds.


\begin{figure}[ht!]
\centering
\includegraphics[width=200px]{graphics/nRF_toolbox_art_tracking_empty.jpg}
\caption{Art Tracking module oberservation screen before measurements}
\label{f:Toolbox_art_tracking_empty}
\end{figure}


When the user presses the \textit{Start Service} button, a services starts that poeriodically queries the tags for the Measurements.
Figure \ref{code:App_main_loop} shows the measurement loop.
Each sensor is assigned a character.
\textit{T} for temperature and humidity, \textit{G} for gyro and \textit{D} for distance.
Each tag has a number, here from one to four since four tags were used in the experiments.
The loop concatenates these two characters and sends the resulting query to the connected tag.
Then the next tag-number is prepared for the next query.
Once all tags have been queried for a sensor, the tag-number starts with the first again and the next sensor is queried.
In between cals the app waits.
The call time for distance-measurement is fixed at 80 seconds.
Distance measurement takes longer than the other sensors, since for every devices three measurements need tobe conducted.
Additionaly the sensors that do not participate in a ranging session are sleeping for a quite generous amount of time, to ensure they don't distrub the ranging session.
80 seconds has been chosen, since it allows enough time for all the ranging to happen, plus two repeats per sensor in case the ranging session fails.
For the other sensors the waiting time in between queries is calculated from the remaining set time, after the ranging time is deducted.

\begin{figure}[h]
    \centering
    \begin{lstlisting}[language=Java]
    private val sensors = listOf("T", "G", "D")
    private val devices = listOf("1", "2", "3", "4")
    private var measurement_type = 0
    private var tag = 0
    private var timeBetweenCals: Long = 3750

    private val runnable = object : Runnable {
        override fun run() {
            if (tag >= list2.size) {
                tag = 0
                measurement_type += 1
            }
            if (measurement_type >= list1.size) {
                measurement_type = 0
            }
            val textToSend = "${list1[measurement_type]}${list2[tag]}"
            artRepository.sendText(textToSend, MacroEol.LF)
            tag += 1
            if(list1[measurement_type] == "D"){
                handler.postDelayed(this, 80000)
            } else {
                handler.postDelayed(this, timeBetweenCals)
            }
        }
    }
    \end{lstlisting}
    % Optionally, add a caption to the figure
    \caption{Section from the ArtMetricService.kt, main measurement loop}
	\label{code:App_main_loop}
\end{figure}

Once the process has started, the queries will appear in the chat window on the right side of the screen.
The responses are on the right side, see figure \ref{code:App_main_loop}.
If the response is inside the set parameters, the message bubble will appear blue.
If the measured value is considered a dangerous value, the text bubble will appear red (see figure \ref{f:Toolbox_art_filled}).
Since the message display is programmed in a asynchronus way, it can happen, that the answer to a query appears before the querry itself, if the querried tag is the same as the connected tag.
The service can be stopped by pressing the \textit{start service} button again or by exiting this screen in any way.


\begin{figure}[ht!]
\centering
\includegraphics[width=200px]{graphics/nRF_toolbox_Bad_Value_2.jpg}
\caption{Art Tracking module, queries and responses}
\label{f:Toolbox_art_filled}
\end{figure}


%TODO: add image that shows an example file
The query-answers are appended to a file that is safed in the app-storage.
The information appendded consits of: the queried tag, the returned values, a timestamp and if the value was unproblematic.
This functionality is intended for experimental evaluation. 
In a real word application, this data should be periodically backed up on a server in a compressed manner.
When pressing the share-button on the top right of the message-box \ref{f:Toolbox_art_filled}.
It will open the Android naitive share functionality, to share the file over mail, an installed messanger, save it to onedrive or send it over Bluetooth.
In this project all files were sent with email.
Pressing the trashcan next to it will delete the chat and empty the file.
This allows the user to distinguish between different testing session.
%\chapter{Evaluation}
\label{chap:evaluation}



\section{Experiments}
\label{s:Experiments}
Five experiments were performed to validate the functionality of the tags.
The first two are non specific and ment to test the setup in a stable environment.
Experiments three to five are intended to test the detection of unwanted circumstances.
For all experiments the query-frequency was set to 330s, so measurment was evaluated once every 330s.
The measurements queries are spread across this timeframe.
Each experiment lasted between 40 minutes and one hour.
The results were stored on the phone and then exported using email.
The analysis of the data and creation of graphs was then performed using a Jupiter Notebook, using Pandas and Pyplot for datamanagment and the creation of graphs.


\subsection{Experiment 1: Static}
\label{ss:exp_1}
The four tags where placed on the corners of a 80 cm by 50 cm rectangle on a wooden table.
Each tag was turned on sequentially and given enough time to establish the network.
The phone then was connected to one tag.
The parameters in the app were left unchanged.
The default parameters are large enough, that no measurement should be large enough to trigger a warning.
The setup was then left untouched for 35 min.
The goal of this experiment was, to gauge by how much the measurements can vary in a static environment.
 

\subsection{Experiment 2: Coordinated movement}
\label{ss:exp_2}

\subsection{Experiment 3: Temperature}
\label{ss:exp_3}
The four tags were placed in the same 80 cm by 50 cm rectangle as in experiment one.
One tag placed on a elevated surface, 4 cm above the table.
Under the tag seven candles were placed (see figure TODO, Bild einfügen).
Next to the tag two thermometers detectors were placed.
Each tag was turned on sequentially and given enough time to establish the network.
The phone then was connected to one tag.
The max Temperature parameter in the app was changed to 35°C.
After 20 minutes the candles were lit.
The experiment was then left alone for another 30 minutes.
The independent thermometers were filmed during the process, to allow for later review and comparesment.
The goal of experiment 3 was to test the temperature detection capabilities of the system.


\subsection{Experiment 4: Gyroscope}
\label{ss:exp_4_1}
Again all for tags were placed on a 80 cm by 50 cm rectangle.
Each tag was turned on sequentially and given enough time to establish the network.
The phone then was connected to one tag.
The maximal allowed angular difference was set to 30°.
After 20 minutes one tag was turned by 90° counterclockwise.
The experiment then ran for another 30 minutes.
The goal of experiment number 4 was to test the detection of unwanted rotations.
Experiment four was repeated with, with the gyro sending the angular velocity for all axes instead of the current position.


\subsection{Experiment 5: Distance}
\label{ss:exp_5}
The same 80 cm by 50 cm rectangle setup was used.
The tags were turned on sequentilay, giving them enough time to build the network.
The phone was connected to one tag.
The max distance parameter was set to 20 centimeter.
After 20 minutes, one tag was moved parallel to the shorter rectangle line about 20 cm towards the tag on the next corner.
The system was then left resting for another 30 minutes.
The goal of experiment number 5 was to test the detection of unwanted movement.

\section{Experiment Results}
\label{s:exp_res}

In this section the results of the experiments are presented.
All eperiments were performed two to three times.
In each section only the data-set from the first experiment run is presented fully.
The other experiments will be mentioned only, if they have differing data or to confrim an unexpected datapoint.

\subsection{Experiment 1: Static}
\label{ss:exp_1_result}
In eperiment one, all measurements are expected to be unchanging.
Table \ref{t:exp1_means} shows the mean values for temperature, humidity and angle during the experiment by tag.
Figures \ref{f:exp1_graphs_temp}, \ref{f:exp1_graphs_hum}, \ref{f:exp1_graphs_gyro}, \ref{f:exp1_graphs_dist} shows the change of these values over time.

\begin{table}[h!]
    \centering
    \begin{tabular}{|c|c|c|}
        \hline
        \textbf{Tag} & \textbf{Temp Mean} & \textbf{Hum Mean} \\
        \hline
        1 & 22.06 & 32.56 \\
        2 & 21.90 & 33.93 \\
        3 & 22.06 & 32.94 \\
        4 & 21.87 & 32.80 \\
        \hline
    \end{tabular}
    \caption{Mean and Variances for Temperature, Humidity, and Gyroscope Data by Tag during experiment 1}
    \label{t:exp1_means}
\end{table}

\begin{figure}[ht!]
\includegraphics[width=\linewidth]{graphics/exp/exp1_temp_plot_0.png}
 \caption{Experiment 1, temperature over time.}
\label{f:exp1_graphs_temp}
\end{figure}


All four tags have a similar mean temperature and are all less than 0.2 °C apart from each other.
The varaince are also small, tag two having the highest one with 0.05 °C variance. 
The graph shows that all tags have a rising temperature.
The increase is quite small with tag two having the biggest increase of 0.5 °C over 20 minutes.
When the experiment was repeated, the means stayed similar and the variance small, but the temperature changed course.
A downward trend was visible, instead of the upward trend seen during the first experiment.

\begin{figure}[ht!]
\includegraphics[width=\linewidth]{graphics/exp/exp1_hum_plot_0.png}
 \caption{Experiment 1, humidty over time.}
\label{f:exp1_graphs_hum}
\end{figure}

Humidity follows a similar trajectory as.
The means only vary by 1.5 \% pt.
The variance is small, with tag three having the biggest variance with 0.06\% pt.
During the first experiment, humidity increased by a small amount.
When the experiment was repeated, the humidity dropped during the experiment.


\begin{figure}[ht!]
\includegraphics[width=\linewidth]{graphics/exp/exp1_gyro_data_plot_0.png}
 \caption{Experiment 1, angles over time.}
\label{f:exp1_graphs_gyro}
\end{figure}

Since all tags were stationary during the experiment, the gyro sensor was expected to be unchanging.
This is not what happened.
looking at the graph \ref{f:exp1_graphs_gyro} it is clear, that the measurement shows a wide range of angles for each tag and axis.
The only exception is tag 2 around the x axis, which stays at 0 for the whole measurement duration.
Since angle measurements fall into modular arithmetic, it "wrappes around" at 360°, means can only meanigfully be taken if the angles are in a small range.
Since this is not the case for most tags, the only thing that can be said is, that tag 2 has a mean of 0 with variance 0 around axis x.


\begin{figure}[ht!]
\includegraphics[width=\linewidth]{graphics/exp/exp1_dist_data_plot_0.png}
 \caption{Experiment 1, distance over time.}
\label{f:exp1_graphs_dist}
\end{figure}

Table \ref{t:exp1_means}, shows the mean of the measured distances, the row entry beeing the queried tag that initiates the distance measurement, and the row corresponding to the responding tag.
By looking to the measurements diagonaly oposed to each other, one can see that the measured distaances is the the same, indipendent of who initiated the measurement, up to a range of two centimeters.
The varince on table \ref{t:exp1_dist_var} also show, that these measurements are stable over time and don't change by much.
Looking at the graphs on figure \ref{f:exp1_graphs_dist}, the pairs of measurements are visible.
One outlier happens when tag 3 measures the distance to tag 1 at very end of the measurements.
When repearing the measurements these outliers happened again, a bit less frequently then twice per hour.
The outliers always affected a measurement involving tag 1.
The distances measured do not correspond to the actual distances the tags had to each other, also seen in table \ref{t:exp1_dist_means}.
The measured distances can be as far of as 0.5 meters.
The two larger distances, 0.8 and 0.94 meters, correspond to the two larger measured values for each tag, while the smallest measured value always corresponds to the smallest distance, 0.5 meters.
The two larger values are not always ordered correctly, 0.94 meters sometimes beeing measured smaller then 0.8 meters.
In repeated experiments, all these facts stayed true.

\begin{table}[h!]
    \centering
    \begin{tabular}{|c|c c c c|}
        \hline
          & \textbf{1} & \textbf{2} & \textbf{3} & \textbf{4} \\
        \hline
		\textbf{1} & 0.0 & 1.094 & 1.084 & 0.657 \\
		\textbf{2} & 1.080 & 0.0 & 0.356 & 0.989 \\
		\textbf{3} & 1.007 & 0.367 & 0.0 & 1.279 \\
		\textbf{4} & 0.666 & 0.987 & 1.281 & 0.0 \\
        \hline
    \end{tabular}
\begin{tabular}{|c|c c c c|}
        \hline
          & \textbf{1} & \textbf{2} & \textbf{3} & \textbf{4} \\
        \hline
		\textbf{1} & 0.0 & 0.8 & 0.94 & 0.5 \\
		\textbf{2} & 0.8 & 0.0 & 0.5 & 0.94 \\
		\textbf{3} & 0.94 & 0.5 & 0.0 & 0.8 \\
		\textbf{4} & 0.5 & 0.94 & 0.8 & 0.0 \\
        \hline
\end{tabular}
    \caption{Left: Mean distances between tags in experiment 1. Right: Expected values.}
    \label{t:exp1_dist_means}
\end{table}

\begin{table}[h!]
    \centering
    \begin{tabular}{|c|c c c c|}
        \hline
          & \textbf{1} & \textbf{2} & \textbf{3} & \textbf{4} \\
        \hline
		\textbf{1} & 0.0 & 0.002 & 0.000 & 0.000 \\
		\textbf{2} & 0.001 & 0.0 & 0.000 & 0.001 \\
		\textbf{3} & 0.024 & 0.001 & 0.0 & 0.000 \\
		\textbf{4} & 0.000 & 0.000 & 0.000 & 0.0 \\
        \hline
    \end{tabular}
    \caption{Variance of distances between tags in experiment 1. Row corresponds to queried tag.}
    \label{t:exp1_dist_var}
\end{table}

\subsection{Experiment 3: Temperature}
\label{ss:exp_3_result}
Experiment 3 introduced heat-sources the system.
Since the main setup was the same as experiment 1 \ref{ss:exp_1_result}, many of the findings are the same.
In this section, only differences in results are discussed.
If a metric is not measioned, one can assume it behaved the same as for experiment 1  (see section \ref{ss:exp_1_result}).

\begin{figure}[ht!]
\includegraphics[width=\linewidth]{graphics/exp/exp3_temp_plot_1.png}
 \caption{Experiment 3, temperature over time, mith external measurement added.}
\label{f:exp3_graphs_temp}
\end{figure}

The progression of the external thermoeter and the internal temperature sensor can be seen in figure \ref{f:exp3_graphs}.
The candles, that functioned as the heat source, were lit at 15.10.
During the next measurement of tag 2, at 15.12, both the external thermoeter and the temperature sensor on tag 2 had not yet registered any change, remaing at 22.4 °C.
The extrenal thermometer started rising 1 minutes later, at 15.13.
During the next measurement at 15.18, the temperature-sensor registered a slightly increased temperature of 23.6 °C, while the external thermometer registered 24.7 °C.
During the next measurement at 17.24 the tag reported 26.3 °C while the thermometer showed 27.1 °C. 
The measured temperature of the external thermometer keeps klimbing faster than the internal temperature sensor, until the end of the experiment, as seen in Figure \ref{f:exp3_graphs_temp}.
Their distance nether exeeds 1 °C and gets smaller towards the end of the experiment.
The other tags do not report any significant change in temperature.

\begin{figure}[ht!]
\includegraphics[width=\linewidth]{graphics/exp/exp3_hum_plot_1.png}
 \caption{Experiment 3, humidity over time, with external measurement added.}
\label{f:exp3_graphs_hum}
\end{figure}


Experiment 3 was intended to test the temperature and not the humidity.
Luckily, the external thermoeter also included a humidity sensor, that could retroactivly be used for evaluation.
Figure \ref{f:exp3_graphs_hum} shows the humidity over time, with a added humidity sensor added to the graph.
Since the external humidity sensor was initialy not intended to be used, it is not perticulalry precise and does not display any digits after the decimal point.
The humidity sensor consitently shows a much higher humidity than the one on the tag.
Once the experiment starts at 15.10, the humidity behaves inversly to the temperature and starts falling.
This happens with the external sensor as well as the internal one in parallel.
The registered values plato at 34\% for the external and 26\% for the internal sensor.
The other tags do not report any significant change in humidity.

\subsection{Experiment 4: Gyroscope}
\label{ss:exp_4_result}

Experiment 4 was intended to check the functionality of the gyroscope.
Temperature and humidity behaviour was the same as in the static experiment \ref{ss:exp_1_result}.
As already seen in during the evaluation of experiment 1, the gyroscope does not work as planned.
Figure \ref{f:exp4_graphs_gyro} shows the values of the gyro over time.
Tag 1 was rotated by 90° at 22.25 around the Z axis.
Their is no disernable change in the output of the gyro during or after this process.

\begin{figure}[ht!]
\includegraphics[width=\linewidth]{graphics/exp/exp4_gyro_data_plot_1.png}
 \caption{Experiment 4, gyroscope over time.}
\label{f:exp4_graphs_gyro}
\end{figure}


Figure \ref{f:exp4_graphs_dist} shows the distances of the tags during the experiment.
Before the event, all tags are in a stable state.
As in experiment 1 \ref{ss:exp_1_result} the distances do not represent what is physically happening.
After the tag is turned at 22.26, all measurements involving tag 1 change, and becomming stable again afterwards.
This can be bit hard to see, since "2-1" and "3-1" have an outlier measurement right before and "1-2" right after.
Distance 1 to 2 and 1 to 3 changes between 0.2 and 0.3 meters and distance 1 to 4 changes by arround 0.4 dm.

\begin{figure}[ht!]
\includegraphics[width=\linewidth]{graphics/exp/exp4_dist_data_plot_1.png}
 \caption{Experiment 4, gyroscope over time.}
\label{f:exp4_graphs_dist}
\end{figure}


\subsection{Experiment 5: Distance}
\label{ss:exp_3_result}

Experiment 5 was intended to test the distance measurement capabilities of the setup.
Temperature and humidity and gyro behave as they do in experiment 1 \ref{ss:exp_1_result}.
They will not be discussed for this experiment.

\begin{figure}[ht!]
\includegraphics[width=\linewidth]{graphics/exp/exp5_dist_data_plot_2.png}
 \caption{Experiment 5, distance over time.}
\label{f:exp5_graphs_dist}
\end{figure}

Figure \ref{f:exp5_graphs_dist} shows the measured distances of the 4 tags over time.
As in the static experiment, the measured distances of two devices are similar and mostly stable, before any movement is introduced.
As in experiment 1 the values reported are not correct.
At 14.24 tag 1 is moved by 0.23 meters toward tag 2.
The measured distances to tag 3 increases while the distance to tags 2 and 4 dicreases.
This represent what is happening in reality, since tag 1 is now closer to tag 2 and 4 and further away from tag 3 as before.
The difference in distance is roughly 0.2 meters for tag 2.
This is correct, since tag 1 was moved about that distance towards tag 2.
The measurements show tag 4 now 0.15 meters closer to tag 1.
The effect on tag 4 should be notissable but not as large as it is.
Since the tag moves lateraly towards tag 4, the difference should only be 0.11 meters.
The same is true for tag 3.
The difference in measured distance between 1 and 3 is between 0.15 and 0.2 meters. 
This is too large for the difference a latteral move, it should only be a 0.02 meters difference.
Their is also a small increase in the distance between tags 2 and 3, but which starts before tag 1 was moved.




%\chapter{Final Considerations}
\label{chap:considerations}

%Regarding Final Considerations:
% I. Some teachers and/or methods use the term Conclusion for a text at the end of the paper that aims to expose the results achieved, this term is not incorrect, but many of the works are bibliographical reviews where in the end no conclusion is obtained and yes Several considerations that were found in the development of written work. Therefore, in each project should be considered/weighted if there will be a Conclusion or Final Considerations. Usually what else happens is that you have Final Considerations. The Final Considerations of a paper aims to show if the goal sought for the project was achieved, as well as give a view of the most important considerations and conclusions on the subject addressed, among other aspects. This should include:
%1. An explanation stating clearly whether or not it has achieved the stated objectives (a subdivision between general objectives and specific objectives can also be made here). In each case the reasons must be explained:
%The. If you have achieved the objectives: inform the main factors that contributed to the success, describing them briefly, but do not leave doubts;
%B. If you have not met the objectives: inform how much of the objective has been achieved and cite the factors that contributed to the failure, describing them briefly, but that leaves no doubt.
%2. Describe the main considerations and conclusions that were obtained as a result of the execution of the work. Here should not be repeated text already in the work, but write the impressions of these considerations and how they contributed to the implementation and achieved the goal;
%3. Name and describe the main difficulties encountered in the execution of the work and project. All the work developed means an evolution for the student, and to reach this evolution, it has had to overcome a series of obstacles. Reporting obstacles and overcoming (or not overcoming) helps to dignify and show the merit of the work itself to the reader/evaluator. It is also a contribution, in the sense that once problems and solutions are exposed, readers/evaluators learn/know ways of solving or approaching such problems;
%4. Discuss whether modifications occurred during the execution of the work within the scope defined in the Project phase and in what was developed. It should be explained what generated those modifications, substantiating and justifying such changes.
%5. The relationship between the proposed schedule and the work schedule can be described. This allows the reader/evaluator to learn from the indicated distortions/hits.
%6. Describe or cite future work that may be done based on this work. During the execution of work, it is sought to reach a defined objective in the project. However, several interesting subjects of research are revealed (being that the same ones are not treated/researched in the work because they do not match the objective and scope of the work). The description of such subjects/themes/research demonstrates the students' perception of development as well as their vision of objectivity in the execution of this work.

%II. Conclusion / Final Considerations aim to show the reader/evaluator the student's perception of the work done. In this way, it is not advisable to do citations and references because theoretically everything that was necessary to quote and refer should already be done within the content of the work. Only in some very specific cases/situations can you make referrals or citations in this part of the work (this should be discussed thoroughly with the supervisor/teacher).

%III. Looking at the described items that should compose the Conclusion / Final Considerations, it is difficult to imagine that this part of the work has less than one page;

\section{Summary}

% I did this in this way, that in that way... and so on

\section{Conclusions}

% Lessons learned

\section{Future Work}

%\chapter{Fundamentals}

In this chapter, the fundamental knowledge that was researched during this thesis is presented.
Section \ref{s:background} introduces the background knowledge required for this thesis and Section.
Section \ref{s:related_works} presents the current state of research on topics surrounding this thesis.

\section{Background}
\label{s:background}

This Section describes the theoretical background used in this thesis.
It covers key aspects, such as communication protocols, two-way ranging, sensor networks, and required graph theory.


\subsection{Wirerless Sensor Networks (WSN)}
\label{S:WSN}
Kevin Ashton coined the term Internet of Things (IoT) in 1999, although the idea predates this term and was before known as embedded internet or pervasive computing \cite{alaba2024evolution}.
It describes the ubiquity of digital devices and their seamless integration into the physical world and everyday life.



At the end of the 90s and during the 2000s, the production of embedded systems and sensors, in particular, rose.
This led to falling prices and sensors becoming widespread.
With the new availability of sensors, Wireless Sensor Networks (WSN) became more widespread.
While not originating in during this time, the term itself was coined in 1980, and the research community became more focused on the topic \cite{jindal2018history}.

\cite{jindal2018history} determined four main challenges in the development of WSNs:
\begin{itemize}
\item Self-organization: A large number of nodes should not require manual installation and maintenance
\item Cooperative Processing: Sensor nodes have limited memory. Evaluating, compressing, and transporting the data becomes a major challenge.
\item Energy efficiency: WSNs often operate where no power supply is available. The sensors, therefore, must run on limited, battery-based energy.
\item Modularity: WSNs should work for various applications and sensor node types.
\end{itemize}

The Ad hoc On-Demand Distance Vector Routing (AODV) Protocol was published in 1999 by E. Perkins and E. M. Royer \cite{royer1999multicast}.
It presents a routing algorithm for wireless ad hoc networks, where routing is only established when needed, and devices can be added to or leave the network at will.
Doing this takes the problem of self-organization and modularity into account.
A modified version of AODV is used in Zigbee to this day \cite{mu2017improved}. 

In 2000, Heinzelman et al. published the low Energy Adaptive Clustering Hierarchy (LEACH) algorithm \cite{heinzelman2000energy}.
Leach divides the sensors into clusters based on location.
The clusters communicate with a network head using cluster heads that collect and transmit the cluster's data.
New cluster heads are elected periodically to spread the increased energy drain from the data transfer to the network head.
LEACH is used \cite{sefati2022cluster, ghazy2023low} and improved \cite{bagherzadeh2022survey} to this day.

\subsection{Ultra Wideband}

\subsubsection{IEEE}

The Ultra Wideband (UWB) communication protocol was introduced in 2003 by the Institute of Electrical and Electronics Engineers (IEEE) as part of the IEEE 802.15.4 standard.
In 2020, updates were made to the protocol when the IEEE 802.15.4z-2020 standard made improvements to the PHY layers of UWB connections. 
It achieved this by introducing a more robust timestamping system on the PHY layer.
This is supplemented by changes to the MAC layer that allow for the exchange of ranging information.
The result is short frames that are transmitted fast between devices, leading to short bursts of communications that are fast, secure, and ideal for ranging.


UWB works by using short radio frequency pulses, resulting in a large bandwidth.
UWB is a lower-power communication form.
This prevents it from interfering with other communication forms with which it shares its wavelength, such as WLAN or Bluetooth. 
Since UWB uses very short, distinct pulses over a short range, it has been found to be useful in ranging systems \cite{hsu_2021}. 
UWB is split into high-rate pulse (HRP) UWB and low-rate pulse (LRP) UWB.
Since ranging is part of this work and LRP is generally not used for ranging \cite{hsu_2021}, I will not discuss it further in this Thesis.
Since UWB devices tend to be small and have low energy consumption, in combination with the capability of ranging and data transfer, they have become popular as Internet of Things (IoT) devices. \\ 

The standard defines the PHY and MAC layer as well as frequency bands for communication.
The 4z expansion tries to integrate UWB into the WPAN standard. In Section \ref{sec:UWB MAC} and \ref{ss:UWB_PHY} I will discuss the PHY and MAC layer.\\

The sending devices emits pulses in a pre-set band of frequencies, using short bursts to transmit the bits.
The signal forms a concave curve in this band, where the two points 10 db below the maximum power spectral density are called the lower- and upper-frequency points, see Figure \ref{f:UWB_spectrum}.
These two points must be at least 500 Hz apart.
The maximum power spectral density must be below the noise level.
This process prevents conflicts with other communications, that use a single frequency with a high power spectral density and modulate signal transmission, such as WI-FI or Bluetooth.
The UWB protocol has the added benefit of being useful for high-accuracy localization.


\begin{figure}[ht!]
\centering
\includegraphics[width=\linewidth]{graphics/UWB_spectrum.jpg}
\caption{Power Spectral Density: Bandwidth B, lower-frequency f\textsubscript{L}, upper-frequency f\textsubscript{H}, \cite{hsu_2021}}
\label{f:UWB_spectrum}
\end{figure}


\subsubsection{UWB supported Nodes}

The IEEE 802.15.4 standard distinguishes between two types of devices.
Full-function devices (FFD) are capable of connecting to multiple other devices, receiving, transmitting, and coordinating. Reduced-function device (RFD), on the other hand, can only connect to one other device and act as a worker. 
In Topological terms, RFDs can only operate as leaves, while FFDs can be any node in a network, including leaves.
RFDs, therefore, are strictly worse but make up for it by requiring fewer resources, such as memory and power.
When FFDs work as PAN coordinators, they can use short addresses to address any node.
The PAN also has a PAN identifier to help communication across multiple networks while still using the short address.
Each device also has an extended address that is not assigned by any coordinator and serves as a universal unique identifier (UUID).


\subsection{UWB MAC}
\label{sec:UWB MAC}

The Mac Layer is part of the Data link layer.
The Mac Frame is the payload of the PHY frame. It carries information about the frame type, frame format, security mechanism, addressing, and frame validation.
The Mac Layer additionally provides rules for beacon management, and channel access.

\begin{figure}[ht!]
	\centering
	\includegraphics[width=\linewidth]{graphics/general_MAC_Frame_Format.png}
	\caption{General MAC Frame Format \cite{IEEE4-2020-7}}
	\label{f:MAC Frame Format}
\end{figure}


\subsubsection{MAC Frame Format}

Figure \ref{f:MAC Frame Format} shows the composition of a UWB-MAC frame.

In the MAC header (MHR), the Frame Control Field includes information about:

\begin{itemize}
  \item the frame-type
  \item if the Auxiliary Security Header Field is used and in what capacity
  \item if additional frames will follow
  \item if an acknowledgment message is expected
  \item if the message is between different PAN-Networks.
  \item of what type the receiver is (PAN coordinator, device, PAN-Network)
  \item the used frame-format standard
  \item where to find the source address
\end{itemize}

The Sequence Number counts up, helping to keep track of the order in which frames have arrived.
The Addressing Fields carry the IDs of the sender and recipient for the frame.
The Auxiliary Security Header Field only exists if specified in the control Field.
It contains additional information needed for the chosen security method.

There are two parts to the information element (IE).
The header IE specifies additional information about the frame, for example, data formatting information or channel time allocation.
The payload IE specifies the length and data type of the payload field.
The payload contains the data that is sent.
It and the IE are of variable length, depending on the frame type and data length.

The MAC footer (MFR) marcs the end of the frame.
It only contains the frame checking sequence (FCS) that can be used to detect corrupted frames using cyclic redundancy checks.

\subsection{UWB PHY}
\label{ss:UWB_PHY}

\subsubsection{PHY Chanel}

The IEEE 802.15.4z amendment defines 16 channels for communication for HRP UWB. 
A channel is defined by its center frequency.
UWB devices can transmit on three different bands: high band, low band, and sub-gigahertz.
For each band, there is one channel that is mandatory to support if a device supports the band.
The other channels are optional, but if two devices want to communicate with each other, they need to use the same band.
The bands, 16 channels, and their ranges, and which channels are mandatory can be found in the table  (see table \ref{Table:: UWB frequency and channel assignments}).


\begin{table}[ht!]
\centering
\begin{tabular}{|l|l|c|c|}
\hline
\textbf{Channel number} & \textbf{Center frequency (MHz)} & \textbf{HRP UWB band}       & \textbf{Mandatory}  \\ 
\hline
0                       & 499.2                           & sub-gigahertz               &  \checkmark     \\ 
\hline
1                       & 3494.4                          & \multirow{4}{*}{Low band}   &                   \\ 
\cline{1-2}\cline{4-4}
2                       & 3993.6                          &                             &                   \\ 
\cline{1-2}\cline{4-4}
3                       & 4492.8                          &                             & \checkmark      \\ 
\cline{1-2}\cline{4-4}
4                       & 3993.6                          &                             &                   \\ 
\hline
5                       & 6489.6                          & \multirow{11}{*}{High band} &                 \\ 
\cline{1-2}\cline{4-4}
6                       & 6988.8                          &                             &                   \\ 
\cline{1-2}\cline{4-4}
7                       & 6489.6                          &                             &                   \\ 
\cline{1-2}\cline{4-4}
8                       & 7488                            &                             &                   \\ 
\cline{1-2}\cline{4-4}
9                       & 7987.2                          &                             & \checkmark       \\ 
\cline{1-2}\cline{4-4}
10                      & 8486.4                          &                             &                   \\ 
\cline{1-2}\cline{4-4}
11                      & 7987.2                          &                             &                   \\ 
\cline{1-2}\cline{4-4}
12                      & 8985.6                          &                             &                   \\ 
\cline{1-2}\cline{4-4}
13                      & 9484.8                          &                             &                   \\ 
\cline{1-2}\cline{4-4}
14                      & 9984                            &                             &                   \\ 
\cline{1-2}\cline{4-4}
15                      & 9484.8                          &                             &                 \\
\hline
\end{tabular}
\caption{ HRP UWB Frequency and Channel Assignments  \cite{IEEE4-2020-7, IEEE4z}}
\label{Table:: UWB frequency and channel assignments}
\end{table}

\subsubsection{Scrambled timestamp sequence}

The 4z amendment added the option to include a scrambled timestamp sequence (STS) into the frame.
The STS is a cyphered sequence that includes the timestamp and is used for ranging.
It is meant to increase the accuracy and integrity of the raging results.
Before transmitting, the receiver and sender exchange a randomly generated key.
The key is then used to encrypt the timestamp using the advanced encryption standard (AES) with 128 bits.
This ensured that the signal had not been intercepted and changed to manipulate the ranging result.
Devices that support STS are called HRP-enhanced ranging capable
devices (HRP-ERDEV).

\subsubsection{Pulse Repetition Frequency}
\label{sec:pule repetition frequency}
The pulse repetition frequency (PRF) is the frequency at wich bursts are sent by the transmitter.
The mean PRF is the average PRF while sending the payload (power switching service data unit PSDU) \cite{hsu_2021}.
The higher the mean PRF, the shorter the airtime of each frame, and it allows for faster communication.
HRP-ERDEV uses a different mean PRF than general devices.
They can work in Base pulse repetition frequency (BPRF) operating at mean PRF 64 MHz or in higher pulse repetition frequency (HPRF) mode operating above BPRF (Table \ref{f:mean PRF}).

\begin{table}[ht!]
\centering
\begin{tabular}{|l|l|l|} 
\hline
\textbf{Standard}          & \textbf{HRP UWB mode} & \textbf{mean PRF}            \\ 
\hline
802.15.4                   & Non HRP ERDEV         & 3.9 MHz, 15.6 MHz, 62.4 MHz  \\ 
\hline
\multirow{2}{*}{802.15.4z} & HRP-ERDEV BPRF        & 62.4 MHz                     \\ 
\cline{2-3}
                           & HRP-ERDEV HPRF        & 124.8 MHz, 249.6 MHz         \\
\hline
\end{tabular}
\caption{HRP UWB Mean PRF (Based on IEEE 802.15.4 and IEEE 802.15.4z, \cite{IEEE4-2020-7, IEEE4z})}
\label{f:mean PRF}
\end{table} %cite euse eigeni bricht

\subsubsection{Symbol Encoding}
UWB sends symbols by transmitting a burst of pulses that encode the symbol.
Since the pulses have clean edges, the arrival time can be measured precisely.
This leads to the burst having two ways to carry information( \cite{QorvoGettingBacktoBasics}):
\begin{itemize}
  \item Binary phase-shift keying (BPSK: Encoding zeros and ones shifting the phases of the pulses so the burst beak for one has an opposite amplitude to the other. 
Figure \ref{f:UWB_signal_description} shows the single 101 binary phase-shift keyed. 
Each bit is set twice, to detect problems with transmission.
  \item Burst position modulation (BPM): Changing the timing of the burst so it falls into a different time slot inside of the possible burst position.	
Figure \ref{f:symbol structure} shows how the burst can be placed in a BPM interval. 
The burst cannot be placed in the guard interval. 
The guard exists to minimize inter-symbol interference from the
signals taking multiple paths.
\end{itemize}

\begin{figure}[ht!]
	\centering
	\includegraphics[width=\linewidth]{graphics/HRP_UWB_PHY_symbol_structure.jpg}
	\caption{HRP UWB PHY Symbol Structure \cite{hsu_2021}}
	\label{f:symbol structure}
\end{figure}

One or both of these encoding strategies can be used in a uwb transmission.
The position of the pulses inside of the burst (see figure \ref{f:UWB_signal_description})  relative to each other can be used to detect the presence of multipath effects and adjust for them. 
Using this, precise arrival times for the whole signal can be calculated.

\begin{figure}[ht!]
	\centering
	\includegraphics[width=\linewidth]{graphics/uwb_signal_tramsmission.png}
	\caption{UWB signal transmission byte encoding, \cite{QorvoGettingBacktoBasics}}
	\label{f:UWB_signal_description}
\end{figure}

Non-HRP ERDEV uses BPM and BPSK.
Some HRP-ERDEV can use only BPSK, which uses a higher PRF and, therefore, reduces airtime.








\subsubsection{PHY Frame}
Figure \ref{f:PPDU general} shows a schematic view of a PHY frame as defined by the IEEE 802.15.4 standard.
The Synchronization header (SHR) contains the information needed to detect the signal and adjust to its parameters.
The PHY header contains meta-information about the payload and its encoding.
The PHY payload contains the data to be sent, namly the MAC frame.

\begin{figure}[!ht]
\centering
\includegraphics[width=\linewidth]{graphics/Schematic_view_PPDU.png}
\caption{Schematic view of a PHY frame defined by IEEE 802.15.4 \cite{IEEE4-2020-7}}
\label{f:PPDU general}
\end{figure}


Figure \ref{f:SHR field} shows the synchronization header, consisting of two parts.
The SYNC section is detectable by the receiver and informs it that a transmission has started.
Depending on the predefined mode, pulses of different lengths are used.
The sequence of pulses specifies a set of channels that can be used for communication.
The preamble can also be used to identify a PAN coordinator.

The SHR ends with the Start of Frame Delimiter (SFD).
It indicates that the synchronization has ended, and the coming signals will be data, starting with the PHY header. 
It also contains a timestamp, which can be used for ranging using the time difference of arrival (ToA), see Section \ref{ss:two_way_ranging}


\begin{figure}[!ht]
\centering
\includegraphics[width=\linewidth]{graphics/SHR_field_structure.png}
\caption{SHR Field Structure \cite{IEEE4-2020-7}}
\label{f:SHR field}
\end{figure}	

The PHY header contains all the information needed to read the PHY payload (see Figure \ref{f:PHR general}).
The first bit defines the data rate used during the payload transfer (see section \ref{sec: pulse repetition frequency}).
The next seven bits define the length of the frame, with a frame length of a maximum of 128 bytes.
The 10th bit shows if ranging will be used with this frame.
The next bit is reserved.
Bits 11 and 12 define the preamble duration. It specifies how many repetitions are used, which can range from 16 to 4096.
The last 6 bits are single error correct, double error detect (SECDED) bits that form a Hammock block and can be used to correct single-bit errors and detect, but not fix double-bit errors.

The last part of the PHY frame is the PHY payload (PSDU).
This contains the MAC frame, as defined in section \ref{sec:UWB MAC}.

\begin{figure}[ht!]
\centering
\includegraphics[width=\linewidth]{graphics/PHR_field_format_4.png}
\caption{General PHR Field Format \cite{IEEE4-2020-7}}
\label{f:PHR general}
\end{figure}

The 802.15.4z amendment contains optional changes to the PHY frame format if the participating devices are HRP-ERDEV devices.
Figure \ref{f:HRP-erdev frame} shows the newly allowed structures for a UWB frame.
Configuration 1 is equivalent to the already existing PHY frame.
The others additionally contain a scrambled time stamp.
This can be placed in different places after the SHR.
Since UWB can also be used only for ranging without transmitting a message, configuration three only contains the SHR and STS, without a payload.

\begin{figure}[ht!]
\centering
\includegraphics[width=\linewidth]{graphics/HRP_ERDEV_frame_structures.jpg}
\caption{HRP-ERDEV Frame Structures \cite{hsu_2021}}
\label{f:HRP-erdev frame}
\end{figure}

Additionally, the PHR can be formatted differently (see figure \ref{f:PHR 4z}. 
The reserved field and preamble duration are removed to make more space for the frame length. 
This allows more data to be sent in one frame, increasing the throughput of the UWB communication.

\begin{figure}[ht!]
\centering
\includegraphics[width=\linewidth]{graphics/HRP_ERDEV_HPRF_mode_PHR.png}
\caption{PHR Field Format for HRP-ERDEV in HPRF Mode \cite{IEEE4z}}
\label{f:PHR 4z}
\end{figure}


\subsection{Two-way ranging}
\label{ss:two_way_ranging}
The IEEE 802.15.4z UWB standard describes two ranging methods: single-sided two-way ranging (SS-TWR) and Double-sided two-way ranging (DS-TWR).
In both instances, the distance measurement is done by calculating the time of flight (ToF) of a signal sent between two devices using timestamps and multiplying it by the speed of light. 
In this Section, both SS-TWR and DS-TWR will be discussed.
Two-way ranging(TWR) refers to DS-TWR in all other parts of the Thesis.

\textbf{Single-sided two-way ranging (SS-TWR)}:
During  SS-TWR, one device sends a message to a second and measures the round-trip time (see figure \ref{f:ss_twr}).
Device A sends a message to B and records a timestamp when the message was sent, $T_{A0}$.
When device B receives the response, it also records a timestamp, $T_{B0}$.
After some delay, device B will send a response to A that contains $T_{B0}$ and the current timestamp $T_{B1}$.
On receiving the response, Device A records its timestamp, $T_{A1}$.
The round trip time $T_{round}$ can now be calculated using the timestamps from A:
\begin{equation}
	\mbox{$T_{round}$} =
	\mbox{$T_{A1} - T_{A0}$}
\end{equation}
The reply delay $T_{reply}$ is calculated using the timestamps from B:
\begin{equation}
	\mbox{$T_{reply}$} =
	\mbox{$T_{B1} - T_{B0}$}
\end{equation}

The ToF can be calculated by subtracting these values.
Since the messages were sent twice the same distance, the ToF must be halved before multiplying it with the speed of light to get the distance.
\begin{equation}
	\mbox{$distance$} =
	\mbox{$(\frac{1}{2}\cdot T_{round}-T_{reply}) \cdot c_{air}$}
\end{equation}

\begin{figure}[ht!]
\centering
\includegraphics[width=\linewidth]{graphics/schematics/SingleSidedTwoWayRanging.png}
\caption{timeline of single-sided two-way ranging (SS-TWR)\cite{IEEE4z}}
\label{f:ss_twr}
\end{figure}

\textbf{Double-sided two-way ranging (DS-TWR)}:
DS-TWR involves both devices A and B performing an SS-TWR and calculating the average between the results.
Figure \ref{f:ds_twr} shows the two separate ranging sessions.
Their result can then be combined to the average ToF for a single message:
\begin{equation}
	\mbox{$T_{prop}$} =
	\mbox{$\frac {T_{Round1}\cdot T_{Round2}-T_{Reply1}\cdot T_{Reply2}}{T_{Round1}+T_{Round2}+T_{Reply1}+T_{Reply2}}$}
\end{equation}
\begin{equation}
	\mbox{$distance$} =
	\mbox{$T_{prop} \cdot c_{air}$}
\end{equation}

The two ranging sessions can have one message overlapping.
Figure \ref{f:ds_twr_3} shows the timeline of an overlapping DS-TWR that only requires three messages.


\begin{figure}[ht!]
\centering
\includegraphics[width=\linewidth]{graphics/schematics/TwoSidedTwoWayRanging.PNG}
\caption{timeline of double-sided two-way ranging (DS-TWR)\cite{IEEE4z}}
\label{f:ds_twr}
\end{figure}

\begin{figure}[ht!]
\centering
\includegraphics[width=\linewidth]{graphics/schematics/TwoSidedRangingThreeMessages.PNG}
\caption{timeline of double-sided two-way ranging (SS-TWR) with three messages\cite{IEEE4z}}
\label{f:ds_twr_3}
\end{figure}


\subsection{K-Connected Graphs}
\label{ss:k_connected_explained}
In order to build a working positional model based on distance measurement, some background in graph theory is required.
The k-connected subgraph of a graph is a subgraph where it takes at least k removals of vertices to create two isolated subgraphs.
A graph $(V,E)$ can have multiple k-connected subgraphs.
They build the set $S_{(V,E)}$.\\
A minimally weighted k-connected subgraph of a weighted graph $S_{(V,E,w)}$, is a k-connected subgraph $(V',E',w') \in S_{(V,E)}$ that has the smallest sum of weights of all k-connected subgraphs.\\
A metric graph is a weighted graph that satisfies the triangle condition.
Meaning for any three edges $e_{AB},e_{BC},e_{CA} \in E$  that connect three verteces $A, B, C \in V$, it holds that $e_{AB} + e_{BC} \geq e_{CA}$.


Finding minimally weighted k-connected subgraphs is an NP-hard problem.
Kahn et al. \cite{khan2007simple} developed distributed approximation algorithm that finds weighted k-connected subgraphs in metric graphs.
It gives an approximation ratio of $\mathcal{O}(k\cdot log(n))$.
This means that the aproximated solution $w^{apr}$ and the optimal sollution $w^{opt}$ fullfill $w^{apr} < k\cdot log(n)\cdot w^{opt})$.
The algorithm puts all vertices in an order, which is determined randomly, and assigns each vertex a rank based on the order.
Each vertex then removes all edges, except for the k lowest weighted ones that connect it to a vertex with a higher rank.\\
There are more precise approximation algorithms to find minimally weighted k-connected subgraphs \cite{kortsarz2003approximating, kortsarz2004approximation}. However, they are centralized, meaning the graph has to be known in its entirety by one actor.


\section{Related Work} % application of concepts
\label{s:related_works}

This Section presents on overview of the literature relevant to this thesis.
This includes IoT systems used in the art world, sensor networks, and wireless ranging, showing the current practices and current state of research.


\subsection{Artwork Tracking}


Since art preservation is an old field and temperature, humidity, light, and vibrations have been known to be detrimental to most artworks, especially paintings, most research in this direction is older than 20 years \cite{mecklenburg1991mechanical, michalski1991paintings, saunders2004effect}.
Still, the invention of new technologies, such as pattern recognition using artificial intelligence, an improvement on existing tools like infrared imaging, and an active need for solutions have kept the research into artwork preservation an active field \cite{borg2020application, schito2017integrated}.
One such new technologies are sensor networks, which have become widespread in the field of art preservation \cite{shah2016customized}.


Artwork tracking during transportation has not been a significant focus in academia.
The most relevant related research was done by Fort et al. \cite{landi2022iot}.
They developed a low-cost, low-powered sensor node to track the temperature, humidity, pressure, and vibrations of artwork and wooden structures.
The sensor node would then report its findings to a remote server.
They confirmed the validity of their sensor in a series of experiments performed in a static building.
They also presented a theoretical framework for their sensor to be used in a transportation scenario, but they do not report implementing or testing this system.
Their sensor used an accelerometer to detect vibrations, and the Bosch BME280 sensor to detect pressure, temperature, and humidity.
Their sensors did not build a network and were not queried but reported their findings directly to either a Wlan router or a BLE-capable smart device.
Fort et al.'s research showes the value of low-cost sensors in detecting threats to artwork.

\subsection{Sensor Networks}

Wireless Sensor Networks (WSN) have become a central aspect if IoT.
Researchers have tried to focus on the most prevalent problems arising from the development of NSWs, mainly power management, security and privacy, data integrity, and availability \cite{gulati2022review}.


\cite{garg2021healthcare} researched WSNs outside of the controlled environment of a house. They propose a WSN that can track the vitals of mountaineers and call for help when measurements have dangerous values. 
They used an Arduino Mega board equipped with a radio transceiver, using LoRa with a star topology.


\cite{jones2021wireless} created a WSN of NRF24101 board intended to monitor linear infrastructures like deepsea wires, using radio and Wi-Fi for communication. Using deep sleep, they were able to optimize energy usage, so the sensor is predicted to last five years on battery.


\cite{spandonidis2022evaluation} used an accelerometer to detect pipeline vibrations to discover leaks. 
They used a narrowband connection for communication and GPS for localization.
Their sensors could query each other for data, to provide a more complete image of the situation.


\subsection{Wireless ranging}

\cite{li2024indoor} made an overview of publications involving positioning systems for industrial settings. 
They looked at the positioning systems in papers using RFID, BLE, UWB, Wi-Fi, and ZigBee. They found that UWB consistently reported the highest accuracy of these methods.
UWB was the least affected by multipath effects, although it was still the most common issue with this technology.


Early research on ranging using UWB was done by Gezici et al. \cite{gezici2008survey, gezici2005localization}.
These papers gave an overview of the different positioning systems for UWB, angle of arrival, received signal strength, time of arrival, and time difference of arrival.
Time of arrival and time difference of arrival were studied further in these publications, presenting error sources and mitigation tools.


Early research focused on the augmentation of UWB-ranging methods.
\cite{venkatesh2007nlos} proposed using integer programs for mitigating the error for ranging without line-of-sight.
\cite{guvencc2007nlos} tried to solve the same issue by using methods based on the statistics of multipath effects.
BiasSub and BiasRed were proposed to reduce the bias in time difference of arrival by applying of a well-known algebraic explicit solution for source localization  \cite{ho2012bias}.
\cite{fan2017performance} improved UWB ranging by eliminating random error. They did this by pre-filtering, using an anti-magnetic ring to eliminate outliers, and using the double-state adaptive Kalman filter to improve position accuracy.
Newer research has also begun incorporating neural networks into UWB positioning systems \cite{stahlke2020nlos, ridolfi2021uwb, che2020machine}.

UWB localization has been used in many applied contexts.
It has been proposed for pedestrian tracking \cite{otim2019effects}, drone flying \cite{macoir2019uwb}, robot navigation \cite{zhu2020adapted}, navigation system for visualy impared people \cite{rosiak2024effectiveness} and tracking people in buildings \cite{elbaum2024investigating}.
UWB positioning systems are particularly interesting for industrial IoT settings.
\cite{barbieri2021uwb} measured the performance of three different UWB antennas: Qorvo, Sewio, and Ubisense. They encountered many multipath-effects in such a complex environment. They mitigated this by employing a Bayesian filtering method.
\cite{belli2024cloud} used UWB positioning, in combination with Real-time kinematic positioning, to track workers while monitoring the factory. The goal was to trigger an alarm if a dangerous situation occurred.



%\chapter{Design}
\label{c:design}

This section presents the principle design of the monitoring system.
The used components are presented in the section \ref{ss:hardware}.
Section \ref{Dataflow} describes the functionalities and responsibilities of the system components.
In Section \ref{ss:network} the network topology and data-flow is discussed

\section{Hardware}
\label{ss:hardware}

This section describes the hardware used in the project. The setup consists of two distinct components: the artwork-tags, of which there are four, and one Phone that provides the interface to the user. The tag itself consists of 4 components:
\begin{enumerate}
	\item nRF52840 microcontroller
	\item DWM3000 UWB Shield
	\item DHT22 temperature and humidity sensor
	\item MPU6050 accelerometer and gyroscope
\end{enumerate}

\subsection{Microcontroler}
The artwork-tag's fundament is built by the nRF52840 DK microcontroller developed by Nordic Semiconductors. 
It is part of the nRF52 series of microcontrollers intended for development.
The nRF52840 DF is specialized for BLE communication, for which it already includes the necessary components.
It is compatible with the nRF52 Software Development Kit (SDK) developed by Nordic Semiconductors.
The SDK makes it possible to use the ble functionalities and to control the pins. 
It also includes implementations for a plethora of pin-based protocols.
It contains 58 pins, 48 of which are data pins, and 10 manage the power supply for additional modules, which include 3.5 and 5 Volt supply pins.
Thirty-two of the pins are installed the same way as the pins on the Arduino uno, making it compatible with many peripherals designed with this common board in mind, such as the dwm3000.
The remaining ten pins are enough to attach the sensors.
The nRF52840 DK includes a USB-B port that is used for powersuply. Additionally, the USB-B port is connected to two pins and is used for UART communication and debugging.
The nRF52 was chosen since it was available, and previous projects have been done with it in combination with the DWM3000 shield.
As a result, a lot of initial setup was already available.

\subsection{UWB shield}
For communication between the tags and distance measurement, the DWM3000 UWB shield developed by Qorvo was chosen.
The DWM3000 is a commonly used device for research involving UWB \cite{coppens2022overview, leu2022ghost, stocker2022performance}.
It allows low-level access but includes an SDK written in C that makes many processes transparent to the user if they wish.
The SDK uses the Serial Peripheral Interface (SPI) for communication between the shield and the microcontroler.



\subsection{Humidity and temperature sensor}

For humidity and temperature sensors I decided to use the DHT22(AM2302) produced by Guangzhou Aosong Electronic Co. \cite{AM2302}. 
It is a commonly used sensor in IoT monitoring systems \cite{ahmad2021evaluation}. 
The vendor claims a temperature range from -40\degree to 80\degree Celsius with a precision of 0.5\degree. 
\cite{ahmad2021evaluation} could experimentally confirm that errors did not exceed 0.1\degree Celsius. 
They also concluded that the sensor is slow in detecting temperature changes. 
This is also confirmed by the user manual \cite{AM2302}, which states that a read-interval of less than 2s is impossible. 

The humidity sensor can detect the full range from 0\% to 99.9\% humidity, with an advertised maximum error of 2 \% pts \cite{AM2302}.
No research confirming or denieing these claims could be found.

The DHT22 sensor uses three pins from the microcontroller: two pins for power supply and ground and one for single-bus communication.
Since no SDK for this type of communication has been built for the nRF52 board series, it had to be implemented manually by reading the high and low voltage on the communication pin, detecting headers and footers, and parsing the binary messages. 

\subsection{Accelerometer and Gyroscope}
The MPU6050 sensor produced by InvenSense Incorporated provides accelerometer and gyroscope data.
The accelerometer reports the acceleration in the three cardinal directions in meters per second.
The gyroscope reports the rotation around the three Euclidean axes in degrees per second.
In this project, the accelerometer data was not used, only the gyroscope.

The MPU6050 uses four pins: two for power supply and ground and two for communication.
The sensor communicates using the I2C protocol, a serial synchronous communication system.
The microcontroller acts as the master and would, in theory, support multiple workers on the same bus. 
Here, only the MPU6050 uses I2C and is the only worker.
While the nRF52 SDK does not supply an I2C API, it offers a Two Wire Interface (TWI) implementation compatible with the I2C protocol.
It used to offer MPU6050-specific support in older versions of the SDK. 

\subsection{Tag technical plan}
The microcontroller builds the base of the tag. 
The other devices are attached to it over the available pins.
In the nRF52 SDK, each data pin is assigned an integer value. 
These often correspond with the pin's name according to the nRF52830 DK manual, but not always.
This Thesis will use the names in the manual to describe the pins.
Pins are the method by which a microcontroller controls its peripherals.


\begin{table}[ht!]
	\centering
	\begin{subtable}[b]{.4\linewidth}
		\centering
		\begin{tabular}{l|l|l|l|l|}
			& Pin & \rotatebox{90}{DWM3000\phantom{.}} & \rotatebox{90}{DHT22}  & \rotatebox{90}{MPU6050\phantom{.}}   \\
			\hline \multicolumn{1}{|l|}{\multirow{10}{*}{\rotatebox{90}{P4}}}
			& P1.10  & \checkmark    &             &             \\
			\multicolumn{1}{|l|}{} & P1.11  & \checkmark    &             &             \\
			\multicolumn{1}{|l|}{} & P1.12  & \checkmark    &             &             \\
			\multicolumn{1}{|l|}{} & P1.13  & \checkmark    &             &             \\
			\multicolumn{1}{|l|}{} & P1.14  & \checkmark    &             &             \\
			\multicolumn{1}{|l|}{} & P1.15  & \checkmark    &             &             \\
			\multicolumn{1}{|l|}{} & GND    & \checkmark    &             &             \\
			\multicolumn{1}{|l|}{} & P0.02  &               &             &             \\
			\multicolumn{1}{|l|}{} & P0.26  & \checkmark    &             &             \\
			\multicolumn{1}{|l|}{} & P0.27  &               &             &             \\
			\hline \multicolumn{1}{|l|}{\multirow{8}{*}{\rotatebox{90}{P3}}}
			& P1.01  & \checkmark    &             &             \\
			\multicolumn{1}{|l|}{} & P1.02  & \checkmark    &             &             \\
			\multicolumn{1}{|l|}{} & P1.03  & \checkmark    &             &             \\
			\multicolumn{1}{|l|}{} & P1.04  & \checkmark    &             &             \\
			\multicolumn{1}{|l|}{} & P1.05  & \checkmark    &             &             \\
			\multicolumn{1}{|l|}{} & P1.06  & \checkmark    &             &             \\
			\multicolumn{1}{|l|}{} & P1.07  & \checkmark    &             &             \\
			\multicolumn{1}{|l|}{} & P1.08  & \checkmark    &             &             \\
			\multicolumn{1}{|l|}{} & P1.10  & \checkmark    &             &             \\
			\hline \multicolumn{1}{|l|}{\multirow{8}{*}{\rotatebox{90}{P1}}}
			& VDD    &               &             &             \\
			\multicolumn{1}{|l|}{} & VDD    &               &             &             \\
			\multicolumn{1}{|l|}{} & RESET  &               &             &             \\
			\multicolumn{1}{|l|}{} & VDD    & \checkmark    & \checkmark  &             \\
			\multicolumn{1}{|l|}{} & 5V     & \checkmark    &             & \checkmark  \\
			\multicolumn{1}{|l|}{} & GND    & \checkmark    & \checkmark  & \checkmark  \\
			\multicolumn{1}{|l|}{} & GND    & \checkmark    &             &             \\
			\multicolumn{1}{|l|}{} & N.C.   &               &             &             \\
			\hline \multicolumn{1}{|l|}{\multirow{6}{*}{\rotatebox{90}{P2}}}
			& P0.03  & \checkmark    &             &             \\
			\multicolumn{1}{|l|}{} & P0.04  & \checkmark    &             &             \\
			\multicolumn{1}{|l|}{} & P0.28  &               &             &             \\
			\multicolumn{1}{|l|}{} & P0.29  &               &             &             \\
			\multicolumn{1}{|l|}{} & P0.30  &               &             &             \\
			\multicolumn{1}{|l|}{} & P0.31  &               &             &             \\
			\hline 
		\end{tabular}
		\caption{Adruino compatible pin assignment}
		\label{table:ArdruinoPins}
	\end{subtable}
	\hspace{.1\linewidth}
	\begin{subtable}[b]{.4\linewidth}
		\centering
		\begin{tabular}{l|l|l|l|l|}
			& Pin & \rotatebox{90}{DWM3000\phantom{.}} & \rotatebox{90}{DHT22}  & \rotatebox{90}{MPU6050\phantom{.}}   \\
			\hline \multicolumn{1}{|l|}{\multirow{8}{*}{\rotatebox{90}{P6}}} 
			& P0.00  &               &             &             \\
			\multicolumn{1}{|l|}{} & P0.01  &               &             &             \\
			\multicolumn{1}{|l|}{} & P0.05  &               &             &             \\
			\multicolumn{1}{|l|}{} & P0.06  &               &             &             \\
			\multicolumn{1}{|l|}{} & P0.07  &               &             &             \\
			\multicolumn{1}{|l|}{} & P0.08  &               &             &             \\
			\multicolumn{1}{|l|}{} & P0.09  &               &             &             \\
			\multicolumn{1}{|l|}{} & P0.10  &               &             &             \\
			\hline \multicolumn{1}{|l|}{\multirow{18}{*}{\rotatebox{90}{P24}}} 
			& P0.11  &               &             & \checkmark  \\
			\multicolumn{1}{|l|}{} & P0.12  &               &             & \checkmark  \\
			\multicolumn{1}{|l|}{} & P0.13  &               & \checkmark  &             \\
			\multicolumn{1}{|l|}{} & P0.14  &               &             &             \\
			\multicolumn{1}{|l|}{} & P0.15  &               &             &             \\
			\multicolumn{1}{|l|}{} & P0.16  &               &             &             \\
			\multicolumn{1}{|l|}{} & P0.17  &               &             &             \\
			\multicolumn{1}{|l|}{} & P0.18  &               &             &             \\
			\multicolumn{1}{|l|}{} & P0.19  &               &             &             \\
			\multicolumn{1}{|l|}{} & P0.20  &               &             &             \\
			\multicolumn{1}{|l|}{} & P0.21  &               &             &             \\
			\multicolumn{1}{|l|}{} & P0.22  &               &             &             \\
			\multicolumn{1}{|l|}{} & P0.23  &               &             &             \\
			\multicolumn{1}{|l|}{} & P0.24  &               &             &             \\
			\multicolumn{1}{|l|}{} & P0.25  &               &             &             \\
			\multicolumn{1}{|l|}{} & P1.00  &               &             &             \\
			\multicolumn{1}{|l|}{} & P1.09  &               &             &             \\
			\multicolumn{1}{|l|}{} & GND    &               &             &             \\
			\hline
		\end{tabular}
		\caption{Non-Adruino compatible pin assignment}
		\label{table:otherPins}
	\end{subtable}
	\caption{Pin assignments and compatibilities}
\end{table}

Some pins are intended for power supply.
On the NRF52840, these pins are located in section P1; see table \ref{table:ArdruinoPins}.
The three VDD pins supply electricity with a Voltage of 3.5 Volts.
A secondary power supply that uses 5 Volts is also available.
What voltage is needed depends on the peripheral.
In this case, the DHT22 runs on 3.5 Volts, while the MPU6050 is made for 5 Volts.
The P1 section also contains two ground pins that need to be connected to the peripherals and a reset pin to restart the microcontroller.
The last pin is not connected (N.C.).
There are additional ground pins in sections P4 and P24 of the board.

The other pins are called data pins.
These pins can transfer data and be used for communication by using voltage modulation.
The nRF52840 has an I/O voltage of 3.3 Volt.
This means that a voltage of 3.3 Volt corresponds to a \textit{Logic high} and 0 Volts represents a \textit{Logic low}.
This allows the data pin to transfer communication in a binary encoding.
How a signal is interpreted is defined by the used communication protocol.
The MPU6050, for example, uses the I2C protocol and uses two datapins.
The protocol states that one pin is used for a serial clock, and the other pin transmits data.
For the data transmission, the protocol defines what a package looks like.
This includes the start condition, the voltage characteristics that signal the beginning of a package, addressing, data encoding, acknowledgments, and stop condition.

The DHT22 sensor does not use a given communication protocol.
It uses one data point to report its sensor data.
How that data is encoded to high and low voltage is specified in the user manual \cite{AM2302} and has to be implemented manually.

The DWM3000 shield is mounted on the 32 pins for Arduino one connections. 
All pins are forwarded and can be used by other devices in a common Arduino-stackable style.
If they are data pins, they will share the data.
Table \ref{table:ArdruinoPins} shows which devices use which pins.
The only pins shared by multiple devices are power and ground pins.
The microcontroller supplies enough power to support this.

The sensors are attached to the same power source and ground as the shield but use different data pins.
The DWM3000 leaves enough pins unused that both sensors could be attached to them.
Since it is not visible which pins the shield leaves free, it was decided to use data pins that are not attached to the DWM3000 for the DHT22 and MPU6050.
Table \ref{table:otherPins} shows how the sensors are connected to the remaining open pins.

\section{Architecture}
\label{ss:dataflow}

In order to discuss how the dataflow works, first, Section \ref{ss:responsibility} will establish what services are implemented in each part of the system.
Section \ref{ss:dataflow} will explain what triggers events and how they are handled inside the system.

\subsection{Responsibilities}
\label{ss:responsibility}
The system consists of the tags, the sensor network, and the phone.
These parts all have their own responsibilities.

\textbf{Tag:} 
The tag is responsible for managing its sensors. 
It has to do the correct setup and convert its output into an understandable form.
The tag can perform ranging with all its neighbors.
Additionally, the tags are responsible for searching for networks to join and reacting appropriately to network requests, be those queries for sensor data, ranging requests, or network management jobs. 
The tags provide a unique, secure universal identifier to be used by queries or the network.
How this is done is part of the certified project and will not be discussed in this thesis.
The tag is also responsible for its own power management.
This is not the focus of this thesis and will only be mentioned when relevant.
A guideline on power management will not be provided here.

\textbf{Network:} The network is responsible for keeping track of all tags taking part in the network. 
It offers a joining protocol for new devices and remains stable when devices leave or become unavaliable. 
It offers the possibility for phones to connect to the network. 
It ensures quieries from phones get transported to the correct tag and the answers to the correct phone.
It ensures a network topology that corresponds to a graph that is at least 3-connected.
On request, it returns a list of devices connected to the phone.

\textbf{Phone:} The phone connects to the network via the provided method.
It offers a graphical user interface (GUI) for the driver to use.
The GUI offers a method for the driver to set the acceptable ranges for all sensor data.
Additionally, it offers a method to set query interval-length.
The phone is responsible for querying sensor data for each tag and measuring once at each interval.
The phone has to evaluate the answer.
The phone has to report the results to the driver using the GUI.
If a parameter falls outside of the acceptable range for its type, the phone is responsible for alerting the driver to this fact.

The Certify project also plans to collect the sensor data on remote servers using a 4G connection.
The plan is to equip each tag with antennas to allow it to send the data directly to the server itself.
Since this is not a part of this thesis, the responsibility for the tag to do this was not added to a list.
A known problem with this plan is, that a 4G connection is not always possible.
Since small tags have very limited memory, the plan to store the sensor data on the tag is not feasible.
If the setup presented in this thesis is used, it would allow for the storage of the data on the phone, which has a much larger memory.
This, again, was not added since it is not part of this thesis.


\subsection{Dataflow}
\label{ss:dataflow}

\begin{figure}[ht!]
	\includegraphics[width=\linewidth]{graphics/schematics/obervation_loop.png}
	\caption{Sequence diagram of setup and observation loop. Setup is performed once, and the observation loop repeats until it is stopped.}
	\label{f:observation_loop}
\end{figure}

Section \ref{s:network} describes how a tag connects to the network.
Figure \ref{f:observation_loop} shows a sequence diagram of the setup and main observation loop of the system.
At the top, the communicating parts are listed.
\begin{itemize}
	\item Human is the driver of the truck
	\item Phone is the phone used by the Human
	\item Network consists of all the used tags and the network they build.
	\item Connected Tag is part of Network but is listed separately. It represents the tag that is communicating to the phone
	\item Tag-j is also part of Network. It represents the tag that is queried during the observation loop
\end{itemize}
Phone and Human communicate by using a GUI. 
Phone and Connected Tag communicate using BLE. 
Every communication inside Network happens using UWB. This includes the communication between the Connected Tag, Network, and Tag-j. \\
When Phone wants to connect to Network, it looks for advertised BLE devices.
It then displays the devices to Human and lets them pick one.
The phone then pairs with the chosen tag, making it the Connected Tag and Phone's connection to Network.
Once connected to Network, Phone will prompt Human to enter the parameters.
These consist of:
\begin{itemize}
	\item Upper and lower limit for sensor data, like temperature and humidity
	\item Maximal displacement value for distance and gyro. These values represent the maximal difference in registered values that is allowed for positional measurements.
	\item Time between measurements. This gives the time period that will pass between measurements for each device and measurement type.
\end{itemize}
Once the parameters are chosen, Human can start the observation.\\
Each iteration of the observation loop begins with a call to Network for a list of all tags currently in Network.
Since the tag network is a dynamic sensor network, the tags in Network can theoretically change. 
In practice, this should only happen when artwork is loaded/unloaded or a tag becomes faulty.
The request for the list is transmitted to Connected Tag over BLE, which then queries Network for all connected devices.
The response is returned to Phone.
Phone then starts a nested loop, iterating over the list of tags and metrics captured by the system.
For each measurement and tag combination (i,j) Pphone contacts Connected Tag for the value, which in turn queries Network.
Once the message arrives at Tag-j, Tag-j gets measurement i. 
In the case of sensors, this entails contacting the sensor and requesting a value.
If metric i is a distance measurement, tag j will commence a two-way ranging operation over UWB with all its registered neighbors and will report the list of distances, together with the tag addresses they correspond to.
Metric i is then transported over Network back to Connected Tag and finally to Phone.
Phone must then evaluate the retrieved data. \\
During the evaluation process, Phone creates an evaluated measurement and marks it as problematic or unproblematic.
What the evaluation looks like depends on the metric.
\begin{itemize}
	\item For most metrics, like humidity and temperature, the evaluated measurement is equivalent to the received measurement. It is then checked if the measurement falls into the acceptable measurement parameters set by Human.
	If it does not, the evaluated value is marked as problematic.
	\item Some metrics require comparison to the previous data. The gyroscope reports the current orientation of the tag.
	This is then compared to all previous measurements, and the maximal angular difference forms the evaluated measurement.
	If the evaluated metric is bigger than allowed by the set parameters, the measurement is marked as problematic
	After evaluation, the original measurement is added to the list of previous measurements.
	\item The distance measurement has a unique evaluation process, which is described in Section \ref{ss:distance_eval}.
\end{itemize}
Once the data evaluation is done, the evaluated measurement is presented to the user over the GUI, together with the address of the tag it belongs to.
If the evaluated measurement is problematic, the driver will be alarmed.


\subsubsection{Distance evaluation}
\label{ss:distance_eval}
The goal of the distance evaluation is to build a working model of where every tag is.
To achieve this, a quadratic program is solved to get the coordinates of all tags.
The steps to do this are as follows:
\begin{enumerate}
	\item Get a list of all current tags, $T:=\{ t_1, t_2, ... \}$.
	\item For each tag, get the last known distance measurements and put it into a set $S_D:=\{ (t_i, t_j, d_{ij}) \} $, where $t_i$ is the tag which measured, $t_2$ the tag that was measured to and $d_{ij}$ the distance measured.
	\item If a tag has no distance measurements, remove it from the list.
	\item Assign each tag $t_i$ a position in a 3D coordinate system, $(x_i,y_i,z_i)$
	\item Pick one random tag $t_{o}$.
	\item Set the values $x_{o},y_{o},z_{o}$ all to 0.
	\item Create the objective function: $L(X,Y,Z) = \sum\limits_{t_i, t_j, d_{ij} \in S_D}|(x_i-x_j)^2+(y_i-y_j)^2+(z_i-z_j)^2-d_{ij}^2|$. For x,y, and z, use variables for all but the six values set in step (6).
	\item Solve the quadratic program consisting of the Objective function L, and no constraints.
\end{enumerate}
Quadratic Programs, in general, are NP-Hard, but Quadratic Programs with a convex function can be solved efficiently.
$(a-b)^2$ and $c$ are convex functions.
The sum of a convex function is always a convex function.
The objective function in (7) only sums up convex functions and is, therefore, convex.
The quadratic program can, therefore, be solved efficiently.\\
By setting the values of the tag $t_{o}$ to zero, the results of the quadratic function become grounded.
It is not strictly necessary, but without it, the returned solution could have values anywhere in the Euclidean space.
The solution will place the other tags near that region by setting one tag to the coordinates at the origin.
There are still an infinite amount of solutions to this quadratic function since all solutions can be rotated around any axis and still return the same objective function.

\begin{figure}[ht!]
	\begin{subfigure}{.4\linewidth}
		\centering
		\includegraphics[height=150px]{graphics/schematics/connected_dots.png}
		\caption{3-edge connected graph}
	\end{subfigure}
\hspace{1cm}
	\begin{subfigure}{.4\linewidth}
		\centering
		\includegraphics[height=150px]{graphics/schematics/connected_dots_k_connected.png}
		\caption{2 connected graph}
	\end{subfigure}
	\caption{ Left: Five dots, all having at least two connections, still blue can move independently. Right: minimal 2-connected graph, no movement possible. }
	\label{f:connected_dots}
\end{figure}

For a point to be clearly placed in Euclidean space, three distances to other points must be known.
This alone is not sufficient to ensure unique results.
The left of Figure \ref{f:connected_dots} illustrates this point in two-dimensional space.
Every circle is connected to two others, but the blue circles can still move without the whole figure moving.
What is needed to keep every point static is for known distances and tags to build a four-connected graph (three in two dimensions).
The left of Figure \ref{f:connected_dots} shows a solution to the problem on the right by creating a three-connected subgraph.

Once the coordinates for all tags are found, they are compared to previous results.
For each tag, the phone calculates how much it has moved.
The evaluated measurement is the distance of the tag that has moved the most.
If the evaluated measurement is larger then the maximal allowed displacement, the measurement is problematic.



\section{Network}
\label{s:network}

For the presented network to work, tags need to be ranked.
This means that for each tag pair $i,j$, one can either say that $rank(i)<rank(j)$ or $rank(j)<rank(i)$.
To achieve this, the universal unique identifiers are used.
No matter what form the UUID has, it can be converted to an integer, by simply interpreting its binary code as one.
Since the UUIDs are unique, no to tags will have the same resulting integer.
When referring to the rank of a tag in this section, the integer representing the UUID is intended.

While not connected to a phone, the tags inside the truck form a decentralized mesh using UWB for communication.
Each tag keeps two lists: a list of known devices and a list of neighbors.
When a new tag joins the network, it sends a joining request over UWB, containing its UUID, using a weak signal.
All tags in the network that receive this request add the new device to their list if known devices. 
If the new device also has a higher rank, they additionally add it to their list of neighbors.
They then answer by sending their own UUID and adress back to the new tag.
By waiting an amount of time that correlates with their UUID, the tags in the network can ensure that their answers don't overlap.
The new tag adds the received addresses and UUIDs to its known device list. If the added tag's rank is also higher than the new tag, it will add it to the list of neighbors.
If the new tag now has four neighbors, it stops. Otherwise, it will repeat the process with an increasingly stronger signal until it has either found four tags with a higher rank or reached maximum signal strength.
Afterward, it starts advertising its BLE connection.
This concludes the network joining process.


A user with a phone can connect to any of the advertised BLE connections.
Once that happens, the tags in the tracks will switch from their decentralized mesh to a star topology, with the connected device serving as the coordinator.
The coordinator will inform all tags about their new status by sending a message using a strong UWB signal.
The tags will then acknowledge this message in order of rank.
The tags in the network will still keep their stored neighbors and known devices.
The coordinator records a list of all acknowledgemtns, thus creating a list of all devices in the network.


The phone can request the list of all tags from the coordinator.
The phone can now also query the tags in the truck by sending the query to the coordinator over BLE, which then will pass it directly to the appropriate tag using BLE.
For all sensor data, this is a simple call-and-response request. \\
If a distance measurement is queried, a tag take the following steps:
\begin{itemize}
	\item It conducts a UWB two-way ranging session with each tag in the neighbor list.
	\item It reports those results to the coordinator tag.
	\item It orders all received distances.
	\item It keeps the tags with the four lowest distances and deletes the rest from the neighbor list.
\end{itemize}
The first time a distance is requested, the tag will perform more ranging sessions than necessary to build a 4-connected graph.
Afterward, it only performs ranging with four other tags unless a new device is added.
Suppose a ranging session does not report a result because a tag left or became unavailable. In that case, the tag adds ads the tag with the shortest previously measured distance and higher rank from the list of known devices back intro the list of neighbours.


This design mirrors the algorithm proposed by \cite[khan2007simple] and presented in Section \ref{ss:k_connected_explained}.
It creates an approximation of a minimally weighted k-connected subgraph based on the measured distances.
This is allowed since the distances are in Euclydean space, which, when mapped to a graph, forms a metric graph.
As discussed in section \label{ss:distance_eval}, a four-connected graph is needed to uniquly identify the position of each tag.
The graph should be minimally weighted so that measurements are between tags that are as close as possible to each other.
This reduces the multipath effect and theirfore increases precision.


If the tags are not connected to a phone and report their data to a remote server, they can still use the same distance measurement to approximate the k-connected subgraph.
The quadratic program can then be calculated on the server.
%\chapter{Implementation}
\label{c:implementation}

In this section, the implementation that was used for the experiment is discussed.
In section \ref{s:tag} the implementation of the tags is presented.
Section \ref{s:app} is about the implementation of the App.

\section{Tag}
\label{s:tag}
The software of the tags consits of n modules:
\begin{enumerate}
	\item Temperature and humidity sensor
	\item Gyroscope
	\item UWB network
	\item Two way ranging
	\item BLE communication
	\item Job handler
\end{enumerate}
The following subsections will discuss the first five modules, followed by how they interact using the job handler module.
The section \ref{ss:combination} discusses chalanges from combining these modules and how they were solved.

\subsection{Temperature and humidity module}
\label{ss:temp_hum_module}
This module is responsible for managing the DHT22 humidity and temperature sensor.
It is responsible to setup the sensor during initial startup and to provide the sensors measurements when queried.
The DHT22 sensor communicates using only one data pin, pin 13, which will be referd to as the data pin in this section.
Dmitry Sysoletin created an implementation \ref{sysoletin2021nrf52_dht11} for the DHT11 sensor together with the nRF52840 board that build the basis for this implementation, by addapting it to the DHT22 and adding functionalities needed by the job handler module.


Since the DHT22 is a very simple sensor, using single bus comunication, not much setup is needed.
The evaluation of the sensor data requires that the voltage of the pin is read out in pre-defined intervals, when reading the sensor data.
To do this, a clock is required.
This resource has to be reserved an initiated at startup.
This is the only setup that is required for the DHT22 sensor.


To initiate a sensor-read the voltage of the data pin is set to 0.
When the sensor is in standby mode, the data pin is on \textit{logic high}, and when set to \textit{logic low}, the sensor will respond with a read of its current value.
A schematc view of a sensor read of the DHT22 can be seen in Figure \ref{f:dht22_signal}.
The temperature and humidity module will then check the Pin State in intervals of 5ms, until a \textit{logic low} is registered, signalling that the sensor has registered the request. 
The module will now monitor the pin state, waiting for \textit{logic low} followed by a \textit{logic high}, this beeing the start condition of the data transfer. \\
The data is transfered in five chunks of eight bits.
Each bit is preceeded by a prolonged \textit{logic low} state, that is detected by the module
The module then proceeds to write the state of the data pin into a 8-bit buffer, \textit{logic high} corresponding to a 1 and \textit{logic low} to 0.\\
Once all five chunks are read, the communication has ended and the module can verify the data.
The first two bytes correspond are combined to form the temperature information in celcius, the second and third form the humidity.
Both values are multiplied by 100 and stored in a 16-bit integer. This doesn't loose data, since the sensor only measures up to a precision of 1 after the decimal point.
The data beeing stored in an integer help with data transfer.
It will be converted back on the phone.
The fith chunk contains the parity and is used to accept or reject the humidity and temperature values.
If the process fails at any state, -100\degree C is returned for the temperature and -100% for humidity.
These form both impossible values, since humidity can't be negative and the DHT22 sensor can only detect temperatures as low as -20\degree C.


\begin{figure}[ht!]
\centering
\includegraphics[width=\linewidth]{graphics/DHT22_signal.png}
\caption{Signal of a DHT22 sensor-read as presented in the manual \cite{AM2302}.}
\label{f:dht22_signal}
\end{figure}

\subsection{Gyroscope}
\label{ss:gyro_module}
This module manages the MPU6050 gyroscope and accelerometer.
It is responsible to setup the sensor and report its result.
An implementation for the MPU6050 was present in the nRF52 15.3.0 SDK, but is no longer avaliable for the nRF52 17.1.0 SDK, used in this project.
The old implementation was ported to this project.
This consisted of replacing deprecated parts of the SDK with updated ones and adding newly required flags to the build.


MPU6050 sensors use the I2C communication protocol.
The nRF52 SDK does not include an implementation for this protocol, but has a Two Wire Interface (TWI) implementation that is compatible with the I2C protocol.
During startup the TWI module has to be initialized.
This is handled by the SDK, but requires some parameters to be passed.
\begin{itemize}
	\item The Serial Clock Line (SCL) defines what pin will be used for the clock shared in the TWI. This implementation uses pin 11.
	\item The Serial Data Line (SDA) defines which pin is used for the data communication. Pin 12 was used.
	\item The frequency which the TWI uses. It is defined in MPU6050 data sheet, and is 100 kHz \cite{MPU6050}.
	\item The Interrupt priority is a rank that determins, how easyely this process can cause an interrupt. It is set to high.
\end{itemize}
After the TWI service is initiated with these parameters, it is enabled, ensuring that its resources are locked and can not be used by other services.


Afterwards the results from the sensor can be read using the TWI service again.
The TWI-TX requires the adress of the read device and a registry where to write the MPU6050 datasheet \cite{MPU6050}.
The adress of the sensor is the same for all MPU6050 sensors and can be found in the manua
It sets a flag to true once the sensor has writen the data, which then can be read using the TWO-RX function.
The result consist of three 16-bit integers, representing the angular velocity arround the X,Y and Z axis, shown in figure \ref{f:MPU6050_orientation}.\\
Returning this data when queried has only limited use.
It represents a measurement of the current situation.
The caller is more interested what happened since the last query.
Two different implementations for the read of the gyroscope were used during the experimental phase of this thesis.
One would try to return the current orientation of the tag. This read will be called the \textit{orientational read}.
The other would return the maximal registered angular velocity since the last read. This will be called the \textit{angular velocity read}.


\begin{figure}[ht!]
\centering
\includegraphics[width=200px]{graphics/MPU6050_orientation.png}
\caption{Schematic view of the MPU6050, showing the direction of the three axis X,Y,Z.}
\label{f:MPU6050_orientation}
\end{figure}



To achieve the orientational read, three orientational variables $x_{angle}$, $y_{angle}$ and $z_{angle}$ keep track of the current rotation around their corresponding axes.
During setup, all three angles are set to zero.
The MPU6050 is read out periodically in between calls.
The elapsed time since the last read is multiplied with the angular velocity at this moment arround the axis and is added to the orientational variables.
When the gyroscope module is queried for its measurement, $x_{angle}$, $y_{angle}$ and $z_{angle}$ are returned.


The angualr velocity read is achieved in a similar manner.
Three angular velocity variables $x_{max}$, $y_{max}$ and $z_{max}$ are created and set to zero during initiation.
The MPU6050 is read out periodically and its values are compared to the angular velocity variables.
If any of the angular velocity values is smaller in absolute magnitude than the corresponding read value, it is replaced by that read value.
When the the gyroscope module is queried, the values of $x_{max}$, $y_{max}$ and $z_{max}$ are returned.
The angular velocity variables $x_{max}$, $y_{max}$ and $z_{max}$ are then set to zero again.

\subsection{Network}
\label{s:network}
The network module is responsible for the managment of the network.
This consists of: sending requests to join a network, managing requests to join a network, keeping track of its neighbours, transmitting messages and sending messages.
Since only four devices were used in this implementation, the processes for the network are much more simplified, then presented in design chapter.
A 4-connected minial graph of 4 verteces must nececarily include that all the nodes are fully connected.
This leads to a simplified network architecture.
Since this implementation was build to run experiments and not to be used in real-world applications, a lot of security measures were canceled.
Messages are not encrypted and devices are not authenticated.
All messages are assumed to reach their destination and no devices is expected to become unavaliable. \\
The Network-module is based on the implementation of \cite{degkwitz2023ultrawideband}.
It is based on published examples from Qorvo, the producer of the DWM3000 shield.
It uses the DWS3000 SDK to communicate with the DWM3000.


The DWM3000 uses the Serial Peripheral Interface (SPI) protocol.
This requires some resources that have to be reserved and some configurations that need to be set.
This is the first thing that happens during the setup of the UWB network.
Next the interrupt-priorities and the communication speed of the SPI connection are configured.
Then the DWM3000 is reset, to insure no cross-effects from previous sessions are possible.
Then the board is told to initialize.
After that the used configurations are send to the shield.
This includes information like chanel number, preamble codes, data rates and header modes.
The SDK contains many pre-defined configurations.
All configurations that allow for RX and TX and that use scrambled timestampt (STS) work for this usecase.
It is crucal that all tags use the same configurations.
For this implementation the same configurations were used, as in \cite{degkwitz2023ultrawideband}.
The configurations can be seen in table \ref{table:DWM_settings}.
The setup finishes with initiating the LEDs, that serve no critical service, but are usefull for debugging.



\begin{figure}[ht]
\caption{Configurations of the DWM3000 for UWB communication}
\begin{longtable}{|l|c|}
\hline
\textbf{Description} & \textbf{Value} \\
\hline
\endfirsthead
\hline
\endfoot
Channel number & 5 \\
TX preamble length & 128 symbols \\
RX preamble acquisition chunk size & 8 chunks \\
TX preamble code & 9 \\
RX preamble code & 9 \\
SFD type selection & 4z 8 symbol\\
Data rate & 6.8 Mbits/s \\
PHY header mode & standard PHR mode \\
PHY header rate & standard PHR rate \\
SFD timeout & 129 \\
STS mode & enabled \\
STS length & 128 bits \\
PDOA mode & off \\
\hline
\end{longtable}
\label{table:DWM_settings}
\end{figure}


The Certify project uses unique, falsifiable identifiers for its tags.
Since this is not avaliable for the tags used here, the device-ID was used instead.
It serves as a 8-bit long adress for the purposes of this implementation.
Each tag also keeps a list of all known adresses, called neighbours.


The adress \textit{0x3F} was used, when a tag wants to join a network.
This was chosen since none of the used devixes had this device-ID, and it corresponds to a question mark when using ASCI encoding.
When a tag wants to joind a message, it sends the adress \textit{0x3F}, followed by the message 'findnet', and its own adress.
It then start listening for answers.
If the listening timed out without any answers, it sends the message again.


For the network to function, the receiving and sending of messages is critical.
The UWB listener function from project \cite{degkwitz2023ultrawideband} was modified.
It waits for a listened message from the shield.
If it receives a message, it coppies it to a buffer.
It then checks the first bit of the message for the receiver adress.
If the receiver adress is equivalent to the tags own adress, it passes the message on to the job-handler module for further evaluation.
Otherwise the message is discarded.
An exception is made, if the receiver-adress is "\textit{0x3F}", indicating that a tag is looking for a network.
In that case, the network module adds the tag to the list of neighbours.
It then waits for a time proportional to its own adress, before continuing.
Since adresses are unique, this ensures that no two tags responde to the new tag at the same time.
Afterwards it sends a new message, beginning with the adress of the new tag, followed by the string 'NEW' and its own adress.
This way it can be added to the neighbours of the new tag as well.\\
For sending messages, the implementation of \cite{degkwitz2023ultrawideband} was modified.
It sets the DWM3000 to TX, passes a int-buffer and lets it transmit, before returning to RX mode.
Do to limitations discussed in section \ref{ss:combination}, the message length could not exceed 10 bytes. 


\subsection{Two-way ranging}
\label{ss:two_way_ranging}
The two-way ranging module is responsible for measuring its distances to the tags in the neighbourhood.
Since it also uses the DWM3000 shield, it requires no additional setup.


When the two way ranging module gets a distance request, it loops over the list of neighbours, performing two-way ranging with each of them.
First it sends a prepare-rainging request to the neighbour it wants to performe ranging with, before performing the ranging.
It then sends the result back over the network to the requesting tag with the following format: 
\begin{equation}
	\mbox{$a_r$DST$a_ta_ncd_{tn}$}
\end{equation}
with
\begin{itemize}
	\item $a_r$: The adress of the requesting tag.
	\item DST: The string "DST", indicating the purpose of the message.
	\item $a_t$: The own adress of the tag performing the measurement.
	\item $a_n$: The adress of the neighbour that the distance was measured to.
	\item $c$: A boolean. If false, this is the last neighbout measured for this query.
	\item $d_{tn}$: The distance to measured. 
\end{itemize}
The reason for each measurement triggering its own response is the message-lenght limitation mentioned in section \ref{s:network}.

When a tag receives a prepare-rainging request intended for another device, it enters a short sleep.
This is because rainging envolves multiple messages beeing send netween both participants.
This would unnecesarily drain energy from the tags that are not envolved.
Because of that they sleep for the expected duration. \\
If the tag is the entended receiver for the prepare-rainging message, it will enter the preparation part of the two-way ranging module.
If will function as deivce A in respect to figure \ref{f:ds_twr_3}.
In a first step, it will clear all RX and TX buffers.
It then sets the expected RT and TX antenna delays, $d_{rx}$ and $d_{tx}$.
They represent the expected time loss during receiving or transmitting messages and are device specific.
These delays will automatically be taken into account, when calculating the timestamps.
It then sends the first polling message and imideatly starts waiting for a response.
The polling message is a constant string with no changing data.
The DWM3000 will automatically store the transmission and reception timestamps, their is no need to retreive it right away.
When the response is received, it checks if it is the expected response.
If it is, the two timestamps $T_{TX_1}^A$ and $T_{RX}^A$ are retireved.
The final transmission time $T_{TX_2}^A$ is calculated by adding a constant $c_A$ to $T_{RX}^A$:\begin{equation}
	\mbox{$T_{TX_2}^A$=}
	\mbox{$T_{RX}^A+c_A$}
\end{equation}
The final message is then prepared, containing all three timestamps $T_{TX_1}^A$, $T_{RX}^A$ and $T_{TX_2}^A$.
The message is loaded into the message buffer of the DWM3000 and a delayed transmission is started.
The delayed tranmission takes the timestamp $T_{TX_2}^A$ and will start the transmission once that timestamp is reached.
Afterwards all caches are cleaned and the tag returns to its previous state, listening for requests.

The tag that performs the ranging roccesponds to device B in figure \ref{f:ds_twr_3}.
Once it has sent the the prepare-rainging message to its neighbour, it will enter the revceiving part of the two-way ranging module.
As device A, device B will also start by settings its antenna delays $d_{rx}$ and $d_{tx}$ and clear all its RX and TX buffers.
It will then start polling for a message.
Once a message from device A is received and validated, it will retreive the timestamp when the message was received, $T_{RX_1}^B$.
Device B will add a constant $c_B$ to this timestamp to get $T_{TX}^B$:
\begin{equation}
	\mbox{$T_{TX}^B$=}
	\mbox{$T_{RX_1}^B+c_B$}
\end{equation}
It will then start a delayed transmission for the response message at $T_{TX}^B$.
The response is a constant string without any data.
Once the response is sent, device B starts to listen for messages again.
When the final message is received from device A and validated, $T_{TX_1}^A$, $T_{RX}^A$ and $T_{TX_2}^A$ are extracted from the message.
Device B also retreives its final timestamp, $T_{RX_2}^B$.
Once this is done, the time of flight for a single message can be calculated, and from that the distance:
\begin{align}
    T_{round1} &= (T_{RX}^A - T_{TX_1}^A) \\
    T_{round2} &= (T_{RX_2}^B - T_{TX}^B) \\
    T_{reply1} &= (T_{TX}^B - T_{RX_1}^B) \\
    T_{reply2} &= (T_{TX_2}^A - T_{RX}^A) \\
    ToF^{AB} &= \frac{(T_{round1}\cdot T_{round2}) - (T_{reply1}\cdot T_{reply2})}{T_{round1} + T_{round2}) + (T_{reply1} + T_{reply2}} \\
    distance &= ToF^{AB} \cdot c_{air}
\end{align}
The distance is then returned, all caches cleared and the module continues with the next distance measurent, if any are remaining.


The TX and RX antenna delay $d_{rx}$, $d_{tx}$ are different for each device.
Qorvo supplies a default value, but it is the same on all devices.
Since te antenna delays are multiplied with the speed of light, even small mistakes in calibration can lead to big errors.
According to qorvo, without the calibration of antenna delays, a measurement can be off by up to 40 cm \cite{DWM3000Calib}.
This will be a constant bias and not change over measurements.\\
Qorov has published a manual on how to calibrate their devices \cite{DWM3000Calib}.
They have not published a codebasis that implements this process.
The calibration process published by Qorvo required things that were not part of this project:
\begin{itemize}
	\item A synchronized clock, shared over all devices, without significant clockdrift
	\item A UART connection to a computer
	\item A pipeline performing statistical analysis and coordinating the devices.
\end{itemize} 
Since implementing this calibration process whould have been out of scope for this thesis, a simpler version was designed.
The tags were set up in a theathedron, so each tag was 30 cm apart from each other.
Then one tag would perform two way ranging with another tag, chosen at random.
The result would be shared between both tags.
If the result was larger than 30 cm, $d_{rx}$ or $d_{tx}$ would be chosen at random and increased.
If it was lower, $d_{rx}$ or $d_{tx}$ would be increased.
Then the second tag would start a new ranging session with a random tag.
This system was left running for over one hour, until all distances measured were in the range of [27 cm, 33 cm].


\subsection{BLE}
\label{ss:ble_module}
The BLE module is responsible for the communication between the UWB network and the phone.
It advertises the tag to the phone and receives messages from the phone and sends messages to the phone using BLE.
The nRF52840 microcontoler is equiped with a antenna with BLE capabilities.
The nRF52 SDK includes libraries for the managment of this antenna.
It also includes the \textit{ble{\_}app{\_}uart} example project.
This project offers advertises a ble connection, handles the paring process.
Once connected, it forwards all incomming comunication to a USB-UART mdoule connected to a computer.
Input from the computer ver USB-UART is sent as a message to the paired device.
The  \textit{ble{\_}app{\_}uart} example project was took as a basis to build the BLE-module.


The nRF52 SDK for BLE requires the use of the S140 SoftDevice.
The S140 SoftDevice is a BLE protocol stack that can be used for the 811, 820, 833 and 840 series of nRF52 boards.
In order for the SoftDevice to be avaliable, a memory 156 kilobyte segment of memory has to be reserved for it, starting at  memory segment 0x0.
The SoftDevice then has to be flashed to the board.

During startup, the BLE module has to initialize a few services and reserve some resources.
Firstly a nRF clock has to be reserved for the BLE module.
Then the powermanagment for the SoftDevice has to be initiated, before the BLE stack inside the SoftDevice can be initialized.
Next the Generic Access Profile (GAP) and the Generic Attribute Profile (GATT) have to be prepared.
The information what functions to call when the SoftDevice receives data has to be set, as well as the advertized name, the UUID, timeout durations and what to do on faults.
The advertized name was left unchanged from the \textit{ble{\_}app{\_}uart} example, "Nordic{\_}UART".\\
Once the SoftDevice is initialized and the tag has connecteced to the UWB network, the BLE connection can be advertized.
The avertisement function of the nRF52 SDK was used for this.


The BLE module listens for queries sent from the Phone to the tag using BLE.
To achiev this, a query-handler function was passed to the SoftDevice during initiation.
All incomming messages will be passed to this function by the SoftDevice.
When a query is received, the BLE module interprets the message.
It checks what is beeing queried and transformes it into a job, readable by the Job Handler module.
The BLE module also offers a service to send messages to the phone.
This service uses the nRF52 SDK to load the message into a the SoftDevice and send it to the phone.


\subsection{Job Handler}
\label{ss:job_handler_module}

The job-handler module connects all other module.
It takes job structs (see figure \ref{code:job_struct}, interprets which module is responsible for handeling them and calles the job together with the relevant data.
The job struct consits of a field for the job-type, that tells the job-handler what type of job this is. It also includes fields to store data, that is needed for the job.

\begin{figure}[h]
    \centering
    \begin{lstlisting}[language=c]
    struct job {
  		enum job_types type;
  		uint8_t* data;
  		int length;
};
    \end{lstlisting}
    \caption{Job struct}
	\label{code:job_struct}
\end{figure}

There are 14 total job types.The following list while decribe the meaning of them, as well as how they are handled by the job-handler:
\begin{itemize}
  \item \textbf{search for network}: This job is triggered after setup. The tag is not connected to the network. It will be passed to the Network module without any aditinal data.
  \item \textbf{join network request}: This job commes from the Network module, when it receives a request from another tag to join the network. It will be passed back to the Network module, with the data of the new devices id.
  \item \textbf{set network and address}: This job commes from the Network module. It informs the network connection has been established. The job is handed back to the Network module, with the received message, to be added to the list of neighbours.
  \item \textbf{ble temp hum request}: This job commes from the BLE module, where a query for temperature and humidity has been registered. The requested tag is extracted from the job. If the request is for this tag, the job is handed to the Temperature and Humidity module. Otherwise it is passed to the network module, to be transmitted to the requested tag.
  \item \textbf{temp hum request }: This job commes from the Network module and informs that a request for a temperature and humidity read has been made. It is passed to the Temperature and Humidity module, together with the requesting tags adress.
  \item \textbf{temp hum answer}: This job commes from the Network module and carries the respons to a temperature and humidity request. It is passed to the BLE module, togehter with the measurement, which will be passed to the phone.
  \item \textbf{ble gyro request}: This follows the same logic as "ble temp hum request", but with the gyroscope module.
  \item \textbf{gyro request}:This follows the same logic as "temp hum request", but with the gyroscope module.
  \item \textbf{gyro answer}:This follows the same logic as "temp hum answer"
  \item \textbf{ble distance request}: This job comes from the BLE module. The phone has queried for a distance. If the queried tag is not this tag, the message is passed to the Network module. Otherwise, it is passed to the Two-Way Ranging module.
  \item \textbf{distances reques}: This job comes from the Network Module. It requests a distance measurement. The job is passed to the Two-Way Ranging module, together with the requeting tags adress.
  \item \textbf{distances prepare}: This job comes from the Network Module. It informs, that another tag is requesting a ranging session. If the ranging session is with this tag, the job is passed to the Two-Way Ranging module. Otherwise the tag goes to sleep for a short time.
  \item \textbf{distances answer}: This job comes from the Network module. It reports that a distance measurement as been returned. The job is handed ober to the BLE module, together with the content of the message.
  \item \textbf{ble get known devices}: This job comes from the BLE module. It requests a list of all neighbours. The job is transfered to the Network-Module.
\end{itemize}


\subsection{Combining modules}
\label{ss:combination}
Each module except for the job-handler module was developed in seperate projects, to ensure operability.
Afterwards the modules were merged into one project.
The Network module was chosen as the base project, that the other projects were merged into.
This was chosen since the Network module was based on \cite{degkwitz2023ultrawideband}, which intern was based on a example published by Qorvo.
The Qorvo example uses a lot of shorthand, magic numbers and development shortcut, that are not easely readable to developers outside the firm.
The Network module whas therefore chosen as a basis, since merging it into another project would likely be cumbersome, since parts would easely be forgotten or interact poorly, without the knowledge or udnerstanding of the developer.
Combining the modules came with several chalanges, that described in this section.


The Qorvo example that builds the basis of the Network module uses the pin-mapping PCA10056.
This is the pin mapping for boards that include the NRF52840 board, but not the NRF52840 development board, that this example was made for and is used in this thesis.
The NRF52840 board does not contain the nesecary pins to attack a DWM3000 board to it.
This wrong pin-mapping leads to mistakes that the Qorvo example has to work around.\\
When switching to the correct pin-mapping, PCA10040, the Network mdoule would no longer work, since those work-arounds now introduced mistakes now.
Sice fixing the Qorvo example code would have been cumbersome, it was decided to instead change the other modules that used pins, the Gyroscope module and the Themperature and Humidity module.
The pins for those modules, pin 11, 13, and 13. where hard coded into the modules, instead of using the pin-mapping.


The nRF52 SDK offers a rich selection of tools, such as SPI and TWI communication, clocks, ble capabilities, SoftDevice, UUIDs.
These tools are all enabled or disbaled in the sdk{\_}config file.
Merging in general requires only to enable the tools needed by the merged module.\\
Three mdoules requie a nRF clock,, Two-Way Ranging, Temperature and BLE.
The nRF SDK offers  exactly three clcoks slots, so all of them have to be enabled with the apropriate clock type.
Each module has to be adapted, so it uses its assigned clock-slot. \\
The nRF52 SDK can suport up to three SPI or TWI conenctions simultaniosly, nemaed SPI0, SPI1, SPI2, TWI1,TWI2 and TWI3.
SPI and TWI share their memory, so SPI0 can not be used while TWI0 is used and vise-versa.
Since the DWM3000 uses two SPI connections and the MPU6050 uses one TWI connection, exactly enough resources remain, for both devices to run simultanisously.
SPI0 and SPI1 were used for the DWM3000 and TWI3 for the MPU6050.\\
All other SDK resources were non-conflicting.
They were ported from the original module implementation to the merged one without change.


As most embeded systems do, the nRF52840 requires static memory allocation during flashing.
The avaliable memory is seperated into flash-memory and random-access-memory (RAM).
Some memory segments are required by every runable system:
\begin{itemize}
	\item FLASH, \textbf{vectors}: The interrupt vector table defines the interrput handlers for the system, like resets, faults.
	\item FLASH, \textbf{init}: The initialization routine sets up clocks, pins and other peripherals.
	\item FLASH, \textbf{text}: This section contains the executable code in mashine language.
	\item FLASH, \textbf{data}: This section contains the initial values for all global values..
	\item \textbf{rodata}: This section contains the constant variables, that will not change at runtime.
	\item RAM, \textbf{data}: During startup, the initial values for changable global variables are coppied to this section. They can change at runtime.
	\item RAM, \textbf{bss}: This section contains the global variables that do not have initial values.
	\item RAM, \textbf{stack} and \textbf{heap}: The stack and heap that build the runntime environment.
\end{itemize}
Neither the MPU6050 nor the DHT22 require any additional memory segments.
The DWM3000 and the BLE module both require additional memory segments. \\
The BLE module reuqires the SoftDevice to be added to memory. The Softdevice requires 156 KB of Flash and 10.7 KB of RAM.Those reserved memory segments need to be the first one in both Flash and RAM. This additionaly requires SoftDevice oberservers for System on Chip (SoC), BLE, state and stack. Additionaly a segment to house the nRF52 SDK memory allocator is required, nrf{\_}balloc. These segments are rather small, never exceeding 32 bytes.\\
The DWM3000 shield requires two additional memory segments, fConfig in Flash and nrf{\_}balloc in RAM. 
Qorvo does not publish what the fConfig module is for, but it is required for the shield to work. \\
Since the base project was made for the DWM3000 shield, it had to be addapted to addinionaly fit the segments needed for the BLE module. This mainly consisted of moving all segments to later adress-spaces to add room for the SoftDevice reserved memory. All other memory segments had to be added as well. To make room for this, the Flash memory had to be expanded. \\
The Qorvo example implementation for the DWM3000 shield uses some work-arrounds. An example of this is the "NRFX{\_}SPIM3{\_}NRF52840{\_}ANOMALY{\_}198{\_}WORKAROUND{\_}ENABLED" present in the SDF configuration. These workarounds let the SPI communication with the shield perform certen memory manipulation. If these workarounds are necessary is doubtfull, but fixing them would have been out of scope for this thesis.
The workarounds do generaly have no effect on the implementation, with one excpetion. When the DWM3000 receiver sends a message longer than 10 bytes to the microcontroler over SPI, it incroaches on the SoftDevice RAM. This behaviour was found experimentaly, the responsible code could not be located. Since the system can be implemented with the restriction of 10 byte messages, this was done.



\section{App}
\label{s:app}
Nordic Semi Conductors, the maker of the used microcontrolers, published the code to a simple app that allows for BLE communication with their devices.
It is called nRF Toolbox.
It is intended to pair with the \textit{ble{\_}app{\_}uart} example, published in the nRF52 SDK.
Since this example code was used as the basis for the ble communication used in this project, it was addapted to work with this project.


The App contains different modules, intended for different examples, among them the Universal Asynchronus Receiver/Transmitter (UART) module (see \ref{f:Toolbox_modules}).
It is intended to be used with the ble\_ app\_ uart example.
When opem it shows the ble services that are currently beeing addvertised and allows the user to connect to one of them \ref{f:Toolbox_connect}.
It then opens a window similar to phone messangers, were the keyboard can be used to tpye messages, that are sent to the connected devices.

\begin{figure}[ht!]
\centering
\includegraphics[width=200px]{graphics/nRF_toolbox_modules.jpg}
\caption{nRF Toolbox module menue, with the added Art Tracking Module}
\label{f:Toolbox_modules}
\end{figure}

\begin{figure}[ht!]
\centering
\includegraphics[width=200px]{graphics/nRF_toolbox_connect.jpg}
 \caption{nRF Toolbox shows avaliable devices to connect to}
\label{f:Toolbox_connect}
\end{figure}

\begin{figure}[ht!]
\centering
\includegraphics[width=200px]{graphics/nRF_toolbox_messanger.jpg}
\caption{nRF Toolbox UART module screen}
\label{f:Toolbox_output}
\end{figure}

Since the development of an application was not the primary focus of this thesis, it was decided to take the nRF Toolbox app and add a new module for art-traking to it.
The UART module searved as the basis for this new module, since it had a lot of usefull services already implemented.
As with the UART module the art-tracking module opens up the same connection page \ref{f:Toolbox_connect}, that allows the user to select the art-tracking and connect to it.

Once connected, the observation screen is shown (figure \ref{f:Toolbox_art_tracking_empty}).
At the bottom seven parameters can be set: \textit{time}, \textit{max Temp}, \textit{min Temp}, \textit{max Hum}, \textit{min Hum}, \textit{max Angle}, \textit{max Dist}.
The parameters \textit{max}/\textit{min} \textit{Temp}/\textit{Hum} represent the expected range of humidity and temperature.
Any measurement outside these parameter will be considered a dangerous value by the app.
The tollerated difference in angle compared to the previous measurement is set by \textit{max Angle}, larger differences are considered dangerous values.
Distance measurement work analogously with \textit{max Dist} in meters.
The \textit{time} set defines the time that passes enbetween measurements in seconds.
The default is set to 350 seconds.
This means that the time that passes between, for example, the temperature measurements on tag 2 are 350 seconds.


\begin{figure}[ht!]
\centering
\includegraphics[width=200px]{graphics/nRF_toolbox_art_tracking_empty.jpg}
\caption{Art Tracking module oberservation screen before measurements}
\label{f:Toolbox_art_tracking_empty}
\end{figure}


When the user presses the \textit{Start Service} button, a services starts that poeriodically queries the tags for the Measurements.
Figure \ref{code:App_main_loop} shows the measurement loop.
Each sensor is assigned a character.
\textit{T} for temperature and humidity, \textit{G} for gyro and \textit{D} for distance.
Each tag has a number, here from one to four since four tags were used in the experiments.
The loop concatenates these two characters and sends the resulting query to the connected tag.
Then the next tag-number is prepared for the next query.
Once all tags have been queried for a sensor, the tag-number starts with the first again and the next sensor is queried.
In between cals the app waits.
The call time for distance-measurement is fixed at 80 seconds.
Distance measurement takes longer than the other sensors, since for every devices three measurements need tobe conducted.
Additionaly the sensors that do not participate in a ranging session are sleeping for a quite generous amount of time, to ensure they don't distrub the ranging session.
80 seconds has been chosen, since it allows enough time for all the ranging to happen, plus two repeats per sensor in case the ranging session fails.
For the other sensors the waiting time in between queries is calculated from the remaining set time, after the ranging time is deducted.

\begin{figure}[h]
    \centering
    \begin{lstlisting}[language=Java]
    private val sensors = listOf("T", "G", "D")
    private val devices = listOf("1", "2", "3", "4")
    private var measurement_type = 0
    private var tag = 0
    private var timeBetweenCals: Long = 3750

    private val runnable = object : Runnable {
        override fun run() {
            if (tag >= list2.size) {
                tag = 0
                measurement_type += 1
            }
            if (measurement_type >= list1.size) {
                measurement_type = 0
            }
            val textToSend = "${list1[measurement_type]}${list2[tag]}"
            artRepository.sendText(textToSend, MacroEol.LF)
            tag += 1
            if(list1[measurement_type] == "D"){
                handler.postDelayed(this, 80000)
            } else {
                handler.postDelayed(this, timeBetweenCals)
            }
        }
    }
    \end{lstlisting}
    % Optionally, add a caption to the figure
    \caption{Section from the ArtMetricService.kt, main measurement loop}
	\label{code:App_main_loop}
\end{figure}

Once the process has started, the queries will appear in the chat window on the right side of the screen.
The responses are on the right side, see figure \ref{code:App_main_loop}.
If the response is inside the set parameters, the message bubble will appear blue.
If the measured value is considered a dangerous value, the text bubble will appear red (see figure \ref{f:Toolbox_art_filled}).
Since the message display is programmed in a asynchronus way, it can happen, that the answer to a query appears before the querry itself, if the querried tag is the same as the connected tag.
The service can be stopped by pressing the \textit{start service} button again or by exiting this screen in any way.


\begin{figure}[ht!]
\centering
\includegraphics[width=200px]{graphics/nRF_toolbox_Bad_Value_2.jpg}
\caption{Art Tracking module, queries and responses}
\label{f:Toolbox_art_filled}
\end{figure}


%TODO: add image that shows an example file
The query-answers are appended to a file that is safed in the app-storage.
The information appendded consits of: the queried tag, the returned values, a timestamp and if the value was unproblematic.
This functionality is intended for experimental evaluation. 
In a real word application, this data should be periodically backed up on a server in a compressed manner.
When pressing the share-button on the top right of the message-box \ref{f:Toolbox_art_filled}.
It will open the Android naitive share functionality, to share the file over mail, an installed messanger, save it to onedrive or send it over Bluetooth.
In this project all files were sent with email.
Pressing the trashcan next to it will delete the chat and empty the file.
This allows the user to distinguish between different testing session.
%\chapter{Evaluation}
\label{chap:evaluation}



\section{Experiments}
\label{s:Experiments}
Five experiments were performed to validate the functionality of the tags.
The first two are non specific and ment to test the setup in a stable environment.
Experiments three to five are intended to test the detection of unwanted circumstances.
For all experiments the query-frequency was set to 330s, so measurment was evaluated once every 330s.
The measurements queries are spread across this timeframe.
Each experiment lasted between 40 minutes and one hour.
The results were stored on the phone and then exported using email.
The analysis of the data and creation of graphs was then performed using a Jupiter Notebook, using Pandas and Pyplot for datamanagment and the creation of graphs.


\subsection{Experiment 1: Static}
\label{ss:exp_1}
The four tags where placed on the corners of a 80 cm by 50 cm rectangle on a wooden table.
Each tag was turned on sequentially and given enough time to establish the network.
The phone then was connected to one tag.
The parameters in the app were left unchanged.
The default parameters are large enough, that no measurement should be large enough to trigger a warning.
The setup was then left untouched for 35 min.
The goal of this experiment was, to gauge by how much the measurements can vary in a static environment.
 

\subsection{Experiment 2: Coordinated movement}
\label{ss:exp_2}

\subsection{Experiment 3: Temperature}
\label{ss:exp_3}
The four tags were placed in the same 80 cm by 50 cm rectangle as in experiment one.
One tag placed on a elevated surface, 4 cm above the table.
Under the tag seven candles were placed (see figure TODO, Bild einfügen).
Next to the tag two thermometers detectors were placed.
Each tag was turned on sequentially and given enough time to establish the network.
The phone then was connected to one tag.
The max Temperature parameter in the app was changed to 35°C.
After 20 minutes the candles were lit.
The experiment was then left alone for another 30 minutes.
The independent thermometers were filmed during the process, to allow for later review and comparesment.
The goal of experiment 3 was to test the temperature detection capabilities of the system.


\subsection{Experiment 4: Gyroscope}
\label{ss:exp_4_1}
Again all for tags were placed on a 80 cm by 50 cm rectangle.
Each tag was turned on sequentially and given enough time to establish the network.
The phone then was connected to one tag.
The maximal allowed angular difference was set to 30°.
After 20 minutes one tag was turned by 90° counterclockwise.
The experiment then ran for another 30 minutes.
The goal of experiment number 4 was to test the detection of unwanted rotations.
Experiment four was repeated with, with the gyro sending the angular velocity for all axes instead of the current position.


\subsection{Experiment 5: Distance}
\label{ss:exp_5}
The same 80 cm by 50 cm rectangle setup was used.
The tags were turned on sequentilay, giving them enough time to build the network.
The phone was connected to one tag.
The max distance parameter was set to 20 centimeter.
After 20 minutes, one tag was moved parallel to the shorter rectangle line about 20 cm towards the tag on the next corner.
The system was then left resting for another 30 minutes.
The goal of experiment number 5 was to test the detection of unwanted movement.

\section{Experiment Results}
\label{s:exp_res}

In this section the results of the experiments are presented.
All eperiments were performed two to three times.
In each section only the data-set from the first experiment run is presented fully.
The other experiments will be mentioned only, if they have differing data or to confrim an unexpected datapoint.

\subsection{Experiment 1: Static}
\label{ss:exp_1_result}
In eperiment one, all measurements are expected to be unchanging.
Table \ref{t:exp1_means} shows the mean values for temperature, humidity and angle during the experiment by tag.
Figures \ref{f:exp1_graphs_temp}, \ref{f:exp1_graphs_hum}, \ref{f:exp1_graphs_gyro}, \ref{f:exp1_graphs_dist} shows the change of these values over time.

\begin{table}[h!]
    \centering
    \begin{tabular}{|c|c|c|}
        \hline
        \textbf{Tag} & \textbf{Temp Mean} & \textbf{Hum Mean} \\
        \hline
        1 & 22.06 & 32.56 \\
        2 & 21.90 & 33.93 \\
        3 & 22.06 & 32.94 \\
        4 & 21.87 & 32.80 \\
        \hline
    \end{tabular}
    \caption{Mean and Variances for Temperature, Humidity, and Gyroscope Data by Tag during experiment 1}
    \label{t:exp1_means}
\end{table}

\begin{figure}[ht!]
\includegraphics[width=\linewidth]{graphics/exp/exp1_temp_plot_0.png}
 \caption{Experiment 1, temperature over time.}
\label{f:exp1_graphs_temp}
\end{figure}


All four tags have a similar mean temperature and are all less than 0.2 °C apart from each other.
The varaince are also small, tag two having the highest one with 0.05 °C variance. 
The graph shows that all tags have a rising temperature.
The increase is quite small with tag two having the biggest increase of 0.5 °C over 20 minutes.
When the experiment was repeated, the means stayed similar and the variance small, but the temperature changed course.
A downward trend was visible, instead of the upward trend seen during the first experiment.

\begin{figure}[ht!]
\includegraphics[width=\linewidth]{graphics/exp/exp1_hum_plot_0.png}
 \caption{Experiment 1, humidty over time.}
\label{f:exp1_graphs_hum}
\end{figure}

Humidity follows a similar trajectory as.
The means only vary by 1.5 \% pt.
The variance is small, with tag three having the biggest variance with 0.06\% pt.
During the first experiment, humidity increased by a small amount.
When the experiment was repeated, the humidity dropped during the experiment.


\begin{figure}[ht!]
\includegraphics[width=\linewidth]{graphics/exp/exp1_gyro_data_plot_0.png}
 \caption{Experiment 1, angles over time.}
\label{f:exp1_graphs_gyro}
\end{figure}

Since all tags were stationary during the experiment, the gyro sensor was expected to be unchanging.
This is not what happened.
looking at the graph \ref{f:exp1_graphs_gyro} it is clear, that the measurement shows a wide range of angles for each tag and axis.
The only exception is tag 2 around the x axis, which stays at 0 for the whole measurement duration.
Since angle measurements fall into modular arithmetic, it "wrappes around" at 360°, means can only meanigfully be taken if the angles are in a small range.
Since this is not the case for most tags, the only thing that can be said is, that tag 2 has a mean of 0 with variance 0 around axis x.


\begin{figure}[ht!]
\includegraphics[width=\linewidth]{graphics/exp/exp1_dist_data_plot_0.png}
 \caption{Experiment 1, distance over time.}
\label{f:exp1_graphs_dist}
\end{figure}

Table \ref{t:exp1_means}, shows the mean of the measured distances, the row entry beeing the queried tag that initiates the distance measurement, and the row corresponding to the responding tag.
By looking to the measurements diagonaly oposed to each other, one can see that the measured distaances is the the same, indipendent of who initiated the measurement, up to a range of two centimeters.
The varince on table \ref{t:exp1_dist_var} also show, that these measurements are stable over time and don't change by much.
Looking at the graphs on figure \ref{f:exp1_graphs_dist}, the pairs of measurements are visible.
One outlier happens when tag 3 measures the distance to tag 1 at very end of the measurements.
When repearing the measurements these outliers happened again, a bit less frequently then twice per hour.
The outliers always affected a measurement involving tag 1.
The distances measured do not correspond to the actual distances the tags had to each other, also seen in table \ref{t:exp1_dist_means}.
The measured distances can be as far of as 0.5 meters.
The two larger distances, 0.8 and 0.94 meters, correspond to the two larger measured values for each tag, while the smallest measured value always corresponds to the smallest distance, 0.5 meters.
The two larger values are not always ordered correctly, 0.94 meters sometimes beeing measured smaller then 0.8 meters.
In repeated experiments, all these facts stayed true.

\begin{table}[h!]
    \centering
    \begin{tabular}{|c|c c c c|}
        \hline
          & \textbf{1} & \textbf{2} & \textbf{3} & \textbf{4} \\
        \hline
		\textbf{1} & 0.0 & 1.094 & 1.084 & 0.657 \\
		\textbf{2} & 1.080 & 0.0 & 0.356 & 0.989 \\
		\textbf{3} & 1.007 & 0.367 & 0.0 & 1.279 \\
		\textbf{4} & 0.666 & 0.987 & 1.281 & 0.0 \\
        \hline
    \end{tabular}
\begin{tabular}{|c|c c c c|}
        \hline
          & \textbf{1} & \textbf{2} & \textbf{3} & \textbf{4} \\
        \hline
		\textbf{1} & 0.0 & 0.8 & 0.94 & 0.5 \\
		\textbf{2} & 0.8 & 0.0 & 0.5 & 0.94 \\
		\textbf{3} & 0.94 & 0.5 & 0.0 & 0.8 \\
		\textbf{4} & 0.5 & 0.94 & 0.8 & 0.0 \\
        \hline
\end{tabular}
    \caption{Left: Mean distances between tags in experiment 1. Right: Expected values.}
    \label{t:exp1_dist_means}
\end{table}

\begin{table}[h!]
    \centering
    \begin{tabular}{|c|c c c c|}
        \hline
          & \textbf{1} & \textbf{2} & \textbf{3} & \textbf{4} \\
        \hline
		\textbf{1} & 0.0 & 0.002 & 0.000 & 0.000 \\
		\textbf{2} & 0.001 & 0.0 & 0.000 & 0.001 \\
		\textbf{3} & 0.024 & 0.001 & 0.0 & 0.000 \\
		\textbf{4} & 0.000 & 0.000 & 0.000 & 0.0 \\
        \hline
    \end{tabular}
    \caption{Variance of distances between tags in experiment 1. Row corresponds to queried tag.}
    \label{t:exp1_dist_var}
\end{table}

\subsection{Experiment 3: Temperature}
\label{ss:exp_3_result}
Experiment 3 introduced heat-sources the system.
Since the main setup was the same as experiment 1 \ref{ss:exp_1_result}, many of the findings are the same.
In this section, only differences in results are discussed.
If a metric is not measioned, one can assume it behaved the same as for experiment 1  (see section \ref{ss:exp_1_result}).

\begin{figure}[ht!]
\includegraphics[width=\linewidth]{graphics/exp/exp3_temp_plot_1.png}
 \caption{Experiment 3, temperature over time, mith external measurement added.}
\label{f:exp3_graphs_temp}
\end{figure}

The progression of the external thermoeter and the internal temperature sensor can be seen in figure \ref{f:exp3_graphs}.
The candles, that functioned as the heat source, were lit at 15.10.
During the next measurement of tag 2, at 15.12, both the external thermoeter and the temperature sensor on tag 2 had not yet registered any change, remaing at 22.4 °C.
The extrenal thermometer started rising 1 minutes later, at 15.13.
During the next measurement at 15.18, the temperature-sensor registered a slightly increased temperature of 23.6 °C, while the external thermometer registered 24.7 °C.
During the next measurement at 17.24 the tag reported 26.3 °C while the thermometer showed 27.1 °C. 
The measured temperature of the external thermometer keeps klimbing faster than the internal temperature sensor, until the end of the experiment, as seen in Figure \ref{f:exp3_graphs_temp}.
Their distance nether exeeds 1 °C and gets smaller towards the end of the experiment.
The other tags do not report any significant change in temperature.

\begin{figure}[ht!]
\includegraphics[width=\linewidth]{graphics/exp/exp3_hum_plot_1.png}
 \caption{Experiment 3, humidity over time, with external measurement added.}
\label{f:exp3_graphs_hum}
\end{figure}


Experiment 3 was intended to test the temperature and not the humidity.
Luckily, the external thermoeter also included a humidity sensor, that could retroactivly be used for evaluation.
Figure \ref{f:exp3_graphs_hum} shows the humidity over time, with a added humidity sensor added to the graph.
Since the external humidity sensor was initialy not intended to be used, it is not perticulalry precise and does not display any digits after the decimal point.
The humidity sensor consitently shows a much higher humidity than the one on the tag.
Once the experiment starts at 15.10, the humidity behaves inversly to the temperature and starts falling.
This happens with the external sensor as well as the internal one in parallel.
The registered values plato at 34\% for the external and 26\% for the internal sensor.
The other tags do not report any significant change in humidity.

\subsection{Experiment 4: Gyroscope}
\label{ss:exp_4_result}

Experiment 4 was intended to check the functionality of the gyroscope.
Temperature and humidity behaviour was the same as in the static experiment \ref{ss:exp_1_result}.
As already seen in during the evaluation of experiment 1, the gyroscope does not work as planned.
Figure \ref{f:exp4_graphs_gyro} shows the values of the gyro over time.
Tag 1 was rotated by 90° at 22.25 around the Z axis.
Their is no disernable change in the output of the gyro during or after this process.

\begin{figure}[ht!]
\includegraphics[width=\linewidth]{graphics/exp/exp4_gyro_data_plot_1.png}
 \caption{Experiment 4, gyroscope over time.}
\label{f:exp4_graphs_gyro}
\end{figure}


Figure \ref{f:exp4_graphs_dist} shows the distances of the tags during the experiment.
Before the event, all tags are in a stable state.
As in experiment 1 \ref{ss:exp_1_result} the distances do not represent what is physically happening.
After the tag is turned at 22.26, all measurements involving tag 1 change, and becomming stable again afterwards.
This can be bit hard to see, since "2-1" and "3-1" have an outlier measurement right before and "1-2" right after.
Distance 1 to 2 and 1 to 3 changes between 0.2 and 0.3 meters and distance 1 to 4 changes by arround 0.4 dm.

\begin{figure}[ht!]
\includegraphics[width=\linewidth]{graphics/exp/exp4_dist_data_plot_1.png}
 \caption{Experiment 4, gyroscope over time.}
\label{f:exp4_graphs_dist}
\end{figure}


\subsection{Experiment 5: Distance}
\label{ss:exp_3_result}

Experiment 5 was intended to test the distance measurement capabilities of the setup.
Temperature and humidity and gyro behave as they do in experiment 1 \ref{ss:exp_1_result}.
They will not be discussed for this experiment.

\begin{figure}[ht!]
\includegraphics[width=\linewidth]{graphics/exp/exp5_dist_data_plot_2.png}
 \caption{Experiment 5, distance over time.}
\label{f:exp5_graphs_dist}
\end{figure}

Figure \ref{f:exp5_graphs_dist} shows the measured distances of the 4 tags over time.
As in the static experiment, the measured distances of two devices are similar and mostly stable, before any movement is introduced.
As in experiment 1 the values reported are not correct.
At 14.24 tag 1 is moved by 0.23 meters toward tag 2.
The measured distances to tag 3 increases while the distance to tags 2 and 4 dicreases.
This represent what is happening in reality, since tag 1 is now closer to tag 2 and 4 and further away from tag 3 as before.
The difference in distance is roughly 0.2 meters for tag 2.
This is correct, since tag 1 was moved about that distance towards tag 2.
The measurements show tag 4 now 0.15 meters closer to tag 1.
The effect on tag 4 should be notissable but not as large as it is.
Since the tag moves lateraly towards tag 4, the difference should only be 0.11 meters.
The same is true for tag 3.
The difference in measured distance between 1 and 3 is between 0.15 and 0.2 meters. 
This is too large for the difference a latteral move, it should only be a 0.02 meters difference.
Their is also a small increase in the distance between tags 2 and 3, but which starts before tag 1 was moved.




%\chapter{Final Considerations}
\label{chap:considerations}

%Regarding Final Considerations:
% I. Some teachers and/or methods use the term Conclusion for a text at the end of the paper that aims to expose the results achieved, this term is not incorrect, but many of the works are bibliographical reviews where in the end no conclusion is obtained and yes Several considerations that were found in the development of written work. Therefore, in each project should be considered/weighted if there will be a Conclusion or Final Considerations. Usually what else happens is that you have Final Considerations. The Final Considerations of a paper aims to show if the goal sought for the project was achieved, as well as give a view of the most important considerations and conclusions on the subject addressed, among other aspects. This should include:
%1. An explanation stating clearly whether or not it has achieved the stated objectives (a subdivision between general objectives and specific objectives can also be made here). In each case the reasons must be explained:
%The. If you have achieved the objectives: inform the main factors that contributed to the success, describing them briefly, but do not leave doubts;
%B. If you have not met the objectives: inform how much of the objective has been achieved and cite the factors that contributed to the failure, describing them briefly, but that leaves no doubt.
%2. Describe the main considerations and conclusions that were obtained as a result of the execution of the work. Here should not be repeated text already in the work, but write the impressions of these considerations and how they contributed to the implementation and achieved the goal;
%3. Name and describe the main difficulties encountered in the execution of the work and project. All the work developed means an evolution for the student, and to reach this evolution, it has had to overcome a series of obstacles. Reporting obstacles and overcoming (or not overcoming) helps to dignify and show the merit of the work itself to the reader/evaluator. It is also a contribution, in the sense that once problems and solutions are exposed, readers/evaluators learn/know ways of solving or approaching such problems;
%4. Discuss whether modifications occurred during the execution of the work within the scope defined in the Project phase and in what was developed. It should be explained what generated those modifications, substantiating and justifying such changes.
%5. The relationship between the proposed schedule and the work schedule can be described. This allows the reader/evaluator to learn from the indicated distortions/hits.
%6. Describe or cite future work that may be done based on this work. During the execution of work, it is sought to reach a defined objective in the project. However, several interesting subjects of research are revealed (being that the same ones are not treated/researched in the work because they do not match the objective and scope of the work). The description of such subjects/themes/research demonstrates the students' perception of development as well as their vision of objectivity in the execution of this work.

%II. Conclusion / Final Considerations aim to show the reader/evaluator the student's perception of the work done. In this way, it is not advisable to do citations and references because theoretically everything that was necessary to quote and refer should already be done within the content of the work. Only in some very specific cases/situations can you make referrals or citations in this part of the work (this should be discussed thoroughly with the supervisor/teacher).

%III. Looking at the described items that should compose the Conclusion / Final Considerations, it is difficult to imagine that this part of the work has less than one page;

\section{Summary}

% I did this in this way, that in that way... and so on

\section{Conclusions}

% Lessons learned

\section{Future Work}
