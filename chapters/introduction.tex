\chapter{Introduction}

Research questions:
\begin{itemize}
  \item What are the advantages and disadvantages to using UWB over BLE for fast network communication.
  \item Can I build a system of sensors that detectes disturbances in transportation and alerts the user.
  \item What is the most adequate network topology to be used in such a system.
\end{itemize}

\section{Motivation}

Certify is an international cooperative project between twelve partners situated in Switzerland and the EU %\citep{Brüschre von certify}.
One of those partners is the university of zurich.
Its focus is on the development of Internet of Things (IoT) systems for security, monitoring and detection %\citep{Brüschüre von certify}.
Next to certifications and the development of frameworks, Certify also consirnes itself with the integration of IoT devices.


One of multiple currently running pilots of Certify is the "Tracking and monitoring of artworks" %\citep{https://certify-project.eu/pilots/}.
The goal of this pilot is to enable the constant tracking and monitoring of artworks by attaching a device to it.
This device allows for unique identification, by using cryptographic methods.
It is also intended to act as a cluster of sensors that collect information about the surounding of the artwork that are relevant to the wellbeeing of the artwork.
The goal is to have constant data on the artwork throughout its lifecycle.
This is intended to help with securing the artwork and helping with chain of custody monitoring.



\section{Thesis Goals}

The goal of this project is to develop a system that implements a localized version of the artwork tracking envisioned by the Certify project, ment for transportation in a truck.
Additionally the system will extend the Certify Projects goal by adding new detection methods and also informs the driver of the truck about potential problems.

The goal is to develop a system that tracks the state of artwork in a truck using different detection methods.
The devices attached to the artworks, called tags, build a local decentralized network.
The phone of the driver can query the network and displayed the collected metrics to them.
If a metric is outside of the accepted norm, the system should allert the driver.

This thesis presents a proof of concept implementation.
The used metrics are not intended to be a full representation of needed sensors to securley Transport art.
Rather they are intended to show different types of sensors that can be used.
This thesis assumes that the data transfer to a server using 4G, as planned by the Certify project, will work and will not implement it in this thesis.



\section{Methodology}

This Thesis was made in four stages:

\subsubsection{Reserach}
In a first step, the basis of the thesis had to be researched.
This involved familiarizing with existing research on the toping of artwork tracking, local IoT networks and commonly used communication protocols.
Existing artwork tracking methods need to be analyzed and evaluated, considering their strength and shortcommings during the transportation in a truck.
The types of sensors that could be relevant need to be chosen, based on existing research, cost and avaliablity.
Options for the network-architecture inside the truck needed to be researched and compared, based on performance, stability and security.
A communication protocol needed to be chosen, based on the same criterias.

\subsubsection{Design}
Once the fundamental knowledge for the project had been aquired, the system had to be designed.
The design was chosen based on feasability, security and stability.

\subsubsection{Implementation}
The desig then was implemented in a simplified manner based on the material that was avaliable.
For this four tags were build, equiped with sensors, communication-capabilities and power suply.
Then the required software was written, using existing implementations when possible and writing new code when required.
A simple example app was also developed, based on an existing communications app published by Nordic Semiconductors and installed on a phone.

\subsubsection{Evaluation}
The developed system of tags and phone was tested in a series of five experiments.
The first four experiments were intended to caputre a specific part of the system, while the last was a general purpose test.
The tests were performed in a manner that insured minimal external influence.
The resulting data from the testswere analyzed using statistical methods.
The goal was to determin the reliability of the system, find limitations and look for emprovements.

\section{Thesis Outline}


