\chapter{Fundamentals}

In this chapter, the fundamental knowledge that was researched during this thesis is presented.
Section \ref{s:background} introduces the background knowledge required for this thesis and Section.
Section \ref{s:related_works} presents the current state of research on topics surrounding this thesis.

\section{Background}
\label{s:background}

This Section describes the theoretical background used in this thesis.
It covers key aspects, such as communication protocols, two-way ranging, sensor networks, and required graph theory.


\subsection{Wirerless Sensor Networks (WSN)}
\label{S:WSN}
Kevin Ashton coined the term Internet of Things (IoT) in 1999, although the idea predates this term and was before known as embedded internet or pervasive computing \cite{alaba2024evolution}.
It describes the ubiquity of digital devices and their seamless integration into the physical world and everyday life.



At the end of the 90s and during the 2000s, the production of embedded systems and sensors, in particular, rose.
This led to falling prices and sensors becoming widespread.
With the new availability of sensors, Wireless Sensor Networks (WSN) became more widespread.
While not originating in during this time, the term itself was coined in 1980, and the research community became more focused on the topic \cite{jindal2018history}.

\cite{jindal2018history} determined four main challenges in the development of WSNs:
\begin{itemize}
\item Self-organization: A large number of nodes should not require manual installation and maintenance
\item Cooperative Processing: Sensor nodes have limited memory. Evaluating, compressing, and transporting the data becomes a major challenge.
\item Energy efficiency: WSNs often operate where no power supply is available. The sensors, therefore, must run on limited, battery-based energy.
\item Modularity: WSNs should work for various applications and sensor node types.
\end{itemize}

The Ad hoc On-Demand Distance Vector Routing (AODV) Protocol was published in 1999 by E. Perkins and E. M. Royer \cite{royer1999multicast}.
It presents a routing algorithm for wireless ad hoc networks, where routing is only established when needed, and devices can be added to or leave the network at will.
Doing this takes the problem of self-organization and modularity into account.
A modified version of AODV is used in Zigbee to this day \cite{mu2017improved}. 

In 2000, Heinzelman et al. published the low Energy Adaptive Clustering Hierarchy (LEACH) algorithm \cite{heinzelman2000energy}.
Leach divides the sensors into clusters based on location.
The clusters communicate with a network head using cluster heads that collect and transmit the cluster's data.
New cluster heads are elected periodically to spread the increased energy drain from the data transfer to the network head.
LEACH is used \cite{sefati2022cluster, ghazy2023low} and improved \cite{bagherzadeh2022survey} to this day.

\subsection{Ultra Wideband}

\subsubsection{IEEE}

The Ultra Wideband (UWB) communication protocol was introduced in 2003 by the Institute of Electrical and Electronics Engineers (IEEE) as part of the IEEE 802.15.4 standard.
In 2020, updates were made to the protocol when the IEEE 802.15.4z-2020 standard made improvements to the PHY layers of UWB connections. 
It achieved this by introducing a more robust timestamping system on the PHY layer.
This is supplemented by changes to the MAC layer that allow for the exchange of ranging information.
The result is short frames that are transmitted fast between devices, leading to short bursts of communications that are fast, secure, and ideal for ranging.


UWB works by using short radio frequency pulses, resulting in a large bandwidth.
UWB is a lower-power communication form.
This prevents it from interfering with other communication forms with which it shares its wavelength, such as WLAN or Bluetooth. 
Since UWB uses very short, distinct pulses over a short range, it has been found to be useful in ranging systems \cite{hsu_2021}. 
UWB is split into high-rate pulse (HRP) UWB and low-rate pulse (LRP) UWB.
Since ranging is part of this work and LRP is generally not used for ranging \cite{hsu_2021}, I will not discuss it further in this Thesis.
Since UWB devices tend to be small and have low energy consumption, in combination with the capability of ranging and data transfer, they have become popular as Internet of Things (IoT) devices. \\ 

The standard defines the PHY and MAC layer as well as frequency bands for communication.
The 4z expansion tries to integrate UWB into the WPAN standard. In Section \ref{sec:UWB MAC} and \ref{ss:UWB_PHY} I will discuss the PHY and MAC layer.\\

The sending devices emits pulses in a pre-set band of frequencies, using short bursts to transmit the bits.
The signal forms a concave curve in this band, where the two points 10 db below the maximum power spectral density are called the lower- and upper-frequency points, see Figure \ref{f:UWB_spectrum}.
These two points must be at least 500 Hz apart.
The maximum power spectral density must be below the noise level.
This process prevents conflicts with other communications, that use a single frequency with a high power spectral density and modulate signal transmission, such as WI-FI or Bluetooth.
The UWB protocol has the added benefit of being useful for high-accuracy localization.


\begin{figure}[ht!]
\centering
\includegraphics[width=\linewidth]{graphics/UWB_spectrum.jpg}
\caption{Power Spectral Density: Bandwidth B, lower-frequency f\textsubscript{L}, upper-frequency f\textsubscript{H}, \cite{hsu_2021}}
\label{f:UWB_spectrum}
\end{figure}


\subsubsection{UWB supported Nodes}

The IEEE 802.15.4 standard distinguishes between two types of devices.
Full-function devices (FFD) are capable of connecting to multiple other devices, receiving, transmitting, and coordinating. Reduced-function device (RFD), on the other hand, can only connect to one other device and act as a worker. 
In Topological terms, RFDs can only operate as leaves, while FFDs can be any node in a network, including leaves.
RFDs, therefore, are strictly worse but make up for it by requiring fewer resources, such as memory and power.
When FFDs work as PAN coordinators, they can use short addresses to address any node.
The PAN also has a PAN identifier to help communication across multiple networks while still using the short address.
Each device also has an extended address that is not assigned by any coordinator and serves as a universal unique identifier (UUID).


\subsection{UWB MAC}
\label{sec:UWB MAC}

The Mac Layer is part of the Data link layer.
The Mac Frame is the payload of the PHY frame. It carries information about the frame type, frame format, security mechanism, addressing, and frame validation.
The Mac Layer additionally provides rules for beacon management, and channel access.

\begin{figure}[ht!]
	\centering
	\includegraphics[width=\linewidth]{graphics/general_MAC_Frame_Format.png}
	\caption{General MAC Frame Format \cite{IEEE4-2020-7}}
	\label{f:MAC Frame Format}
\end{figure}


\subsubsection{MAC Frame Format}

Figure \ref{f:MAC Frame Format} shows the composition of a UWB-MAC frame.

In the MAC header (MHR), the Frame Control Field includes information about:

\begin{itemize}
  \item the frame-type
  \item if the Auxiliary Security Header Field is used and in what capacity
  \item if additional frames will follow
  \item if an acknowledgment message is expected
  \item if the message is between different PAN-Networks.
  \item of what type the receiver is (PAN coordinator, device, PAN-Network)
  \item the used frame-format standard
  \item where to find the source address
\end{itemize}

The Sequence Number counts up, helping to keep track of the order in which frames have arrived.
The Addressing Fields carry the IDs of the sender and recipient for the frame.
The Auxiliary Security Header Field only exists if specified in the control Field.
It contains additional information needed for the chosen security method.

There are two parts to the information element (IE).
The header IE specifies additional information about the frame, for example, data formatting information or channel time allocation.
The payload IE specifies the length and data type of the payload field.
The payload contains the data that is sent.
It and the IE are of variable length, depending on the frame type and data length.

The MAC footer (MFR) marcs the end of the frame.
It only contains the frame checking sequence (FCS) that can be used to detect corrupted frames using cyclic redundancy checks.

\subsection{UWB PHY}
\label{ss:UWB_PHY}

\subsubsection{PHY Chanel}

The IEEE 802.15.4z amendment defines 16 channels for communication for HRP UWB. 
A channel is defined by its center frequency.
UWB devices can transmit on three different bands: high band, low band, and sub-gigahertz.
For each band, there is one channel that is mandatory to support if a device supports the band.
The other channels are optional, but if two devices want to communicate with each other, they need to use the same band.
The bands, 16 channels, and their ranges, and which channels are mandatory can be found in the table  (see table \ref{Table:: UWB frequency and channel assignments}).


\begin{table}[ht!]
\centering
\begin{tabular}{|l|l|c|c|}
\hline
\textbf{Channel number} & \textbf{Center frequency (MHz)} & \textbf{HRP UWB band}       & \textbf{Mandatory}  \\ 
\hline
0                       & 499.2                           & sub-gigahertz               &  \checkmark     \\ 
\hline
1                       & 3494.4                          & \multirow{4}{*}{Low band}   &                   \\ 
\cline{1-2}\cline{4-4}
2                       & 3993.6                          &                             &                   \\ 
\cline{1-2}\cline{4-4}
3                       & 4492.8                          &                             & \checkmark      \\ 
\cline{1-2}\cline{4-4}
4                       & 3993.6                          &                             &                   \\ 
\hline
5                       & 6489.6                          & \multirow{11}{*}{High band} &                 \\ 
\cline{1-2}\cline{4-4}
6                       & 6988.8                          &                             &                   \\ 
\cline{1-2}\cline{4-4}
7                       & 6489.6                          &                             &                   \\ 
\cline{1-2}\cline{4-4}
8                       & 7488                            &                             &                   \\ 
\cline{1-2}\cline{4-4}
9                       & 7987.2                          &                             & \checkmark       \\ 
\cline{1-2}\cline{4-4}
10                      & 8486.4                          &                             &                   \\ 
\cline{1-2}\cline{4-4}
11                      & 7987.2                          &                             &                   \\ 
\cline{1-2}\cline{4-4}
12                      & 8985.6                          &                             &                   \\ 
\cline{1-2}\cline{4-4}
13                      & 9484.8                          &                             &                   \\ 
\cline{1-2}\cline{4-4}
14                      & 9984                            &                             &                   \\ 
\cline{1-2}\cline{4-4}
15                      & 9484.8                          &                             &                 \\
\hline
\end{tabular}
\caption{ HRP UWB Frequency and Channel Assignments  \cite{IEEE4-2020-7, IEEE4z}}
\label{Table:: UWB frequency and channel assignments}
\end{table}

\subsubsection{Scrambled timestamp sequence}

The 4z amendment added the option to include a scrambled timestamp sequence (STS) into the frame.
The STS is a cyphered sequence that includes the timestamp and is used for ranging.
It is meant to increase the accuracy and integrity of the raging results.
Before transmitting, the receiver and sender exchange a randomly generated key.
The key is then used to encrypt the timestamp using the advanced encryption standard (AES) with 128 bits.
This ensured that the signal had not been intercepted and changed to manipulate the ranging result.
Devices that support STS are called HRP-enhanced ranging capable
devices (HRP-ERDEV).

\subsubsection{Pulse Repetition Frequency}
\label{sec:pule repetition frequency}
The pulse repetition frequency (PRF) is the frequency at wich bursts are sent by the transmitter.
The mean PRF is the average PRF while sending the payload (power switching service data unit PSDU) \cite{hsu_2021}.
The higher the mean PRF, the shorter the airtime of each frame, and it allows for faster communication.
HRP-ERDEV uses a different mean PRF than general devices.
They can work in Base pulse repetition frequency (BPRF) operating at mean PRF 64 MHz or in higher pulse repetition frequency (HPRF) mode operating above BPRF (Table \ref{f:mean PRF}).

\begin{table}[ht!]
\centering
\begin{tabular}{|l|l|l|} 
\hline
\textbf{Standard}          & \textbf{HRP UWB mode} & \textbf{mean PRF}            \\ 
\hline
802.15.4                   & Non HRP ERDEV         & 3.9 MHz, 15.6 MHz, 62.4 MHz  \\ 
\hline
\multirow{2}{*}{802.15.4z} & HRP-ERDEV BPRF        & 62.4 MHz                     \\ 
\cline{2-3}
                           & HRP-ERDEV HPRF        & 124.8 MHz, 249.6 MHz         \\
\hline
\end{tabular}
\caption{HRP UWB Mean PRF (Based on IEEE 802.15.4 and IEEE 802.15.4z, \cite{IEEE4-2020-7, IEEE4z})}
\label{f:mean PRF}
\end{table} %cite euse eigeni bricht

\subsubsection{Symbol Encoding}
UWB sends symbols by transmitting a burst of pulses that encode the symbol.
Since the pulses have clean edges, the arrival time can be measured precisely.
This leads to the burst having two ways to carry information( \cite{QorvoGettingBacktoBasics}):
\begin{itemize}
  \item Binary phase-shift keying (BPSK: Encoding zeros and ones shifting the phases of the pulses so the burst beak for one has an opposite amplitude to the other. 
Figure \ref{f:UWB_signal_description} shows the single 101 binary phase-shift keyed. 
Each bit is set twice, to detect problems with transmission.
  \item Burst position modulation (BPM): Changing the timing of the burst so it falls into a different time slot inside of the possible burst position.	
Figure \ref{f:symbol structure} shows how the burst can be placed in a BPM interval. 
The burst cannot be placed in the guard interval. 
The guard exists to minimize inter-symbol interference from the
signals taking multiple paths.
\end{itemize}

\begin{figure}[ht!]
	\centering
	\includegraphics[width=\linewidth]{graphics/HRP_UWB_PHY_symbol_structure.jpg}
	\caption{HRP UWB PHY Symbol Structure \cite{hsu_2021}}
	\label{f:symbol structure}
\end{figure}

One or both of these encoding strategies can be used in a uwb transmission.
The position of the pulses inside of the burst (see figure \ref{f:UWB_signal_description})  relative to each other can be used to detect the presence of multipath effects and adjust for them. 
Using this, precise arrival times for the whole signal can be calculated.

\begin{figure}[ht!]
	\centering
	\includegraphics[width=\linewidth]{graphics/uwb_signal_tramsmission.png}
	\caption{UWB signal transmission byte encoding, \cite{QorvoGettingBacktoBasics}}
	\label{f:UWB_signal_description}
\end{figure}

Non-HRP ERDEV uses BPM and BPSK.
Some HRP-ERDEV can use only BPSK, which uses a higher PRF and, therefore, reduces airtime.








\subsubsection{PHY Frame}
Figure \ref{f:PPDU general} shows a schematic view of a PHY frame as defined by the IEEE 802.15.4 standard.
The Synchronization header (SHR) contains the information needed to detect the signal and adjust to its parameters.
The PHY header contains meta-information about the payload and its encoding.
The PHY payload contains the data to be sent, namly the MAC frame.

\begin{figure}[!ht]
\centering
\includegraphics[width=\linewidth]{graphics/Schematic_view_PPDU.png}
\caption{Schematic view of a PHY frame defined by IEEE 802.15.4 \cite{IEEE4-2020-7}}
\label{f:PPDU general}
\end{figure}


Figure \ref{f:SHR field} shows the synchronization header, consisting of two parts.
The SYNC section is detectable by the receiver and informs it that a transmission has started.
Depending on the predefined mode, pulses of different lengths are used.
The sequence of pulses specifies a set of channels that can be used for communication.
The preamble can also be used to identify a PAN coordinator.

The SHR ends with the Start of Frame Delimiter (SFD).
It indicates that the synchronization has ended, and the coming signals will be data, starting with the PHY header. 
It also contains a timestamp, which can be used for ranging using the time difference of arrival (ToA), see Section \ref{ss:two_way_ranging}


\begin{figure}[!ht]
\centering
\includegraphics[width=\linewidth]{graphics/SHR_field_structure.png}
\caption{SHR Field Structure \cite{IEEE4-2020-7}}
\label{f:SHR field}
\end{figure}	

The PHY header contains all the information needed to read the PHY payload (see Figure \ref{f:PHR general}).
The first bit defines the data rate used during the payload transfer (see section \ref{sec: pulse repetition frequency}).
The next seven bits define the length of the frame, with a frame length of a maximum of 128 bytes.
The 10th bit shows if ranging will be used with this frame.
The next bit is reserved.
Bits 11 and 12 define the preamble duration. It specifies how many repetitions are used, which can range from 16 to 4096.
The last 6 bits are single error correct, double error detect (SECDED) bits that form a Hammock block and can be used to correct single-bit errors and detect, but not fix double-bit errors.

The last part of the PHY frame is the PHY payload (PSDU).
This contains the MAC frame, as defined in section \ref{sec:UWB MAC}.

\begin{figure}[ht!]
\centering
\includegraphics[width=\linewidth]{graphics/PHR_field_format_4.png}
\caption{General PHR Field Format \cite{IEEE4-2020-7}}
\label{f:PHR general}
\end{figure}

The 802.15.4z amendment contains optional changes to the PHY frame format if the participating devices are HRP-ERDEV devices.
Figure \ref{f:HRP-erdev frame} shows the newly allowed structures for a UWB frame.
Configuration 1 is equivalent to the already existing PHY frame.
The others additionally contain a scrambled time stamp.
This can be placed in different places after the SHR.
Since UWB can also be used only for ranging without transmitting a message, configuration three only contains the SHR and STS, without a payload.

\begin{figure}[ht!]
\centering
\includegraphics[width=\linewidth]{graphics/HRP_ERDEV_frame_structures.jpg}
\caption{HRP-ERDEV Frame Structures \cite{hsu_2021}}
\label{f:HRP-erdev frame}
\end{figure}

Additionally, the PHR can be formatted differently (see figure \ref{f:PHR 4z}. 
The reserved field and preamble duration are removed to make more space for the frame length. 
This allows more data to be sent in one frame, increasing the throughput of the UWB communication.

\begin{figure}[ht!]
\centering
\includegraphics[width=\linewidth]{graphics/HRP_ERDEV_HPRF_mode_PHR.png}
\caption{PHR Field Format for HRP-ERDEV in HPRF Mode \cite{IEEE4z}}
\label{f:PHR 4z}
\end{figure}


\subsection{Two-way ranging}
\label{ss:two_way_ranging}
The IEEE 802.15.4z UWB standard describes two ranging methods: single-sided two-way ranging (SS-TWR) and Double-sided two-way ranging (DS-TWR).
In both instances, the distance measurement is done by calculating the time of flight (ToF) of a signal sent between two devices using timestamps and multiplying it by the speed of light. 
In this Section, both SS-TWR and DS-TWR will be discussed.
Two-way ranging(TWR) refers to DS-TWR in all other parts of the Thesis.

\textbf{Single-sided two-way ranging (SS-TWR)}:
During  SS-TWR, one device sends a message to a second and measures the round-trip time (see figure \ref{f:ss_twr}).
Device A sends a message to B and records a timestamp when the message was sent, $T_{A0}$.
When device B receives the response, it also records a timestamp, $T_{B0}$.
After some delay, device B will send a response to A that contains $T_{B0}$ and the current timestamp $T_{B1}$.
On receiving the response, Device A records its timestamp, $T_{A1}$.
The round trip time $T_{round}$ can now be calculated using the timestamps from A:
\begin{equation}
	\mbox{$T_{round}$} =
	\mbox{$T_{A1} - T_{A0}$}
\end{equation}
The reply delay $T_{reply}$ is calculated using the timestamps from B:
\begin{equation}
	\mbox{$T_{reply}$} =
	\mbox{$T_{B1} - T_{B0}$}
\end{equation}

The ToF can be calculated by subtracting these values.
Since the messages were sent twice the same distance, the ToF must be halved before multiplying it with the speed of light to get the distance.
\begin{equation}
	\mbox{$distance$} =
	\mbox{$(\frac{1}{2}\cdot T_{round}-T_{reply}) \cdot c_{air}$}
\end{equation}

\begin{figure}[ht!]
\centering
\includegraphics[width=\linewidth]{graphics/schematics/SingleSidedTwoWayRanging.png}
\caption{timeline of single-sided two-way ranging (SS-TWR)\cite{IEEE4z}}
\label{f:ss_twr}
\end{figure}

\textbf{Double-sided two-way ranging (DS-TWR)}:
DS-TWR involves both devices A and B performing an SS-TWR and calculating the average between the results.
Figure \ref{f:ds_twr} shows the two separate ranging sessions.
Their result can then be combined to the average ToF for a single message:
\begin{equation}
	\mbox{$T_{prop}$} =
	\mbox{$\frac {T_{Round1}\cdot T_{Round2}-T_{Reply1}\cdot T_{Reply2}}{T_{Round1}+T_{Round2}+T_{Reply1}+T_{Reply2}}$}
\end{equation}
\begin{equation}
	\mbox{$distance$} =
	\mbox{$T_{prop} \cdot c_{air}$}
\end{equation}

The two ranging sessions can have one message overlapping.
Figure \ref{f:ds_twr_3} shows the timeline of an overlapping DS-TWR that only requires three messages.


\begin{figure}[ht!]
\centering
\includegraphics[width=\linewidth]{graphics/schematics/TwoSidedTwoWayRanging.PNG}
\caption{timeline of double-sided two-way ranging (DS-TWR)\cite{IEEE4z}}
\label{f:ds_twr}
\end{figure}

\begin{figure}[ht!]
\centering
\includegraphics[width=\linewidth]{graphics/schematics/TwoSidedRangingThreeMessages.PNG}
\caption{timeline of double-sided two-way ranging (SS-TWR) with three messages\cite{IEEE4z}}
\label{f:ds_twr_3}
\end{figure}


\subsection{K-Connected Graphs}
\label{ss:k_connected_explained}
In order to build a working positional model based on distance measurement, some background in graph theory is required.
The k-connected subgraph of a graph is a subgraph where it takes at least k removals of vertices to create two isolated subgraphs.
A graph $(V,E)$ can have multiple k-connected subgraphs.
They build the set $S_{(V,E)}$.\\
A minimally weighted k-connected subgraph of a weighted graph $S_{(V,E,w)}$, is a k-connected subgraph $(V',E',w') \in S_{(V,E)}$ that has the smallest sum of weights of all k-connected subgraphs.\\
A metric graph is a weighted graph that satisfies the triangle condition.
Meaning for any three edges $e_{AB},e_{BC},e_{CA} \in E$  that connect three verteces $A, B, C \in V$, it holds that $e_{AB} + e_{BC} \geq e_{CA}$.


Finding minimally weighted k-connected subgraphs is an NP-hard problem.
Kahn et al. \cite{khan2007simple} developed distributed approximation algorithm that finds weighted k-connected subgraphs in metric graphs.
It gives an approximation ratio of $\mathcal{O}(k\cdot log(n))$.
This means that the aproximated solution $w^{apr}$ and the optimal sollution $w^{opt}$ fullfill $w^{apr} < k\cdot log(n)\cdot w^{opt})$.
The algorithm puts all vertices in an order, which is determined randomly, and assigns each vertex a rank based on the order.
Each vertex then removes all edges, except for the k lowest weighted ones that connect it to a vertex with a higher rank.\\
There are more precise approximation algorithms to find minimally weighted k-connected subgraphs \cite{kortsarz2003approximating, kortsarz2004approximation}. However, they are centralized, meaning the graph has to be known in its entirety by one actor.


\section{Related Work} % application of concepts
\label{s:related_works}

This Section presents on overview of the literature relevant to this thesis.
This includes IoT systems used in the art world, sensor networks, and wireless ranging, showing the current practices and current state of research.


\subsection{Artwork Tracking}


Since art preservation is an old field and temperature, humidity, light, and vibrations have been known to be detrimental to most artworks, especially paintings, most research in this direction is older than 20 years \cite{mecklenburg1991mechanical, michalski1991paintings, saunders2004effect}.
Still, the invention of new technologies, such as pattern recognition using artificial intelligence, an improvement on existing tools like infrared imaging, and an active need for solutions have kept the research into artwork preservation an active field \cite{borg2020application, schito2017integrated}.
One such new technologies are sensor networks, which have become widespread in the field of art preservation \cite{shah2016customized}.


Artwork tracking during transportation has not been a significant focus in academia.
The most relevant related research was done by Fort et al. \cite{landi2022iot}.
They developed a low-cost, low-powered sensor node to track the temperature, humidity, pressure, and vibrations of artwork and wooden structures.
The sensor node would then report its findings to a remote server.
They confirmed the validity of their sensor in a series of experiments performed in a static building.
They also presented a theoretical framework for their sensor to be used in a transportation scenario, but they do not report implementing or testing this system.
Their sensor used an accelerometer to detect vibrations, and the Bosch BME280 sensor to detect pressure, temperature, and humidity.
Their sensors did not build a network and were not queried but reported their findings directly to either a Wlan router or a BLE-capable smart device.
Fort et al.'s research showes the value of low-cost sensors in detecting threats to artwork.

\subsection{Sensor Networks}

Wireless Sensor Networks (WSN) have become a central aspect if IoT.
Researchers have tried to focus on the most prevalent problems arising from the development of NSWs, mainly power management, security and privacy, data integrity, and availability \cite{gulati2022review}.


\cite{garg2021healthcare} researched WSNs outside of the controlled environment of a house. They propose a WSN that can track the vitals of mountaineers and call for help when measurements have dangerous values. 
They used an Arduino Mega board equipped with a radio transceiver, using LoRa with a star topology.


\cite{jones2021wireless} created a WSN of NRF24101 board intended to monitor linear infrastructures like deepsea wires, using radio and Wi-Fi for communication. Using deep sleep, they were able to optimize energy usage, so the sensor is predicted to last five years on battery.


\cite{spandonidis2022evaluation} used an accelerometer to detect pipeline vibrations to discover leaks. 
They used a narrowband connection for communication and GPS for localization.
Their sensors could query each other for data, to provide a more complete image of the situation.


\subsection{Wireless ranging}

\cite{li2024indoor} made an overview of publications involving positioning systems for industrial settings. 
They looked at the positioning systems in papers using RFID, BLE, UWB, Wi-Fi, and ZigBee. They found that UWB consistently reported the highest accuracy of these methods.
UWB was the least affected by multipath effects, although it was still the most common issue with this technology.


Early research on ranging using UWB was done by Gezici et al. \cite{gezici2008survey, gezici2005localization}.
These papers gave an overview of the different positioning systems for UWB, angle of arrival, received signal strength, time of arrival, and time difference of arrival.
Time of arrival and time difference of arrival were studied further in these publications, presenting error sources and mitigation tools.


Early research focused on the augmentation of UWB-ranging methods.
\cite{venkatesh2007nlos} proposed using integer programs for mitigating the error for ranging without line-of-sight.
\cite{guvencc2007nlos} tried to solve the same issue by using methods based on the statistics of multipath effects.
BiasSub and BiasRed were proposed to reduce the bias in time difference of arrival by applying of a well-known algebraic explicit solution for source localization  \cite{ho2012bias}.
\cite{fan2017performance} improved UWB ranging by eliminating random error. They did this by pre-filtering, using an anti-magnetic ring to eliminate outliers, and using the double-state adaptive Kalman filter to improve position accuracy.
Newer research has also begun incorporating neural networks into UWB positioning systems \cite{stahlke2020nlos, ridolfi2021uwb, che2020machine}.

UWB localization has been used in many applied contexts.
It has been proposed for pedestrian tracking \cite{otim2019effects}, drone flying \cite{macoir2019uwb}, robot navigation \cite{zhu2020adapted}, navigation system for visualy impared people \cite{rosiak2024effectiveness} and tracking people in buildings \cite{elbaum2024investigating}.
UWB positioning systems are particularly interesting for industrial IoT settings.
\cite{barbieri2021uwb} measured the performance of three different UWB antennas: Qorvo, Sewio, and Ubisense. They encountered many multipath-effects in such a complex environment. They mitigated this by employing a Bayesian filtering method.
\cite{belli2024cloud} used UWB positioning, in combination with Real-time kinematic positioning, to track workers while monitoring the factory. The goal was to trigger an alarm if a dangerous situation occurred.


